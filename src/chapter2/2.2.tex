\documentclass[11pt]{article}
\usepackage{amssymb}
\usepackage{tipa}
\usepackage{amsmath}

\newcommand{\then}{\rightarrow} 
\newcommand{\bicond}{\leftrightarrow}

\title{\textbf{How to Prove It} \\ {\Large\itshape Daniel J. Velleman} \\ {\Large\itshape Chapter 1: Operations on Sets}}

\author{\textbf{Nathaniel Curnick} \\ \textit{Textbook Solutions}}

\date{}

%----------------------------------------------------------------------------------------

\begin{document}

\maketitle

\section*{Exercise 1}

Negate these statements and then reexpress the results as equivalent positive
statements

\noindent (a) Everyone who is majoring in math has a friend who needs help with 
his or her homework

$$\exists x (M(x) \wedge \forall y (F(y, x) \then \neg H(y)))$$

Where $M(x)$ means ``$x$ is majoring in math'',
$F(y,x)$ means ``$y$ is friends with $x$'',
and $H(y)$ means ``$y$ needs help with their homework''.

There is a math major whos friends don't need help with their homework

\noindent (b) Everyone has a roommate who dislikes everyone

$$\exists x \forall y (R(x, y) \then \exists z L(y, z))$$

Where $R(x, y)$ means ``$x$ and $y$ are roommates'' and 
$L(y, z)$ means ``$y$ likes $z$''.

There is someone who all of their roommates like someone

\noindent (c) $A \cup B \subseteq C \setminus D$

$$\forall x (x \in (A \cup B) \then x \in (C \setminus D))$$

$$\forall x (x \in A \vee x \in B \then x \in C \wedge x \notin D)$$

$$\neg \forall x (x \in A \vee x \in B \then x \in C \wedge x \notin D)$$

$$\exists x \neg (\neg(x \in A \vee x \in B) \vee x \in C \wedge x \notin D)$$

$$\exists x (x \in A \vee x \in B \wedge x \notin C \vee x \in D)$$

\noindent (d) $\exists x \forall y (y > x \then \exists z (z^2 + 5z = y))$

$$\neg \exists x \forall y (y > x \then \exists z(z^2 + 5z = y))$$

$$\forall x \neg \forall y (y > x \then \exists z (z^2 + 5z = y))$$

$$\forall x \exists y \neg (y > x \then \exists z (z^2 + 5z = y))$$

$$\forall x \exists y \neg (\neg (y > x) \vee \exists z (z^2 + 5z = y))$$

$$\forall x \exists y (y > x \wedge \neg \exists z (z^2 + 5z = y))$$

$$\forall x \exists y (y > x \wedge \forall z (z^2 + 5z \neq y))$$

\section*{Exercise 2}

Negate these statements and then reexpress the results as equivalent positive 
statements

\noindent (a) There is someone in the freshman class who doesn't have a roommate

$$\forall x (F(x) \then \exists y R(x,y))$$

Where $F(x)$ means ``$x$ is a freshman'' and $R(x,y)$ means ``$x$ and $y$ are 
roommates''

\noindent (b) Everyone likes someone, but no one likes everyone

$$\exists x \forall y \neg L(x, y) \vee \exists z \forall a L(z, a)$$

Where $L(x,y)$ means ``$x$ likes $y$''

Either someone doesn't like anyone or likes everyone

\noindent (c) $\forall a \in A \exists b \in B (a \in C \bicond b \in C)$

TODO

\noindent(d) $\forall y > 0 \exists x (ax^2 + bx + c = y)$

TODO

\section*{Exercise 3}

Are these statements true or false? The universe of discourse is $\mathbb{N}$

\noindent (a) $\forall x (x < 7 \then \exists a \exists b \exists c (a^2 + b^2 + c^2 = x))$

True, if we make a table for all possible $x$ then we find 

\begin{center}
    \begin{tabular}{ c c c c c }
    $x$ & $a$ & $b$ & $c$ & $a^2 + b^2 + c^2$ \\ 
    0 & 0 & 0 & 0 & 0\\  
    1 & 1 & 0 & 0 & 1\\
    2 & 1 & 1 & 0 & 2\\  
    3 & 1 & 1 & 1 & 3\\
    4 & 2 & 0 & 0 & 4\\
    5 & 2 & 1 & 0 & 5\\
    6 & 2 & 1 & 1 & 6
    \end{tabular}
\end{center}

\noindent (b) $\exists! x(x^2 + 3 = 4x)$

$$x^2 - 4x + 3 = 0$$

$$(x-1)(x-3)$$

$$x = 1, x = 3$$

Therefore, false 

\noindent (c) $\exists! x(x^2 = 4x + 5)$

$$x^2 - 4x + 5 = 0$$

$$x = 2 \pm i$$

Therefore, false 

\noindent (d) $\exists x \exists y (x^2 = 4x + 5 \wedge y^2 = 4y + 5)$

$$x^2 - 4x - 5 = 0$$

$$(x-5)(x+1)$$

$$x = 5, x = -1$$

Therefore, true 

\section*{Exercise 4}

Show that the second quantifier negation law, which says that 
$\neg \forall x P(x)$ is equivalent to $\exists x \neg P(x)$, can be derived
from the first, which says that $\neg \exists x P(x)$ is equivalent to 
$\forall x \neg P(x)$

Show: $\neg \forall x P(x) \equiv \exists x \neg P(x)$

From: $\neg \exists x P(x) \equiv \forall x \neg P(x)$

$$\neg \exists x \neg P(x) = \forall x \neg \neg P(x)$$

$$\neg \exists x \neg P(x) = \forall x P(x)$$

$$\neg \neg \exists x \neg P(x) = \neg \forall x P(x)$$

$$\exists x \neg P(x) = \neg \forall x P(x)$$

\section*{Exercise 5}

Show that $\neg \exists x \in A P(x)$ is equivalent to 
$\forall x \in A \neg P(x)$

$$\neg \exists x (x \in A \wedge P(x))$$

$$\forall x \neg (x \in A \wedge P(x))$$

$$\forall x (x \notin A \vee \neg P(x))$$

$$\forall x (x \in A \then \neg P(x))$$

$$\forall x \notin A \neg P(x)$$

\section*{Exercise 6}

Show that the existential quantifier distributes over disjunction. In other
words, show hat $\exists x (P(x) \vee Q(x))$ is equivalent to 
$\exists x P(x) \vee \exists x Q(x)$.

$$\exists x (P(x) \vee Q(x))$$

$$\neg \neg \exists x (P(x) \vee Q(x))$$

$$\neg \forall x \neg (P(x) \vee Q(x))$$

$$\neg \forall x (\neg P(x) \wedge \neg Q(x))$$

$$\neg \forall x \neg P(x) \wedge \neg \forall x \neg Q(x)$$

$$\neg \exists P(x) \wedge \neg \exists x Q(x)$$

$$\exists x P(x) \vee \exists x Q(x)$$

\section*{Exercise 7} 

Show that $\exists x (P(x) \then Q(x))$ is equivalent to 
$\forall x P(x) \then \exists x Q(x)$

$$\exists x (\neg P(x) \vee Q(x))$$

$$\exists x \neg P(x) \vee \exists x Q(x)$$

$$\neg \forall x P(x) \vee \exists x Q(x)$$

$$\forall x P(x) \then \exists x Q(x)$$

\section*{Exercise 8}

Show that $(\forall x \in A P(x)) \wedge (\forall x \in B P(x))$ is equivalent 
to $\forall x \in (A \cup B) P(x)$

$$\forall x (x \in A \then P(x)) \wedge \forall x (x \in B \then P(x))$$

$$\forall x ((x x \in A \then P(x)) \wedge (x \in B \then P(x)))$$

$$\forall x ((x \notin A \vee P(x)) \wedge (x \notin B \vee P(x)))$$

$$\forall x ((x \notin A \wedge x \notin B) \vee P(x))$$

$$\forall x (\neg (x \in A \vee x \in B) \vee P(x))$$

$$\forall x ((x \in A \vee x \in B) \then P(x))$$

$$\forall x (x \in (A \cup B) \then P(x))$$

$$\forall x \in (A \cup B) P(x)$$

\section*{Exercise 9}

Is $\forall x (P(x) \vee Q(x))$ equivalent to 
$\forall x P(x) \vee \forall x Q(x)$?

No 

Consider this:
$P(x)$ means ``bright eyed'' and $Q(x)$ means ``bushy tailed''

For $\forall x (P(x) \vee Q(x))$ it would mean ``everyone is either bright eyed
or bushy tailed'' i.e. each person is individually bright eyed or bushy tailed 

For $\forall x P(x) \vee \forall x Q(x)$ it would mean ``Everyone is either
bright eyed or everyons is bushy tailed'' i.e. everyone belongs to the set of 
bright eyed people or everyone belongs to the set of bushy tailed people 

\section*{Exercise 10}

\noindent (a) Show that $\exists x \in A P(x) \vee \exists x \in B P(x)$ is 
equivalent to $\exists x \in (A \cup B) P(x)$

$$\exists x (x \in A \then P(x) \vee \exists x (x \in B \then P(x)))$$

$$\exists x (x \in A \then P(x) \vee x \in B \then P(x))$$

$$\exists x ((x \notin A \vee P(x)) \vee (x \notin B \vee P(x)))$$

$$\exists x ((x \in A \wedge x \in B) \then P(x))$$

$$\exists x (x \in (A \cap B) \then P(x))$$

$$\exists x \in (A \cap B) P(x)$$

\noindent (b) Is $\exists x \in A P(x) \wedge \exists x \in B P(x)$ equivalent 
to $\exists x \in (A \cap B) P(x)$? Explain

No, since $\exists$ is not distributive over $\wedge$

\section*{Exercise 11}

Show that the statements $A \subseteq B$ and $A \setminus B = \emptyset$ are 
equivalent by writing each in logical symbols and then showing that the 
resulting formulas are equivalent 

$$A \subseteq B = \forall x (x \in A \then x \in B)$$

$$A \setminus B = \emptyset = \neg \exists x (x \in A \wedge x \notin B)$$

$$\forall x (x \in A \then x \in B)$$

$$\neg \exists x \neg (x \notin A \vee x \in B)$$

$$\neg \exists x (x \in A \wedge x \notin B)$$

\section*{Exercise 12}

Show that the statements $C \subseteq A \cup B$ and $C \setminus A \subseteq B$
are equivalent by writing each in logical symbols and then showing that the 
resulting formulas are equivalent 

$$C \subseteq A \cup B = \forall x (x \in C \then x \in (A \cup B))$$

$$\forall x (x \in C \then (x \in A \vee x \in B))$$

$$\forall x (x \notin C \vee (x \in A \vee x \in B))$$

$$C \setminus A \subseteq B = \forall x (x \in (C \setminus A) \then x \in B)$$

$$\forall x (x \in (C \setminus A) \then x \in B)$$

$$\forall x ((x \in C \wedge x \notin A) \then x \in B)$$

$$\forall x (\neg (x \in C \wedge x \notin A) \vee x \in B)$$

$$\forall x (x \notin C \vee x \in A \vee x \in B)$$

\section*{Exercise 13}

\noindent (a) Show that the statements $A \subseteq B$ and $A \cup B = B$ are 
equivalent by writing each in logical symbols and then showing that the 
resulting formulas are equivalent

$$A \subseteq B = \forall x (x \in A \then x \in B)$$

$$A \cup B = B \equiv \forall x ((x \in A \vee x \in B) \then x \in B)$$

$$\forall x (\neg (x \in A \vee x \in B) \vee x \in B)$$

$$\forall x ((x \notin A \wedge x \notin B) \vee x \in B)$$

$$\forall x (x \notin A \vee x \in B \wedge x \notin B \vee x \in B)$$

$$\forall x (x \notin A \vee x \in B)$$

$$\forall x (\neg (x \in A) \vee x \in B)$$

$$\forall x (x \in A \then x \in B)$$

\noindent (b) Show that the statements $A \subseteq B$ and $A \cap B = A$ are 
equivalent

TODO

\section*{Exercise 14}

Show that the statements $A \subseteq B$ and $A \cap B = A$ are equivalent

TODO

\section*{Exercise 15}

Let $T(x, y)$ mean ``$x$ is a teacher of $y$''. What do the following statements
mean? Under what circumstances would each one be true? Are any of them
equivalent to each other?

\noindent (a) $\exists! y T(x,y)$

$x$ teaches only one person

\noindent (b) $\exists x \exists! y T(x,y)$

There is at least one teacher who only teaches one student 

\noindent (c) $\exists! x \exists y T(x,y)$

There is at least one student who has one teacher

\noindent (d) $\exists y \exists! x T(x, y)$

Equivalent to (c)

\noindent (e) $\exists! x \exists! y T(x, y)$

There is exactly one student who has exactly one teacher 

\noindent (f) TODO

\end{document}