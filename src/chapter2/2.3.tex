\documentclass[11pt]{article}
\usepackage{amssymb}
\usepackage{tipa}
\usepackage{amsmath}
\usepackage[scr]{rsfso}


\newcommand{\then}{\rightarrow} 
\newcommand{\bicond}{\leftrightarrow}
\newcommand{\powerset}[1]{\mathscr{P}(#1)}
\newcommand{\family}{\mathcal{F}}

\title{\textbf{How to Prove It} \\ {\Large\itshape Daniel J. Velleman} \\ {\Large\itshape Chapter 1: Operations on Sets}}

\author{\textbf{Nathaniel Curnick} \\ \textit{Textbook Solutions}}

\date{}

%----------------------------------------------------------------------------------------

\begin{document}

\maketitle

\section*{Exercise 1}

Analyse the logical forms of the following statements. You may use the symbols
$\in$, $\notin$, $=$, $\neq$, $\wedge$, $\vee$, $\then$, $\bicond$, $\forall$,
$\exists$ in your answers, but not $\subseteq$, $\nsubseteq$, $\mathscr{P}$, 
$\cup$, $\cap$, $\setminus$, $\{$, $\}$, or $\neg$. (Thus, you must write out the
definitions of some set theory notation, and you must use equivalences to get 
rid of any occurances of $\neg$).

\noindent (a) $\family \subseteq \powerset{A}$

$$\forall x (x \in \family \then \forall y (y \in x \then y \in A))$$

\noindent (b) $A \subseteq (2n + 1 | n \in \mathbb{N})$

$$\forall x (x \in A \then \exists n \in \mathbb{N} (x = 2n + 1))$$

\noindent (c) $\{n^2 + n + 1 | n \in \mathbb{N}\} \subseteq 
\{2n + 1 | n \in \mathbb{N}\}$

$$\forall x (\exists x \in \mathbb{N} (x = 2n + 1) 
\then \exists x \in \mathbb{N} (2n + 1))$$

\noindent (d) $\powerset{\bigcup_{i \in I} A_i} \nsubseteq 
\bigcup_{i \in I} \powerset{A_i}$

TODO

\section*{Exercise 2}

Analyse the logical forms of the following statements. You may use the symbols
$\in$, $\notin$, $=$, $\neq$, $\wedge$, $\vee$, $\then$, $\bicond$, $\forall$,
$\exists$ in your answers, but not $\subseteq$, $\nsubseteq$, $\mathscr{P}$, 
$\cup$, $\cap$, $\backslash$, $\{$, $\}$, or $\neg$. (Thus, you must write out the
definitions of some set theory notation, and you must use equivalences to get 
rid of any occurances of $\neg$).

\noindent (a) $x \in \bigcup \family \setminus \bigcup \mathcal{G}$

$$x \in \bigcup \family \wedge x \notin \bigcup \mathcal{G}$$

$$\exists A \in \family (x \in A) \wedge \forall A \in \mathcal{G} (x \notin A)$$

\noindent (b) $\{x \in B | x \notin C\} \in \powerset{A}$

$$\forall x ((x \in B \wedge x \notin C) \then x \notin \powerset{A})$$

$$\forall x ((x \in B \wedge x \notin C) \then x \in A)$$

\noindent (c) $x \in \bigcap_{i \in I} (A_i \cup B_i)$

$$\forall i \in I (x \in A_i \vee x \in B_i)$$

\noindent (d) $x \in (\bigcap_{i \in I} A_i) \cup (\bigcap_{i \in I} B_i)$

$$\forall i \in I (x \in A_i) \vee \forall i \in I (x \in B_i)$$

\section*{Exercise 3}

We've seen that $\powerset{\emptyset} = \{\emptyset\}$, and 
$\{\emptyset\} = \emptyset$. What is $\powerset{\{\emptyset\}}$?

$$\powerset{\{\emptyset\}} = \{\emptyset, \{\emptyset\}\}$$

\section*{Exercise 4}

Suppose $\family = \{ \{\text{red, green, blue}\}, 
\{\text{orange, red, blue}\}, \{\text{purple, red, green, blue}\} \}$. Find 
$\bigcap \family$ and $\bigcup \family$.

$$\bigcap \family = \{\text{red, blue}\}$$

$$\bigcup \family = \{\text{red, green, blue, orange, purple}\}$$

\section*{Exercise 5}

Suppose $\family = \{ \{3, 7, 12\}, \{5, 7, 16\}, \{5, 12, 23\} \}$. Find 
$\bigcap \family$ and $\bigcup \family$.

$$\bigcap \family = \{\} = \emptyset$$

$$\bigcup \family = \{3,7,12,5,16,23\}$$

\section*{Exercise 6}

Let $I = \{2, 3, 4, 5\}$, and for each $i \in I$ let $A_i = \{i, i+1, i-1, 2i\}$

\noindent (a) List the elements of all the sets $A_i$, for $i \in I$

$$A = \{ \{2,3,1,4\}, \{3,4,2,6\}, \{4,5,3,8\}, \{5,6,4,10\} \}$$

\noindent (b) Find $\bigcap_{i \in I} A_i$ and $\bigcup_{i \in I} A_i$

$$\bigcup A = \{2,3,1,4,6,5,8,10\}$$

$$\bigcap A = \{4\}$$

\section*{Exercise 8}

Let $I = \{2,3\}$ and for each $i \in I$ let $A_i = \{i. 2i\}$ 
and $B_i = \{i, i+1\}$

\noindent (a) List the elements of the sets $A_i$ and $B_i$ for $i \in I$

$$A = \{ \{2,4\}, \{3,6\} \}$$

$$B = \{ \{2,3\}, \{3,4\} \}$$

\noindent (b) Find $\bigcap_{i \in I} (A_i \cup B_i)$ and 
$(\bigcap_{i \in I} A_i) \cup (\bigcap_{i \in I} B_i)$. Are they the same?

$$A_i \cup B_i = \{\{2,3,4\}, \{3,4,6\}\}$$

$$\bigcap_{i \in I} (A_i \cup B_i) = \{3,4\}$$

$$\bigcap_{i \in I} A_i = \{\}$$

$$\bigcap_{i \in I} B_i = \{3\}$$

$$(\bigcap_{i \in I} A_i) \cup (\bigcap_{i \in I} B_i) = \{3\}$$

\noindent (c) In parts (c) and (d) of exercise 2 you analysed the statements 
$x \in \bigcap_{i \in I} (A_i \cup B_i)$ and 
$x \in (\bigcap_{i \in I} A_i) \cup (\bigcap_{i \in I} B_i)$. What can you 
conclude from your answer to part (b) about whether or not these statements 
are equivalent?

Not equivalent

\section*{Exercise 9} 

\noindent (a) Analyse the logical forms of the statements 
$x \in \bigcup_{i \in I} (A_i \setminus B_i)$, 
$x \in (\bigcup_{i \in I} A_i) \setminus (\bigcup_{i \in I} B_i)$ and 
$x \in (\bigcup_{i \in I} A_i) \setminus (\bigcup_{i \in I} B_i)$. Do you think 
any of these statements are equivalent to each other?

$$x \in \bigcup_{i \in I} (A_i \setminus B_i) = 
\exists i \in I (x \in A_i \wedge \notin B_i)$$

$$x \in (\bigcup_{i \in I} A_i) \setminus (\bigcup_{i \in I} B_i) = 
\exists i \in I (x \in A_i) \wedge \neg \exists i \in I (x \in B_i)$$

$$x \in (\bigcup_{i \in I} A_i) \setminus (\bigcup_{i \in I} B_i) =
\exists i \in I (x \in A_i) \wedge \neg \forall i \in I (x \in B_i)$$

None are equivalent

\noindent (b) Let $I$, $A_i$ and $B_i$ be defined as in exercise 8. Find 
$x \in \bigcup_{i \in I} (A_i \setminus B_i)$, 
$x \in (\bigcup_{i \in I} A_i) \setminus (\bigcup_{i \in I} B_i)$ and 
$x \in (\bigcup_{i \in I} A_i) \setminus (\bigcup_{i \in I} B_i)$.
Now do you think any of the statements in part (a) are equivalent?

$$x \in \bigcup_{i \in I} (A_i \setminus B_i) = \{4, 6\}$$

$$x \in (\bigcup_{i \in I} A_i) \setminus (\bigcup_{i \in I} B_i) = \{6\}$$

$$x \in (\bigcup_{i \in I} A_i) \setminus (\bigcup_{i \in I} B_i) = \{2, 4, 6\}$$

\section*{Exercise 10}

Give an example of an index set $I$ and indexed families os sets 
$\{A_i | i \in I\}$ and $\{B_i | i \in I\}$ such that 
$\bigcup_{i \in I} (A_i \cap B_i) \neq 
(\bigcup_{i \in I} A_i) \cap (\bigcup_{i \in I} B_i)$.

$$I = \{1, 2\}$$

$$A = \{i\}$$

$$B = \{2i\}$$

$$A = \{\{1\}, \{2\}\}$$

$$B = \{\{2\}, \{4\}\}$$

$$\bigcup_{i \in I} (A_i \cap B_i) = \emptyset$$

$$(\bigcup_{i \in I} A_i) \cap (\bigcup_{i \in I} B_i) = \{2\}$$

\section*{Exercise 11}

Show that for any sets $A$ and $B$, 
$\powerset{A \cap B} = \powerset{A} \cap \powerset{B}$, by showing that the 
statements $x \in \powerset{A \cap B}$ and $x \in \powerset{A} \cap \powerset{B}$
are equivalent

$$x \in \powerset{A \cap B} = \forall y (y \in x \then (y \in A \wedge y \in B))$$

$$x \in \powerset{A} \cap \powerset{B} = 
\forall y (y \in x \then y \in A) \wedge \forall y (y \in x \then y \in B)$$

$$\forall y (y \in x \then (y \in A \wedge y \in B))$$

$$\forall y (y \in x \then y \in A \wedge y \in x \then y \in B)$$

$$\forall y (y \in x \then y \in A) \wedge \forall y (y \in x \then y \in B)$$

\section*{Exercise 12} 

Give examples of sets $A$ and $B$ for which 
$\powerset{A \cup B} = \powerset{A} \cup \powerset{B}$

$$A = \{1, 2\}, B = \{2 ,3\}$$

\section*{Exercise 13}

Verify the following identities by writing out (using logical symbols) what it 
means for an object $x$ to be an element of each set and then using logical 
equivalences

\noindent (a) $\bigcup_{i \in I} (A_i \cup B_i) = 
(\bigcup_{i \in I} A_i) \cup (\bigcup_{i \in I} B_i)$

$$\bigcup_{i \in I} (A_i \cup B_i) = 
(\bigcup_{i \in I} A_i) \cup (\bigcup_{i \in I} B_i)$$

$$\bigcup_{i \in I} (A_i \cup B_i)$$

$$\exists i \in I (x \in A_i \vee x \in B_i)$$

$$\exists i \in I (x \in A_i) \vee \exists i \in I (x \in B_i)$$

$$(\bigcup_{i \in I} A_i) \cup (\bigcup_{i \in I} B_i)$$

\noindent (b) $(\bigcap \family) \cap (\bigcap \mathcal{G}) = 
\bigcap (\family \cup \mathcal{G})$

$$\forall A \in \family (x \in A) \wedge \forall A \in \mathcal{G} (x \in A)$$

$$\forall A \in \family \cup \mathcal{G} (x \in A)$$

$$\bigcap (\family \cup \mathcal{G})$$

\noindent (c) $\bigcap_{i \in I} (A_i \setminus B_i) = 
(\bigcap_{i \in I} A_i) \setminus (\bigcup_{i \in I} B_i)$

$$\forall i \in I (x \in A_i \wedge x \notin B_i)$$

$$\forall i \in I x \in A_i \wedge \forall i \in I x \notin B_i$$

$$(\bigcap_{i \in I}) A_i \setminus (\bigcup_{i \in I} B_i)$$

\section*{Exercise 14}

Sometimes  each set in an indexed family of sets has \textit{two} indicies.
For this problem, use the following definitions:
$I = \{1,2\}, J = \{3,4\}$. For each $i \in I$ and $j \in J$, let 
$A_{i, j} = \{i, j, i+j\}$. Thus, for example, $A_{2, 3} = \{2,3,5\}$.

\noindent (a) For each $j \in J$ let 
$B_j = \bigcup_{i \in I} A_{i,j} = A_{1,j} \cup A_{2,j}$. Find $B_3$ and $B_4$.

For $j = 3$

$$A_{1,3} = \{1,3,4\}$$

$$A_{2,3} = \{2,3,5\}$$

$$B_3 = \{1,3,4,2,5\}$$

For $j = 4$

$$A_{1,4} = \{1,4,5\}$$

$$A_{2,4} = \{2,4,6\}$$

$$B_4 = \{1,4,5,2,6\}$$

\noindent (b) Find $\bigcap_{j \in J} B_j$. (Note that, replacing $B_j$ with its
definition, we could say that $\bigcap_{j \in J} B_j = 
\bigcap_{j \in J} (\bigcup_{i \in I} A_{i, j}))$

$$B_3 = \{1,3,4,2,5\}$$

$$B_4 = \{1,4,5,2,6\}$$

$$B_3 \cap B_4 = \{1,4,5,2\}$$

\noindent (c) Find $\bigcup_{i \in I} (\bigcap_{j \in J} A_{i, j})$. (Hint: You 
may want to do this in two steps, corresponding to parts (a) and (b).) Are 
$\bigcap_{j \in J} (\bigcup_{i \in I} A_{i, j})$ and 
$\bigcup_{i \in I} (\bigcap_{j \in J} A_{i, j})$ equal?

Start with $\bigcup_{i \in I} (\bigcap_{j \in J} A_{i, j})$. Define 
$\bigcap_{j \in J} A_{i, j} = A_{i, 3} \cap A_{i, 4} = C_i$

$$C_1 = \{1, 4\}$$

$$C_2 = \{2\}$$

$$C_1 \cup C_2 = \{1,2,4\}$$

Not equivalent

\noindent (d) Analyse the logical forms of the statements 
$x \in \bigcap_{j \in J} (\bigcup_{i \in I} A_{i, j})$ and 
$x \in \bigcup_{i \in I} (\bigcap_{j \in J} A_{i, j})$. Are they equivalent?

$$x \in \bigcap_{j \in J} (\bigcup_{i \in I} A_{i, j})$$

$$\forall j \in J \exists i \in I (x \in A_{i, j})$$

$$x \in \bigcup_{i \in I} (\bigcap_{j \in J} A_{i, j})$$

$$\exists i \in I \forall j \in J (x \in A_{i, j})$$

Not equivalent!

\section*{Exercise 15}

\noindent (a) Show that if $\family = \emptyset$, then the statement 
$x \in \bigcup \family$ will be false no matter what $x$ is. It follows that 
$\bigcup \emptyset = \emptyset$.

$$x \in \bigcup \family = \exists A \in \family (x \in A)$$

Since $\family = \emptyset$, then there is no set $A \in \family$.

\noindent (b) Show that if $\family = \emptyset$, then the statement 
$x \in \bigcap \family$ will be true no matter what $x$ is.

$$x \in \bigcap \family = \forall A \in \family (x \in A)$$

This is vacuously true, since for all $A$ in $\family$ is empty

\section*{Exercise 16} In Section 2.3 we saw that a set can have other sets as 
elements. When discussing sets whos elements are sets, it might seem most 
natural to consider the universe of discourse to be the set of all sets. 
However, as we will see in this problem, assuming that there is such a set leads 
to contradictions.

Suppose $U$ were the set of all sets. Note that in particular $U$ is a set, so 
we would have $U \in U$. This is not yet a contradiction; although most sets are
not elements of themselves, perhaps some sets are elements of themselves. 
But it siggests that the sets in the universe $U$ could be split into two 
categories: the unusual sets that, like $U$ itself, are elements of themselves, 
and the more typical sets that are not. Let $R$ be the set of sets in the second 
category. In other words, $R = \{A \in U | A \notin A\}$. This means that for 
any set $A$ in the universe $U$, $A$ will be an element of $R$ iff $A \notin A$.
In other words, we have $\forall A \in U (A \in R \bicond A \notin A)$.

\noindent (a) Show that applying this last fact to the set $R$ itself (in other 
words, plugging $R$ for $A$) leads to a contradiction. This contradiction was
discovered by Bertrand Russell (1872-1970) in 1901, and is known as 
\textit{Russell's Paradox}.

$$\forall A \in U (A \in R \bicond A \notin A)$$

Thus, $R \in R \bicond R \notin R$. This is clearly a contradiction.

\noindent (b) Think some more about the paradox in part (a). What do you think 
it tells us about sets?

There is no universal set of all sets


\end{document}