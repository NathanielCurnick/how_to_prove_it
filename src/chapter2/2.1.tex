\documentclass[11pt]{article}
\usepackage{amssymb}
\usepackage{tipa}
\usepackage{amsmath}

\newcommand{\then}{\rightarrow} 
\newcommand{\bicond}{\leftrightarrow}

\title{\textbf{How to Prove It} \\ {\Large\itshape Daniel J. Velleman} \\ {\Large\itshape Chapter 1: Operations on Sets}}

\author{\textbf{Nathaniel Curnick} \\ \textit{Textbook Solutions}}

\date{}

%----------------------------------------------------------------------------------------

\begin{document}

\maketitle

\section*{Exercise 1}

\noindent Analyse the logical forms of the following statements

\noindent (a) Anyone who has forgiven at least one person is a saint

$$\forall x (\exists y F(x, y) \then S(x))$$

Where $F(x, y)$ is ``$x$ has forgiven $y$'' and $S(x)$ is ``$x$ is a saint''

\noindent (b) Nobody in the calculus class is smarter than everybody in the 
discrete math class

$$\neg \exists ( C(x) \wedge \forall y (D(y) \then S(x, y)))$$

Where 
$C(x)$ means ``$x$ is in the calculus class'',
$D(x)$ means ``$x$ is in the discrete math class'',
$S(x, y)$ means ``$x$ is smarter than $y$''

\noindent (c) Everyone likes Mary, except Mary herself 

$$\forall x (\neg (x = m) \then L(x, m))$$

Where $m$ means ``Mary'' and $L(x, m)$ means ``$x$ likes $m$''

\noindent (d) Jane saw a police officer, and Roger saw him too 

$$\exists x (P(x) \wedge S(j, x)) \wedge \exists y (P(y) \wedge S(r, y))$$

Where 
$P(x)$ means ``$X$ is a police officer'',
$S(x, y)$ means ``$x$ saw $y$'',
$j$ means ``Jane'',
$r$ means ``Roger''

\noindent (e) Jane saw a police officer, and Roger saw him too 

$$\exists x (P(x) \wedge S(j, x) \wedge S(r, x))$$

\section*{Exercise 2}

\noindent Analyse the logical forms of the following statements

\noindent (a) Anyone who has bought a Rolls Royce with cash must have a rich uncle 

$$\forall x (R(x) \then U(x))$$

Where 
$R(x)$ means ``$x$ bought a Rolls Royce'',
$U(x)$ means ``$x$ has a rich uncle''

\noindent (b) If anyone in the dorm has the measles, then everyone who has a 
friend in the dorm will have to be quarantined 

$$\exists x (D(x) \wedge M(x)) \then \forall y (\exists z (F(y, z) \wedge D(z) \then Q(y))$$

Where
$D(x)$ means ``$x$ is in the dorm'',
$M(x)$ means ``$x$ has measles'',
$F(x, y)$ means ``$x$ and $y$ are friends'',
$Q(y)$ means ``$y$ must quarantine''

\noindent (c) If nobody failed the test, then everybody who got an A will tutor someone who got a D

$$\neg \exists x F(x) \then \forall y (G(y, \text{`}A\text{'}) \then \exists z (G(z, \text{`}D\text{'}) \wedge T(y, z)))$$

Where
$F(x)$ means ``$x$ failed the test'',
$G(x, \text{`}X\text{'})$ ``$x$ got grade `X' on the test'',
$T(y, z)$ means ``$y$ tutors $z$''

\noindent (d) If anyone can do it, Jones can

$$\exists x (D(x) \then D(j))$$

Where $D(x)$ means ``$x$ can do it'' and $j$ means ``Jones''

\noindent (e) If Jones can do it, then anyone can

$$D(j) \then \forall x D(x)$$

\section*{Exercise 3}

\noindent (a) Every number that is larger than $x$ is larger than $y$

$$\forall z (z > x \then z > y)$$

Free variables are $x$ and $y$

\noindent (b) For every number $a$, the equation $ax^2 + 4x - 2 = 0$ has at 
least one solution iff $a \geq -2$

$$\forall a \exists x (ax^2 + 4x - 2 = 0) \bicond a \geq -2$$

No free variables

\noindent (c) All solutions of the inequality $x^3 - 3x < 3$ are smaller than 10

$$\forall x (x^2 - 3x < 3 \then x < 10)$$

No free variables

\noindent (d) If there is a number $x$ such that $x^2 + 5x = w$ and there is a 
number $y$ such that $4 - y^2 = w$, then $w$ is strictly between -10 and 10

$$\exists x (x^2 + 5x = w \wedge \exists y (4 - y^2 = w) \then -10 \leq w \leq 10)$$

$w$ is free

\section*{Exercise 4}

\noindent Translate the following statements into idiomatic English

\noindent (a) $\forall x ((H(x) \wedge \neg \exists y M(x, y)) \then U(x))$, 
where $H(x)$ means ``$x$ is a man'', $M(x, y)$ ``$x$ is married to $y$'',
and $U(x)$ means ``$x$ is unhappy''

All unmarried men are unhappy

\noindent (b) $\exists z (P(z, x) \wedge S(z, y) \wedge W(y))$, where $P(z, x)$ 
means ``$z$ is a parent of $x$'', $S(z, y)$ means ``$z$ and $y$ are siblings'',
and $W(y)$ means ``$y$ is a woman''

$y$ is a sister of one of $x$'s parents i.e. $y$ is $x$'s aunt

\section*{Exercise 5}

\noindent Translate the following statements into idiomatic mathematical 
English

\noindent (a) $\forall x ((P(x) \wedge \neg (x = 2)) \then O(x))$ where $P(x)$ 
means ``$x$ is prime'' and $O(x)$ means ``$x$ is odd''

All the prime number thats are not 2 are odd

\noindent (b) $\exists x (P(x) \wedge \forall y (P(y) \then y \leq x))$ where 
$P(x)$ means ``$x$ is a perfect number''

There exists a perfect number such that all perfect numbers are less than or 
equal to it (i.e. there is a largest perfect number)

\section*{Exercise 6}

\noindent Translate the following statements into idiomatic mathematical 
English. Are they true or false? The universe of discourse is $\mathbb{R}$

\noindent (a) $\neg \exists x (x^2 + 2x + 3 = 0 \wedge x^2 + 2x - 3 = 0)$

$x^2 + 2x + 3 = 0$ and $x^2 + 2x - 3 = 0$ do not share a solution

Solutions to $x^2 + 2x + 3 = 0$ are $-1 + i \sqrt{2}$ or $-1 - i \sqrt{2}$. 
Solutions to $x^2 + 2x - 3 = 0$ are $-3$ and $1$. 
Therefore, this statement is true

\noindent (b) $\neg (\exists x (x^2 + 2x + 3 = 0) \wedge \exists x (x^2 + 2x - 3 = 0))$

$x^2 + 2x + 3 = 0$ and $x^2 + 2x - 3 = 0$ do not share a solution.
(This is stated differently to (a). In this one we consider all the solutions 
of $x^2 + 2x - 3 = 0$ against each solution of $x^2 + 2x + 3 = 0$ rather than 
in (a) where we compare each possible solution at the same time)

This is true, for the same reasons as (a)

\noindent (c) $\neg \exists x (x^2 + 2x + 3 = 0) \wedge \neg \exists x (x^2 + 2x - 3 = 0)$

Neither $x^2 + 2x + 3 = 0$ or $x^2 + 2x - 3 = 0$ have a solution

This statement is false, since $x^2 + 2x - 3 = 0$ has a solution. 
($x^2 + 2x + 3 = 0$ doesn't, since the universe of discourse is $\mathbb{R}$)

\section*{Exercise 7}

\noindent Are these statements true or false? The universe of discourse is the 
set of all people, and $P(x, y)$ means ``$x$ is a parent of $y$''

\noindent (a) $\exists x \forall y P(x, y)$

There exists some parent who is the parent of everyone. False, since nobody 
can be their own parent 

\noindent (b) $\forall x \exists y P(x, y)$

Everyone is a parent to someone. False, since some people do not have children.

\noindent (c) $\neg \exists x \exists y P(x, y)$

There does not exist anyone who is a parent of someone, so it is false

\noindent (d) $\exists x \neg exist y P(x, y)$

There is someone who is a parent of nobody, which is true 

\noindent (e) $\exists x \exists y \neg P(x, y)$

There exists someone who is not a parent, so true

\section*{Exercise 8}

\noindent Are these statements true or false? The universe of discourse is $\mathbb{N}$

\noindent (a) $\forall x \exists y (2x - y = 0)$

For all $x$, there exists a $y$ such that $2x - y = 0$, which is true 

\noindent (b) $\exists y \forall x (2x - y = 0)$

False - we can not use \textit{any} $x$ for a given $y$

\noindent (c) $\forall x \exists y (x - 2y = 0)$

This is false. Consider $x = 1$, then $y = 0.5$ but we are in $\mathbb{N}$, so 
not fractional numbers 

\noindent (d) $\forall x (x < 10 \then \forall y (y < x \then y < 9))$

True 

\noindent (e) $\exists y \exists z (y + z = 100)$

True 

\noindent (f) $\forall x \exists y (y > x \wedge \exists z (y + z = 100))$

False, since if $x > 100$ then $z < 0$ and we do not have negative numbers 
(domain is $\mathbb{N}$)

\section*{Exercise 9}

\noindent Same as exercise 8 but with $\mathbb{R}$ as the universe of discourse

\noindent (a)

True 

\noindent (b) 

False 

\noindent (c) 

True 

\noindent (d) 

False e.g. $x=9.9$, $y = 9.8$

\noindent (e) 

True 

\noindent (f)

True 

\section*{Exercise 10}

\noindent Same as exercise 8. but with $\mathbf{Z}$ as the universe of discourse 

\noindent (a)

True 

\noindent (b) 

True 

\noindent (c) 

False 

\noindent (d) 

True

\noindent (e) 

True 

\noindent (f)

True 


\end{document}