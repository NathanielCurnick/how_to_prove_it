\documentclass[11pt]{article}


\title{\textbf{How to Prove It} \\ {\Large\itshape Daniel J. Velleman} \\ {\Large\itshape Introduction}}

\author{\textbf{Nathaniel Curnick} \\ \textit{Textbook Solutions}}

\date{}

%----------------------------------------------------------------------------------------

\begin{document}

\maketitle

\section*{Exercise 1}
\noindent (a) Factor $2^{15} - 1 = 32767$ into a product of two smaller positive integers

One possible solution is $4681 \times 7 = 32767$

\noindent (b) Find an integer $x$ such that $1 < x < 2^{32767} - 1$ and $2^{32767} - 1$ is divisible by x

Since we found that $32767$ was divisible by $4681$, we know that $2^{4681} - 1$ is a divisor of $2^{32767} - 1$

\section*{Exercise 2}

\noindent Make some conjectures about the values of n for which $3^n -1$ is prime

I made the following table 
\begin{center}
    \begin{tabular}{ c c c c }
     n & Prime? & $3^n - 1$ & Prime?\\ 
     1 & No & 2 & No \\  
     2 & Yes & 8 & No \\
     3 & Yes & 26 & No\\
     4 & No & 80 & No\\
     5 & Yes & 242 & No \\
     6 & No & 728 & No \\
     7 & Yes & 2186 & No \\
     8 & No & 6560 & No \\
     9 & No & 19682 & No \\
     10 & No & 59048 & No    
    \end{tabular}
    \end{center}

My conjecture is that no $3^n - 1$ is prime. This isn't a proof.

N.B. given how the question was asked I could have just made up any old (probably false) conjecture, but I tried to go with something sensible instead

\section*{Exercise 3}

\noindent The proof of Theorem 3 gives a method for finding a prime number different from any in a given list of prime numbers

\noindent (a) Use this method to find a prime number different from 2,3,5 and 7

Theorem 3 essentially tells us that we can find a new prime if we have a list of primes via the product of the primes $+1$. So we can do $2 \times 3 \times 5 \times 7 + 1= 211$

\noindent (b) Use this method to find a prime different from 2, 5, and 11

The same as the last one, we can do $2 \times 5 + 1 = 7$. The 11 is probably there to confuse you, although you could also get 3,7 and do $2 \times 3 \times 5 \times 7 \times 11 + 1 = 2311$

\section*{Exercise 4}

\noindent Find 5 consecutive integers that are not prime

Theorem 4 shows that $(n+1)! + 2$ will have n consecutive number that are not prime. Therefore, the numbers are 722, 723, 724, 725, 726.

\section*{Exercise 5}

\noindent Use the table in Figure 1.1 and the discussion on p. 5 to find two more perfect numbers

If $2^n -1$ is prime then $2^{n-1}(2^n-1)$ is perfect

From the table $2^5 -1$ and $2^7 -1$ are prime, so the perfect numbers are 496 and 8128.

\section*{Exercise 6}

\noindent The sequence 3, 5, 7 is a list of three prime numbers such that each pair of adjacent numbers in the list differ by two. Are there any more such ``triplet'' primes?

No, because after 10 of all three consecutive odd numbers, at least one will be divisible by 3 (not a prime)

\section*{Exercise 7}

\noindent A pair of distinct positive integers $(m, n)$ are called \textit{amicable} if the sum of all positive integers smaller than $n$ that divide $n$ is $m$ and the sum of all positive integers smaller than $m$ that divide $m$ is $n$. Show that $(220, 284)$ are amicable

The factors of 220 are 1, 2, 4, 5, 10, 11, 20, 22, 44, 55, 110 (that are smaller than 220), and their sum is 284.

The factors of 284 are 1, 2, 4, 71, 142 (that are smaller than 284) and they sum to 220.

Therefore, $(220, 284)$ are amicable.
\end{document}