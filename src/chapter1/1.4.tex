\documentclass[11pt]{article}
\usepackage{amssymb}
\usepackage{tipa}
\usepackage{amsmath}

\title{\textbf{How to Prove It} \\ {\Large\itshape Daniel J. Velleman} \\ {\Large\itshape Chapter 1: Operations on Sets}}

\author{\textbf{Nathaniel Curnick} \\ \textit{Textbook Solutions}}

\date{}

%----------------------------------------------------------------------------------------

\begin{document}

\maketitle

\section*{Exercise 1}

Let $A = \{1,3,12,35\}$, $B = \{3,7,12,20\}$, $C = \{x \text{\textbar} x \text{ is a prime number}\}$. List the elemts of the following sets. 
Are any of the sets below disjoint from the others? Are any of the sets below subsets of the others?

\noindent (a) $A \cap B$

$$A \cap B = \{3,12\}$$

Which is a subset of $A$, $B$, and $C$

\noindent (b) $(A \cup B) \setminus C$

$$A \cup B = \{1,3,7,12,20,35\}$$

$$(A \cup B) \setminus C = \{12, 20, 35\}$$

\noindent (c) $A \cup (B \setminus C)$

$$B \setminus C = \{\}$$

\section*{Exercise 2}

Let $A = \{\text{United States}, \text{Germany}, \text{China}, \text{Australia}\}$, $B = \{\text{Germany}, \text{France}, \text{India}, \text{Brazil}\}$ and $C = \{x \text{\textbar} x \text{ is a country in Europe}\}$. List the elements of the following sets. Are any of the sets below disjoint from any of the others? Are any of the sets below subsets of the others?

\noindent (a) $A \cup B$

$$A \cup B = \{\text{United States}, \text{Germany}, \text{China}, \text{Australia}, \text{France}, \text{India}, \text{Brazil}\}$$

\noindent (b) $(A \cap B) \setminus C$

$$A \cap B = \{\text{Germany}\}$$

$$(A \cap B) \setminus C = \{\}$$

\noindent (c) $(B \cap C) \setminus A$

$$B \cap C = \{\text{Germany}, \text{France}\}$$

$$(B \cap C) \setminus A = \{\text{France}\}$$

\section*{Exercise 3}

% TODO Find some sort of Venn Diagram maker

\section*{Exercise 4}

% TODO

\section*{Exercise 5}

Verify the identities in exercise 4 by writing out (using logical symbols) what it means for an object $x$ to be an element of each set and then using logical equivalences

\noindent (a) $A \setminus (A \cap B) = A \setminus B$

$$x \in (A \setminus (A \cap B))$$
$$x \in A \wedge (x \in \neg A \vee x \in \neg B)$$
$$(x \in A \wedge x \in \neg A) \vee (x \in A \wedge x \in \neg B)$$
$$x \in A \wedge x \in \neg B = A \setminus B$$

\noindent (b) $A \cup (B \cap C) = (A \cup B) \cap (A \cup C)$

$$x \in (A \cup (B \cap C))$$
$$(x \in A) \vee (x \in B \wedge x \in C)$$
$$(x \in A \vee x \in B) \wedge (x \in A \vee x \in C) = (A \cup B) \cap (A \cup C)$$

\section*{Exercise 6}

% TODO

\section*{Exercise 7}

Verify the identities in exercise 6 by writing out (using logical symbols) what it means for an object $x$ to be an element of each set and then using logical equivalences

\noindent (a) $(A \cup B) \setminus C = (A \setminus C) \cup (B \setminus C)$

$$x \in ((A \cup B) \setminus C)$$
$$x \in A \vee (x \in B \wedge x \in \neg C)$$
$$(x \in A \vee x \in B) \wedge (x \in A \vee x \in \neg C)$$
$$(x \in A \vee x \in B) \wedge \neg(x \in \neg A \wedge x \in C)$$
$$(A \cup B) \setminus (C \setminus A)$$



\end{document}