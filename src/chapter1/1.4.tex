\documentclass[11pt]{article}
\usepackage{amssymb}
\usepackage{tipa}
\usepackage{amsmath}

\title{\textbf{How to Prove It} \\ {\Large\itshape Daniel J. Velleman} \\ {\Large\itshape Chapter 1: Operations on Sets}}

\author{\textbf{Nathaniel Curnick} \\ \textit{Textbook Solutions}}

\date{}

%----------------------------------------------------------------------------------------

\begin{document}

\maketitle

\section*{Exercise 1}

Let $A = \{1,3,12,35\}$, $B = \{3,7,12,20\}$, $C = \{x \text{\textbar} x \text{ is a prime number}\}$. List the elemts of the following sets. 
Are any of the sets below disjoint from the others? Are any of the sets below subsets of the others?

\noindent (a) $A \cap B$

$$A \cap B = \{3,12\}$$

Which is a subset of $A$, $B$, and $C$

\noindent (b) $(A \cup B) \setminus C$

$$A \cup B = \{1,3,7,12,20,35\}$$

$$(A \cup B) \setminus C = \{12, 20, 35\}$$

\noindent (c) $A \cup (B \setminus C)$

$$B \setminus C = \{\}$$

\section*{Exercise 2}

Let $A = \{\text{United States}, \text{Germany}, \text{China}, \text{Australia}\}$, $B = \{\text{Germany}, \text{France}, \text{India}, \text{Brazil}\}$ and $C = \{x \text{\textbar} x \text{ is a country in Europe}\}$. List the elements of the following sets. Are any of the sets below disjoint from any of the others? Are any of the sets below subsets of the others?

\noindent (a) $A \cup B$

$$A \cup B = \{\text{United States}, \text{Germany}, \text{China}, \text{Australia}, \text{France}, \text{India}, \text{Brazil}\}$$

\noindent (b) $(A \cap B) \setminus C$

$$A \cap B = \{\text{Germany}\}$$

$$(A \cap B) \setminus C = \{\}$$

\noindent (c) $(B \cap C) \setminus A$

$$B \cap C = \{\text{Germany}, \text{France}\}$$

$$(B \cap C) \setminus A = \{\text{France}\}$$

\section*{Exercise 3}

% TODO Find some sort of Venn Diagram maker

\section*{Exercise 4}

% TODO

\section*{Exercise 5}

Verify the identities in exercise 4 by writing out (using logical symbols) what it means for an object $x$ to be an element of each set and then using logical equivalences

\noindent (a) $A \setminus (A \cap B) = A \setminus B$

$$x \in (A \setminus (A \cap B))$$
$$x \in A \wedge (x \in \neg A \vee x \in \neg B)$$
$$(x \in A \wedge x \in \neg A) \vee (x \in A \wedge x \in \neg B)$$
$$x \in A \wedge x \in \neg B = A \setminus B$$

\noindent (b) $A \cup (B \cap C) = (A \cup B) \cap (A \cup C)$

$$x \in (A \cup (B \cap C))$$
$$(x \in A) \vee (x \in B \wedge x \in C)$$
$$(x \in A \vee x \in B) \wedge (x \in A \vee x \in C) = (A \cup B) \cap (A \cup C)$$

\section*{Exercise 6}

% TODO

\section*{Exercise 7}

Verify the identities in exercise 6 by writing out (using logical symbols) what 
it means for an object $x$ to be an element of each set and then using logical 
equivalences

\noindent (a) $(A \cup B) \setminus C = (A \setminus C) \cup (B \setminus C)$

$$x \in ((A \cup B) \setminus C)$$
$$x \in A \vee (x \in B \wedge x \in \neg C)$$
$$(x \in A \vee x \in B) \wedge (x \in A \vee x \in \neg C)$$
$$(x \in A \vee x \in B) \wedge \neg(x \in \neg A \wedge x \in C)$$
$$(A \cup B) \setminus (C \setminus A)$$

\section*{Exercise 8}

Use any method you wish to verify the following identities:

\noindent (a) $(A \setminus B) \cap C = (A \cap C) \setminus B$

Take the LHS

$$(x \in A \wedge x \notin B) \wedge x \in C$$
$$(x \in A \wedge x \in C) \wedge x \notin B$$
$$(A \cap C) \setminus B$$

\noindent (b) $$(A \cap B) \setminus B = \emptyset$$

Take the LHS

$$x \in A \wedge x \in B \wedge x \notin B$$

This forms a contradiction - nothing can be in $B$ and not in $B$ at the same 
time, thus this is the empty set

\noindent (c) $$A \setminus (A \setminus B) = A \cap B$$

Take LHS

$$x \in A \wedge \neg (x \in A \wedge x \notin B)$$
$$x \in A \wedge (x \notin A \vee x \in B)$$
$$(x \in A \wedge x \notin A) \vee x \in A \wedge x \in B$$

Here the contradiction can be ignored

$$x \in A \wedge x \in B$$
$$A \cap B$$

\section*{Exercise 9}

% TODO: Proof of which are equal

For each of the following sets, write out (using logical symbols) what it means 
for an object $x$ to be an element of the set. Then determine which of these 
sets must be equal to each other by determining which statements are equivalent

\noindent (a) $(A \setminus B) \setminus C$

$$(x \in A \wedge x \notin B) \wedge n \notin C$$

\noindent (b) $A \setminus (B \setminus C)$

$$x \in A \wedge \neg (x \in B \wedge x \notin C)$$

\noindent (c) $(A \setminus B) \cup (A \cap C)$

$$(x \in A \wedge x \notin B) \vee (x \in A \wedge x \in C)$$

\noindent (d) $(A \setminus B) \cap (A \setminus C)$

$$(x \in A \wedge x \notin B) \wedge (x \in A \wedge x \notin C)$$

\noindent (e) $A \setminus (B \cup C)$

$$x \in A \wedge \neg (x \in B \vee x \in C)$$

\section*{Exercise 10}

It was shown in this section that for any sets $A$ and $B$, 
$(A \cup B) \ B \subseteq A$

\noindent (a) Give an example of two sets $A$ and $B$ for which 
$(A \cup B) \setminus B = A$.

$$A = \{1, 2\}, B = \{3, 4\}$$

(any set where $A$ and $B$ do not share any elements)

\noindent (b) Show that for all sets $A$ and $B$, 
$(A \cup B) \setminus B = A \setminus B$

Take LHS 

$$(x \in A \vee x \in B) \wedge x \notin B$$
$$x \in A \wedge x \notin B \vee x \in B \wedge x \notin B$$

Here, the contradiction can be ignored

$$x \in A \wedge x \notin B$$
$$A \setminus B$$

\section*{Question 11}

\noindent Support A and B are sets. Is it necessarily true that 
$(A \setminus B) \cup B = A$? If not, is one of these sets necessarily a subset 
of the other? Is $(A \setminus B) \cup B$ always equal to either $A \setminus B$
or $A \cup B$?

Take LHS

$$(x \in A \wedge x \notin B) \vee x \in B$$
$$x \in A \vee x \in B \wedge x \notin B \vee x \in B$$
$$x \in A \vee x \in B$$
$$A \cup B$$

\section*{Question 12}

% TODO

\section*{Question 13}

% TODO

\section*{Question 14}

% TODO

\section*{Question 15}

% TODO

\section*{Question 16}

% TODO

\section*{Question 17}

% TODO



\end{document}