\documentclass[11pt]{article}
\usepackage{amssymb}
\usepackage{tipa}
\usepackage{amsmath}

\newcommand{\then}{\rightarrow} 
\newcommand{\bicond}{\leftrightarrow}

\title{\textbf{How to Prove It} \\ {\Large\itshape Daniel J. Velleman} \\ {\Large\itshape Chapter 1: Operations on Sets}}

\author{\textbf{Nathaniel Curnick} \\ \textit{Textbook Solutions}}

\date{}

%----------------------------------------------------------------------------------------

\begin{document}

\maketitle

\section*{Exercise 1}

Analyze the logical forms of the following statements:

\noindent (a) If this gas either has an unpleasant smell or is not explosive, 
then it isn't hydrogen

$$(S \vee \neg E) \then \neg H$$

Where,

$S$ = Gas has an unpleasant smell,
$E$ = Gas is explosive,
$H$ = Gas is hydrogen.

\noindent (b) Having both a fever and a headache is a sufficient condition for 
George to go to the doctor

$$(F \wedge H) \then D$$

Where,

$F$ = George has a fever,
$H$ = George has a headache,
$D$ = George goes to the doctor.

\noindent (c) Both having a fever and having a headache are sufficient conditions
for George to go to the doctor.

$$(F \vee H) \then D$$

\noindent (d) If $x \neq 2$, then a necessary condition for $x$ to be prime is that
$x$ be odd.

$$(x \neq 2) \then (P(x) \then O(x))$$

Where,

$P(x)$ is true when $x$ is prime,
$O(x)$ is true when $x$ is odd.

\section*{Exercise 2}

Analyze the logical forms of the following statements:

\noindent (a) Mary will sell her house only if she can get a good price and find a
nice apartment.

$$H \then P \wedge A$$

Where,
$H$ = Mary will sell her house,
$P$ = Mary will get a good price,
$A$ = Mary will find a nice apartment.

\noindent (b) Having both a good credit history and an adequate down payment is a
necessary condition for getting a mortgage.

$$M \then (C \wedge D)$$

$C$ = Having good credit history,
$D$ = Having adequate down payment,
$M$ = Getting mortgage.

\noindent (c) John will drop out of school, unless someone stops him

If John is stopped, he will not drop out of school. If John is not stopped, then 
he will drop out of school

$$\neg S \bicond D$$

Where,
$S$ is stops John,
$D$ is John drops out of school.

\noindent (d) If x is divisible by either 4 or 6, then it isn't prime.

$$D(x, 4) \vee D(x, 6) \then \neg P(x)$$

\noindent Analyze the logical form of the following statement:


\noindent (a) If it is raining, then it is windy and the sun is not shining.
Now analyze the following statements. Also, for each statement determine
whether the statement is equivalent to either statement (a) or its converse.

$$R \then W \wedge \neg S$$

Where,
$R$ is it is raining,
$W$ is it is windy,
$S$ is sun is shining

\noindent (b) It is windy and not sunny only if it is raining.

$$W \wedge \neg S \then R$$

Converse of (a)

\noindent (c) Rain is a sufficient condition for wind with no sunshine.

$$R \then W \wedge \neg S$$

Equivalent to (a)

\noindent (d) Rain is a necessary condition for wind with no sunshine.

$$W \wedge \neg S \then R$$

Converse of (a)

\noindent (e) It's not raining, if either the sun is shining or it's not windy.

$$S \vee \neg W \then \neg R$$

Equivalent to (a)

\noindent (f) Wind is a necessary condition for it to be rainy, and so is a lack of
sunshine.

$$(R \then W) \wedge (R \then \neg S)$$ % TODOL: proof

Equivalent to (a)

\noindent (g) Either it is windy only if it is raining, or it is not sunny only if it is
raining.

$$(W \then R) \vee (\neg S \then R)$$ % TODO: proof

Converse of (a)

\section*{Exercise 4}

\noindent Use truth tables to determine whether or not the following arguments 
are valid 

\noindent (a) Either sales or expenses will go up. If sales go up, then the boss
will be happy. If expenses go up, then the boss will be unhappy. Therefore, 
sales and expenses will not both go up.

Let's make $S$ mean ``sales will go up'', $E$ mean ``expenses will go up'' and 
$B$ mean ``boss will be happy''

\[
  \begin{array}{ r l }
               & S \vee E \\
               & S \then B \\
               & E \then \neg B \\
    \cline{2-2}
    \therefore & \neg (S \wedge E)
  \end{array}
\]

Let's rewrite this with only logical connectives

\[
  \begin{array}{ r l }
               & S \vee E \\
               & \neg S \vee B \\
               & \neg E \vee \neg B \\
    \cline{2-2}
    \therefore & \neg (S \wedge E)
  \end{array}
\]

Let's also define
$U = S \vee E$, 
$V = \neg S \vee B$,
$W = \neg E \vee \neg B$,
$X = \neg (S \wedge E)$.

\begin{center}
\begin{tabular}{ c c c c c c c }
$S$ & $E$ & $B$ & $U$ & $V$ & $W$ & $X$\\ 
T & T & T & T & T & F & F\\  
T & T & F & T & F & T & F\\
T & F & T & T & T & T & T\\  
T & F & F & T & F & T & T\\
F & T & T & T & T & F & T\\  
F & T & F & T & T & T & T\\
F & F & T & F & T & T & T\\  
F & F & F & F & T & T & T\\
\end{tabular}
\end{center}

The only lines where the premises are all true are lines 3 and 6, and the 
conclusion is truth both times. Therefore, the argument is valid

\noindent (b) If the tax rate and the unemployment rate both go up, then there 
will be a recession. If the GDP goes up, then there will not be a recession. The 
GDP and taxes are both going up. Therefore, the unemployment rate is not going up 

Let's define $T$ to mean ``tax rate go up'', $U$ to mean ``unemployment go up'',
$R$ to mean ``will be a recession'' and $G$ to mean ``GDP will go up''.

\[
  \begin{array}{ r l }
               & (T \wedge U) \then R \\
               & G \then \neg R \\
               & G \wedge T\\
    \cline{2-2}
    \therefore & \neg U
  \end{array}
\]

Rewriting with logical connectives 

\[
  \begin{array}{ r l }
               & \neg (T \wedge U) \vee R \\
               & \neg G \vee \neg R \\
               & G \wedge T\\
    \cline{2-2}
    \therefore & \neg U
  \end{array}
\]

Let's also define
$A = \neg (T \wedge U) \vee R$,
$B = \neg G \vee \neg R$,
$C = G \wedge T$
$D = \neg U$.

\begin{center}
\begin{tabular}{ c c c c c c c c }
$T$ & $U$ & $R$ & $G$ & $A$ & $B$ & $C$ & $D$\\
T & T & T & T & T & F & T & F\\
T & T & T & F & T & T & F & F\\
T & T & F & T & F & T & T & F\\
T & T & F & F & F & T & F & F\\
T & F & T & T & T & F & T & T\\
T & F & T & F & T & T & F & T\\
T & F & F & T & T & T & T & T\\
T & F & F & F & T & T & F & T\\
F & T & T & T & T & F & F & F\\
F & T & T & F & T & T & F & F\\
F & T & F & T & T & T & F & F\\
F & T & F & F & T & T & F & F\\
F & F & T & T & T & F & F & T\\
F & F & T & F & T & T & F & T\\
F & F & F & T & T & T & F & T\\
F & F & F & F & T & T & F & T\\
\end{tabular}
\end{center}

The premises are all true on line 7 only, where the conclusion is true, so the 
argument is valid

\noindent (c) The warning light will come on if and only if the pressure is too
high and the relief valve is clogged. The relief valve is not clogged. 
Therefore, the warning light will come on if and only if the pressure is too
high

Let's define
$W$ is ``warning light will come on'',
$P$ is ``pressure too high'',
$R$ is ``relief valve clogged''.

\[
  \begin{array}{ r l }
               & W \bicond (P \wedge R)\\
               & \neg R \\
    \cline{2-2}
    \therefore & W \bicond P
  \end{array}
\]

Rewriting with logical connectives 

\[
  \begin{array}{ r l }
               & (\neg W \vee (P \wedge R)) \wedge (\neg (P \wedge R) \vee W)\\
               & \neg R \\
    \cline{2-2}
    \therefore & (\neg Q \vee P) \wedge (\neg P \vee W)
  \end{array}
\]

Let's also define
$A = (\neg W \vee (P \wedge R)) \wedge (\neg (P \wedge R) \vee W)$,
$B = \neg R$
$C = (\neg Q \vee P) \wedge (\neg P \vee W)$.

\begin{center}
\begin{tabular}{ c c c c c c }
$W$ & $P$ & $R$ & $A$ & $B$ & $C$\\ 
T & T & T & T & F & T\\  
T & T & F & F & T & T\\
T & F & T & F & F & F\\  
T & F & F & F & T & F\\
F & T & T & F & F & F\\  
F & T & F & T & T & F\\
F & F & T & T & F & T\\  
F & F & F & T & T & T\\
\end{tabular}
\end{center}

Lines 6 and 8 have all premises true, yet on line 6 the conclusion is false. 
Therefore, this is not a valid argument

\section*{Exercise 5}

\noindent (a) If Jones is convicted then he will go to prison. Jones will be 
convicted only if Smith testifies against him. Therefore, Jones won't go to 
prison unless Smith testifies against him.

Let's define 
$C$ as ``Jones is convicted'',
$P$ as ``Jones will go to prison'',
$S$ as ``Smith testifies''

The line ``Jones won't go to prison unless Smith testifies against him'' can be 
quite tricky due to the negative. But, this has the same meaning as 
``If Smith testifies then Jones goes to prison''. 

\[
  \begin{array}{ r l }
               & C \then P\\
               & P \then S \\
    \cline{2-2}
    \therefore & S \then P
  \end{array}
\]

Rewriting with logical connectives

\[
  \begin{array}{ r l }
               & \neg C \vee P\\
               & \neg P \vee S \\
    \cline{2-2}
    \therefore & \neg S \vee P
  \end{array}
\]

Let's also define 
$X = \neg C \vee P$,
$Y = \neg P \vee S$,
$Z = \neg S \vee P$

\begin{center}
\begin{tabular}{ c c c c c c }
$C$ & $P$ & $S$ & $X$ & $Y$ & $Z$\\ 
T & T & T & T & T & T\\  
T & T & F & T & F & T\\
T & F & T & F & T & F\\  
T & F & F & F & T & T\\
F & T & T & T & T & T\\  
F & T & F & T & F & T\\
F & F & T & T & T & F\\  
F & F & F & T & T & T\\
\end{tabular}
\end{center}

Lines 1, 5, 7 and 8 all have true premises, yet line 7 has a false conclusion, 
therefore this argument is not valid

\noindent (b) Either the Democrats or the Republicans will have a majority in
the Senate, but not both. Having a Democratic majority is a necessary condition
for the bill to pass. Therefore, if the Republicans have a majority in the 
Senate then the bill won't pass.

Let's define 
$D$ as ``Democratic majority'',
$R$ as ``Republican majority'',
$B$ as ``bill passes''

\[
  \begin{array}{ r l }
               & (D \vee R) \wedge \neg (D \wedge R)\\
               & B \then D \\
    \cline{2-2}
    \therefore & R \then \neg B
  \end{array}
\]

Let's rewrite with only logical connections

\[
  \begin{array}{ r l }
               & (D \vee R) \wedge \neg (D \wedge R)\\
               & \neg B \vee D \\
    \cline{2-2}
    \therefore & \neg R \vee \neg B
  \end{array}
\]

Let's also define 
$X = (D \vee R) \wedge \neg (D \wedge R)$,
$Y = \neg B \vee D$,
$Z = \neg R \vee \neg B$

\begin{center}
\begin{tabular}{ c c c c c c }
$D$ & $R$ & $B$ & $X$ & $Y$ & $Z$\\ 
T & T & T & F & T & F\\  
T & T & F & F & T & T\\
T & F & T & T & T & T\\  
T & F & F & T & T & T\\
F & T & T & T & F & F\\  
F & T & F & T & T & T\\
F & F & T & F & F & T\\  
F & F & F & F & T & T\\
\end{tabular}
\end{center}

Lines 3, 4 and 6 all have the premises true, and the conclusions are all true, 
therefore, this is a valid argument

\section*{Exercise 6}

\noindent (a) Show that $P \bicond Q$ is equivalent to 
$(P \wedge Q) \vee (\neg P \wedge \neg Q)$

Taking LHS

$$(P \then Q) \wedge (Q \then P)$$
$$(\neg P \vee Q) \wedge (\neg Q \vee P)$$
$$((\neg P \vee Q) \wedge \neg Q) \vee ((\neg P \vee Q) \wedge P)$$
$$(\neg P \wedge \neg Q) \vee (Q \wedge \neg Q) \vee (\neg P \wedge P) \vee (Q \wedge P)$$

Contradictions can be ignored here

$$(\neg P \wedge \neg Q) \vee (Q \wedge P)$$

\noindent (b) Show that $(P \then Q) \vee (P \then R)$ is equivalent to 
$P \then (Q \vee R)$

Taking LHS

$$\neg P \vee Q \vee \neg P \vee R$$
$$\neg P \vee (Q \vee R)$$
$$P \then (Q \vee R)$$

\section*{Exercise 7}

\noindent (a) Show that $(P \then R) \wedge (Q \then R)$ is equivalent to 
$(P \vee Q) \then R$

Taking LHS

$$(\neg P \vee R) \wedge (\neg Q \vee R)$$
$$((\neg P \vee R) \wedge \neg Q) \vee ((\neg P \vee R) \wedge R)$$
$$\neg P \wedge \neg Q \vee R \wedge \neg Q \vee \neg P \wedge R \vee R \wedge R$$

Use of absorbtion and idempotent laws 

$$\neg P \wedge \neg Q \vee R$$

Use of De Morgan law 

$$\neg (P \vee Q) \vee R$$
$$(P \vee Q) \then R$$

\noindent (b) Formulate and verify a similar equivalence involving 
$(P \then R) \vee (Q \then R)$

$$\neg P \vee R \vee \neg Q \vee R$$
$$\neg P \vee \neg Q \vee R$$
$$\neg (P \wedge Q) \vee R$$
$$(P \wedge Q) \then R$$

\section*{Exercise 8}

\noindent (a) Show that $(P \then Q) \wedge (Q \then R)$ is equivalent to 
$(P \then R) \wedge ((P \bicond Q) \vee (R \bicond Q))$

Use a truth table for this one 

$$(P \then Q) \wedge (Q \then R) \equiv (\neg P \vee Q) \wedge (\neg Q \vee R) = A$$

$$
    (P \then R) \wedge ((P \bicond Q) \vee (R \bicond Q)) 
    \equiv 
    \neg P \vee R \wedge (\neg P \vee Q \wedge \neg Q \vee P) \vee (\neg R \vee Q \wedge \neg Q \vee R)
    = B
$$

Use truth table for

\begin{center}
\begin{tabular}{ c c c c c }
    $P$ & $Q$ & $R$ & $A$ & $B$ \\ 
T & T & T & T & T\\  
T & T & F & F & F\\
T & F & T & F & F\\  
T & F & F & F & F\\
F & T & T & T & T\\  
F & T & F & F & F\\
F & F & T & T & T\\  
F & F & F & T & T\\
\end{tabular}
\end{center}

\noindent (b) Show that $(P \then Q) \vee (Q \then R)$ is a tautology

$$\neg P \vee Q \vee \neg Q \vee R$$

$$Q \vee \neg Q$$ 

is always true, therefore, tautology

\section*{Exercise 9}

\noindent Find a formula involving only the connectives $\neg$ and $\then$ that 
is equivalent to $P \wedge Q$

$$\neg (P \then \neg Q)$$
$$\neg (\neg P \vee \neg Q)$$
$$P \wedge Q$$

\section*{Exercise 10}

\noindent Find a formula involving only the connectives $\neg$ and $\then$ that 
is equivalent to $P \bicond Q$

$$(P \then Q) \wedge (\neg P \then \neg Q)$$

\section*{Exercise 11}

\noindent (a) Show that $(P \vee Q) \bicond Q$ is equivalent to $P \then Q$

Taking LHS

$$(P \vee Q) \then Q \wedge Q \then (P \vee Q)$$
$$\neg (P \vee Q) \vee Q \wedge (\neg Q \vee P \vee Q)$$

Ignore the tautology in this case

$$\neg (P \vee Q) \vee Q$$
$$(P \vee Q) \then Q$$

\noindent (b) Show that $(P \wedge Q) \bicond Q$ is equivalent to $Q \then P$

Taking LHS

$$(P \wedge Q) \then Q \wedge Q \then (P \wedge Q)$$
$$\neg (P \wedge Q) \vee Q \wedge \neg Q \vee (P \wedge Q)$$
$$\neg P \vee Q \vee \neg Q \wedge \neg Q \vee (P \wedge Q)$$

Can ignore tautology in this case

$$\neg Q \vee (P \wedge Q)$$
$$Q \then (P \wedge Q)$$

\section*{Exercise 12}

\noindent Which of the following are equivalent?

\noindent (a) $P \then (Q \then R)$

$$\neg P \vee \neg Q \vee R$$

\noindent (b) $Q \then (P \then R)$

$$\neg Q \vee \neg P \vee R$$

\noindent (c) $(P \then Q) \wedge (P \then R)$

$$\neg P \vee Q \wedge \neg P \vee R$$
$$\neg P \vee (Q \wedge R)$$

\noindent (d) $(P \wedge Q) \then R$

$$\neg (P \wedge Q) \vee R$$
$$\neg P \vee \neg Q \vee R$$

\noindent (e) $P \then (Q \wedge R)$

$$\neg P \vee (Q \wedge R)$$

Therefore a, b, d are equivalent and c, e are equivalent





\end{document}