\documentclass[11pt]{article}
\usepackage{amssymb}
\usepackage{tipa}
\usepackage{amsmath}

\title{\textbf{How to Prove It} \\ {\Large\itshape Daniel J. Velleman} \\ {\Large\itshape Chapter 1: Variables and Sets}}

\author{\textbf{Nathaniel Curnick} \\ \textit{Textbook Solutions}}

\date{}

%----------------------------------------------------------------------------------------

\begin{document}

\maketitle

\section*{Exercise 1}

Analyse the logical forms of the following statements:

\noindent (a) 3 is a common divisor of 6, 9, and 15. (Note: You did this in exercise 2 of Section 1.1, but you should be able to give a better answer now.)

$$D(6) \wedge D(9) \wedge D(15)$$

Where $D(x)$ means ``$x$ is divisible by 3''

\noindent (b) x is divisible by both 2 and 3 but not 4.

$$D(x, 2) \wedge D(x, 3) \wedge \neg D(x, 4)$$

Where $D(x,y)$ means $x$ is divisible by $y$

\noindent (c) x and y are natural numbers, and exactly one of them is prime.

$$N(x) \wedge N(y) \wedge ((P(x) \wedge \neg P(y)) \vee (P(y) \wedge \neg P(x)))$$

Where $N(x)$ means ``$x$ is a natural number'' and $P(x)$ means ``$x$ is prime''

\section*{Exercise 2}

Analyze the logical forms of the following statements:

\noindent  (a) x and y are men, and either x is taller than y or y is taller than x.

$$M(x) \wedge M(y) \wedge (T(x,y) \vee T(y,x))$$

Where $M(x)$ means ``$x$ is a man'' and $T(x,y)$ means ``$x$ is taller than $y$''

\noindent (b) Either x or y has brown eyes, and either x or y has red hair.

$$(B(x) \vee B(y)) \wedge (R(x) \vee R(y))$$

Where $B(x)$ means ``$x$ has brown eyes'' and $R(x)$ means ``$x$ has red hair''

\noindent (c) Either x or y has both brown eyes and red hair.

$$(B(x) \wedge R(x)) \vee (B(y) \wedge R(y))$$

Symbols mean the same as (b)

\section*{Exercise 3}

Write definitions using elementhood tests for the following sets:
    
\noindent (a) \{Mercury, Venus, Earth, Mars, Jupiter, Saturn, Uranus, Neptune, Pluto\}.

\{x \textpipe x is a planet\}

\noindent (b) \{Brown, Columbia, Cornell, Dartmouth, Harvard, Princeton, University of Pennsylvania, Yale\}.

\{x \textpipe x is an Ivy League University\}

\noindent (c) \{Alabama, Alaska, Arizona, . . . , Wisconsin, Wyoming\}.

\{x \textpipe x is a State of the United States\}

\noindent (d) \{Alberta, British Columbia, Manitoba, New Brunswick, Newfoundland and Labrador, Northwest Territories, Nova Scotia, Nunavut, Ontario, Prince Edward Island, Quebec, Saskatchewan, Yukon\}.

\{x \textpipe x is a State of Canada\}

\section*{Exercise 4}

Write definitions using elementhood tests for the following sets:

\noindent (a) \{1, 4, 9, 16, 25, 36, 49, . . .\}.

$$\{x^2 \text{\textbar} x \in \mathbb{Z}^+ \}$$

\noindent (b) \{1, 2, 4, 8, 16, 32, 64, . . .\}.

$$\{2^x \text{\textbar} x \in \mathbb{N} \}$$

\noindent (c) \{10, 11, 12, 13, 14, 15, 16, 17, 18, 19\}

$$\{x \in \mathbb{N} \text{\textbar} 9 < x < 20 \}$$

\section*{Exercise 5}

Simplify the following statements. Which variables are free and which are bound? If the statement has no free variables, say whether it is true or false.

\noindent (a) $-3 \in \{x \in \mathbb{R} \text{\textbar} 13 - 2x > 1\}$

$$(-3 \in \mathbb{R}) \wedge (13 - 2(-3) > 1)$$

No free variables
$x$ is a bound variable
Statement is true, since $13 - 2x$ is 19 when $x=-3$, which is larger than 1

\noindent (b) $4 \in \{x \in \mathbb{R}^- \text{\textbar} 12 - 2x > 1 \}$

$$(4 \in \mathbb{R}) \wedge (4 < 0) \wedge (13 - 2(4) > 1)$$

No free variables
$x$ is a bound variable
This statement is false since $4 \notin \mathbb{R}^-$

\noindent (c) $5 \notin \{x \in \mathbb{R} \text{\textbar} 13 - 2x > c\}$

$$(5 \notin \mathbb{R}) \wedge (13 - 2(5) > c)$$

$c$ is a free variable, $x$ is a bound variable. We can not necessarily say it is true or false.

\section*{Exercise 6}

Simplify the following statements. Which variables are free and which are bound? If the statement has no free variables, say whether it is true or false.

\noindent (a) $w \in \{x \in \mathbb{R} \text{\textbar} 13-2x > c\}$

$$(w \in \mathbb{R}) \wedge (13-2(w) > c)$$

$w$ and $c$ are free variables, $x$ is a bound variable. We can not say if it is true or false

\noindent (b) $4 \in \{x \in \mathbb{R} \text{\textbar} 13 - 2x \in \{y \text{\textbar} y \text{ is a prime number}\}\}$
(It might make this statement easier to read if we let $P = \{y \text{\textbar} y \text{ is a prime number}\}$;
using this notation, we could rewrite the statement as 
$4 \in \{x \in \mathbb{R} \text{\textbar} 13 - 2x \in P\}$.)

$$(4 \in \mathbb{R}) \wedge (13 - 2(4) \in P)$$

$x$ and $y$ are bound variables, this statement is true since $13 - 2(4) = 5$, which is prime

\noindent (c) $4 \in \{x \in \{y \text{\textbar} y \text{ is a prime number}\} \text{\textbar} 13 - 2x > 1\}$
(Using the same notation as in part (b), we could write this as $4 \in \{x \in P \text{\textbar} 13 - 2x > 1\}$.)

$$(4 \in P) \wedge (13 - 2(4) > 1)$$

No free variables, $x$ and $y$ are bound. False since 4 is not prime.

\section*{Exercise 7}

List the elements of the following sets

\noindent (a) $\{x \in \mathbb{R} \text{\textbar} 2x^2 + x - 1 = 0 \}$

$$2x^2 + x - 1 = 0 $$
$$(2x -1)(x+1)=0$$

Therefore, the set is $\{-1, 0.5\}$

\noindent (b) $\{x \in \mathbb{R}^+ \text{\textbar} 2x^2 + x - 1 = 0 \}$

Using (a) the set is $\{0.5\}$

\noindent (c) $\{x \in \mathbb{Z} \text{\textbar} 2x^2 + x - 1 = 0 \}$

Using (a) the set is $\{-1\}$

\noindent (d) $\{x \in \mathbb{N} \text{\textbar} 2x^2 + x - 1 = 0 \}$

Using (a) the set is $\{\}$ (empty set)

\section*{Exercise 8}

What are the truth sets of the following statements? List a few elements of the truth set if you can

\noindent (a) Elizabeth Taylor once married to $x$

$$\{\text{Conrad Hilton Jr.}, \text{Michael Wilding}, \text{Mike Todd}, \text{Eddie Fisher}, \dots \}$$

\noindent (b) $x$ is a logical connective studied in section 1.1

$$\{\neg, \vee, \wedge \}$$

\noindent (c) $x$ is the author of this book

$$\{\text{Daniel J. Velleman}\}$$

\section*{Exercise 9}

What are the truth sets of the following statements? List a few elements of the truth set if you can

\noindent (a) $x$ is a real number and $x^2 -4x + 3 = 0$

$$x^2 -4x + 3 = 0$$
$$(x-1),(x-3) = 0$$

Set is $\{1,3\}$

\noindent (b) $x$ is a real number and $x^2 -2x + 3 = 0$

$$x^2 -2x + 3 = 0$$

Has no real roots, therefore the set is $\{\}$

\noindent (c) $x$ is a real number and $5 \in \{y \in \mathbb{R} \text{\textbar} x^2 + y^2 < 50 \}$

$$\{x \in \mathbb{R} \text{\textbar} x^2 < 25\}$$

\end{document}