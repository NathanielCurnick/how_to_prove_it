\documentclass[11pt]{article}
\usepackage{amssymb}

\title{\textbf{How to Prove It} \\ {\Large\itshape Daniel J. Velleman} \\ {\Large\itshape Chapter 1: Truth Tables}}

\author{\textbf{Nathaniel Curnick} \\ \textit{Textbook Solutions}}

\date{}

%----------------------------------------------------------------------------------------

\begin{document}

\maketitle

\section*{Exercise 1}

Make truth tables for the following formulas:

\noindent (a) $ \neg P \vee Q $

\begin{center}
\begin{tabular}{ c c c c }
 $P$ & $\neg P$ & $Q$ & $ \neg P \vee Q $ \\ 
T & F & T & T\\  
T & F & F & F\\
F & T & T & T\\  
F & T & F & T
\end{tabular}
\end{center}

\noindent (b) $ (S \vee G) \wedge (\neg S \vee \neg G) $

First rewrite as 

\begin{center}
\begin{tabular}{ c c c  }
 $S$ & $G$ & $ (S \vee G) \wedge \neg (S \vee G) $\\ 
T & T & F\\  
T & F & T\\
F & T & T\\  
F & F & F
\end{tabular}
\end{center}

\section*{Exercise 2}
Make truth tables for the following formulas:

\noindent (a) $ \neg [P \wedge (Q \vee \neg P)] $

\begin{center}
\begin{tabular}{ c c c  }
 $P$ & $Q$ & $ \neg [P \wedge (Q \vee \neg P)] $\\ 
T & T & F\\  
T & F & T\\
F & T & T\\  
F & F & T
\end{tabular}
\end{center}

\noindent (b) $ (P \vee Q) \wedge (\neg P \vee R) $

\begin{center}
\begin{tabular}{ c c c c c c }
 $P$ & $Q$ & $R$ & $(P \vee Q)$ & $(\neg P \vee R)$ & $ (P \vee Q) \wedge (\neg P \vee R) $\\ 
T & T & T & T & T & T\\  
T & T & F & T & F & F\\
T & F & T & T & T & T\\  
T & F & F & T & F & F\\
F & T & T & T & T & T\\  
F & T & F & T & T & T\\
F & F & T & F & T & F\\  
F & F & F & F & T & F
\end{tabular}
\end{center}

\section*{Exercise 3}

In this exercise we will use the symbol $+$ to mean \textit{exclusive or}. In other words $P + Q$ means ``P or Q, but not both''.

\noindent (a) Make a truth table for $P + Q$

\begin{center}
\begin{tabular}{ c c c  }
 $P$ & $Q$ & $P + Q$\\ 
T & T & F\\  
T & F & T\\
F & T & T\\  
F & F & F
\end{tabular}
\end{center}

\noindent (b) Find a formula using only the connectives $\wedge$, $\vee$ and $\neg$ that is equivalent to $P + Q$. Justify your answer with a truth table.

$$(P \vee Q) \wedge \neg (P \wedge Q)$$

\begin{center}
\begin{tabular}{ c c c c c }
 $P$ & $Q$ & $P \vee Q$ & $ \neg (P \wedge Q) $ & $(P \vee Q) \wedge \neg (P \wedge Q)$\\ 
T & T & T & F & F\\  
T & F & T & T & T\\
F & T & T & T & T\\  
F & F & F & T & F
\end{tabular}
\end{center}

\section*{Exercise 4}

Find a formula using only the connectives $\wedge$ and $\neg$ that is equivalent to $P \vee Q$. Justify your answer with a truth table

$$\neg (\neg P \wedge \neg Q)$$

\begin{center}
\begin{tabular}{ c c c  }
 $P$ & $Q$ & $\neg (\neg P \wedge \neg Q)$\\ 
T & T & T\\  
T & F & T\\
F & T & T\\  
F & F & F
\end{tabular}
\end{center}

\section*{Exercise 5}

Some mathematicians use the symbol $\downarrow$ to mean \textit{nor}. In other words $P \downarrow Q$ means ``neither P nor Q''.

\noindent (a) Make a truth table for $P \downarrow Q$

\begin{center}
\begin{tabular}{ c c c }
 $P$ & $Q$ & $P \downarrow Q$\\ 
T & T & F\\  
T & F & F\\
F & T & F\\  
F & F & T
\end{tabular}
\end{center}

\noindent (b) Find a formula using only the connectives $\wedge$, $\vee$ and $\neg$ that is equivalent to $P \downarrow Q$

$$\neg P \wedge \neg Q$$

\begin{center}
\begin{tabular}{ c c c }
 $P$ & $Q$ & $\neg P \wedge \neg Q$\\ 
T & T & F\\  
T & F & F\\
F & T & F\\  
F & F & T
\end{tabular}
\end{center}

\noindent (c) Final formulas using only the connective $\downarrow$ that are equivalent to $\neg P$, $P \vee Q$ and $P \wedge Q$

$$\neg P = P \downarrow P$$

\begin{center}
\begin{tabular}{c c}
$P$ & $P \downarrow P$\\
T & F\\
F & T
\end{tabular}
\end{center}

$$P \vee Q = (P \downarrow Q) \downarrow (P \downarrow Q)$$

\begin{center}
\begin{tabular}{ c c c }
 $P$ & $Q$ & $(P \downarrow Q) \downarrow (P \downarrow Q)$\\ 
T & T & T\\  
T & F & T\\
F & T & T\\  
F & F & F
\end{tabular}
\end{center}

$$P \wedge Q = (P \downarrow P) \downarrow (Q \downarrow Q)$$

\begin{center}
\begin{tabular}{ c c c }
 $P$ & $Q$ & $(P \downarrow P) \downarrow (Q \downarrow Q)$\\ 
T & T & T\\  
T & F & F\\
F & T & F\\  
F & F & F
\end{tabular}
\end{center}

\section*{Exercise 6}

Some mathematicians write $P | Q$ to mean ``P and Q are not both true''.
(This connective is called \textit{nand} and is used in the study of circuits in computer science)

\noindent (a) Make a truth table for $P | Q$

\begin{center}
\begin{tabular}{ c c c }
 $P$ & $Q$ & $P | Q$\\ 
T & T & F\\  
T & F & T\\
F & T & T\\  
F & F & T
\end{tabular}
\end{center}

\noindent (b) Find a formula using only the connectives $\wedge$, $\vee$ and $\neg$ that is equivalent to $P | Q$

$$\neg (P \wedge Q)$$

\noindent (c) Final formulas using only the connective $|$ that are equivalent to $\neg P$, $P \vee Q$ and $P \wedge Q$

$$\neg P = P|P$$

$$P \vee Q = (P | P) | (Q | Q)$$

$$P \wedge Q = (P | Q) | (P | Q)$$

Notice how $P \vee Q$ and $P \wedge Q$ have a kind of symmetry when written out with only $\downarrow$ or $|$

\section*{Exercise 7}

TODO

\section*{Exercise 8}

Use truth tables to determine which of the following formulas are equivalent to each other

\noindent (a) $(P \wedge Q) \vee (\neg P \wedge \neg Q)$

\begin{center}
\begin{tabular}{ c c c }
 $P$ & $Q$ & $(P \wedge Q) \vee (\neg P \wedge \neg Q)$\\ 
T & T & T\\  
T & F & F\\
F & T & F\\  
F & F & T
\end{tabular}
\end{center}

\noindent (b) $\neg P \vee Q$

\begin{center}
\begin{tabular}{ c c c }
 $P$ & $Q$ & $\neg P \vee Q$\\ 
T & T & T\\  
T & F & F\\
F & T & T\\  
F & F & T
\end{tabular}
\end{center}

\noindent (c) $(P \vee \neg Q) \wedge (Q \vee \neg P)$

\begin{center}
\begin{tabular}{ c c c }
 $P$ & $Q$ & $(P \vee \neg Q) \wedge (Q \vee \neg P)$\\ 
T & T & T\\  
T & F & F\\
F & T & F\\  
F & F & T
\end{tabular}
\end{center}

\noindent (d) $\neg (P \vee Q)$

\begin{center}
\begin{tabular}{ c c c }
 $P$ & $Q$ & $(P \vee \neg Q) \wedge (Q \vee \neg P)$\\ 
T & T & F\\  
T & F & F\\
F & T & F\\  
F & F & T
\end{tabular}
\end{center}

\noindent (e) $(Q \wedge P) \vee \neg P$

\begin{center}
\begin{tabular}{ c c c }
 $P$ & $Q$ & $(Q \wedge P) \vee \neg P$\\ 
T & T & T\\  
T & F & F\\
F & T & T\\  
F & F & T
\end{tabular}
\end{center}

(b) and (e) are the same

(a) and (c) are the same

\section*{Exercise 9}

Use truth tables to determine which of these statements are tautologies, which are contradictions and which are neither

\noindent (a) $(P \vee Q) \wedge (\neg P \vee \neg Q)$

\begin{center}
\begin{tabular}{ c c c }
 $P$ & $Q$ & $(P \vee Q) \wedge (\neg P \vee \neg Q)$\\ 
T & T & F\\  
T & F & T\\
F & T & T\\  
F & F & F
\end{tabular}
\end{center}

Neither a tautology or contradiction

\noindent (b) $(P \vee Q) \wedge (\neg P \wedge \neg Q)$\\ 

\begin{center}
\begin{tabular}{ c c c }
 $P$ & $Q$ & $(P \vee Q) \wedge (\neg P \wedge \neg Q)$\\ 
T & T & F\\  
T & F & F\\
F & T & F\\  
F & F & F
\end{tabular}
\end{center}

This is a contradiction

\noindent (c) $(P \vee Q) \vee (\neg P \vee \neg Q)$

\begin{center}
\begin{tabular}{ c c c }
 $P$ & $Q$ & $(P \vee Q) \vee (\neg P \vee \neg Q)$\\ 
T & T & T\\  
T & F & T\\
F & T & T\\  
F & F & T
\end{tabular}
\end{center}

This is a tautology

\noindent (d) $[P \wedge (Q \vee \neg R)] \vee (\neg P \vee R)$

\begin{center}
\begin{tabular}{ c c c c }
 $P$ & $Q$ & $R$ & $[P \wedge (Q \vee \neg R)] \vee (\neg P \vee R)$ \\ 
T & T & T & T\\  
T & T & F & T\\
T & F & T & T\\  
T & F & F & T\\
F & T & T & T\\  
F & T & F & T\\
F & F & T & T\\  
F & F & F & T\\
\end{tabular}
\end{center}

This is a tautology

\section*{Exercise 10}

Use truth tables to check these laws

\noindent (a) The second De Morgan's law. (The first was checked in the text)

Reminder that the second De Morgan's law is 

$$\neg (P \vee Q) = \neg P \wedge \neg Q$$

\begin{center}
\begin{tabular}{ c c c c }
 $P$ & $Q$ & $\neg (P \vee Q)$ & $\neg P \wedge \neg Q$\\ 
T & T & F & F\\  
T & F & F & F\\
F & T & F & F\\  
F & F & T & T
\end{tabular}
\end{center}

\noindent (b) The distributive laws

Reminder that the distributive laws are 

$$P \wedge (Q \wedge R) = (P \wedge Q) \wedge R$$

$$P \vee (Q \vee R) = (P \vee Q) \vee R$$

\begin{center}
\begin{tabular}{ c c c c c c c }
 $P$ & $Q$ & $R$ & $Q \wedge R$ & $P \wedge (Q \wedge R)$ & $(P \wedge Q)$ & $(P \wedge Q) \wedge R$\\ 
T & T & T & T & T & T & T\\  
T & T & F & F & F & T & F\\
T & F & T & F & F & F & F\\  
T & F & F & F & F & F & F\\
F & T & T & T & F & F & F\\  
F & T & F & F & F & F & F\\
F & F & T & F & F & F & F\\  
F & F & F & F & F & F & F
\end{tabular}
\end{center}

\begin{center}
\begin{tabular}{ c c c c c c c }
 $P$ & $Q$ & $R$ & $Q \vee R$ & $P \vee (Q \vee R)$ & $(P \vee Q)$ & $(P \vee Q) \vee R$\\ 
T & T & T & T & T & T & T\\  
T & T & F & T & T & T & T\\
T & F & T & T & T & T & T\\  
T & F & F & F & T & T & T\\
F & T & T & T & T & T & T\\  
F & T & F & T & T & T & T\\
F & F & T & T & T & F & T\\  
F & F & F & F & F & F & F
\end{tabular}
\end{center}

\section*{Exercise 11}

Use the laws stated in the text to find simpler formulas equivalent to these formulas.

\noindent (a) $\neg (\neg P \wedge \neg Q) $

Use of De Morgan's law

$$\neg (\neg P \wedge \neg Q) = \neg \neg P \vee \neg \neg Q$$

Use of double negation law

$$\neg \neg P \vee \neg \neg Q = P \vee Q$$

\noindent (b) $(P \wedge Q) \vee (P \wedge \neg Q)$

Use of distributive law

$$((P \wedge Q) \vee P) \wedge ((P \wedge Q) \vee \neg Q)$$

Use of associative law

$$(P \vee (P \wedge Q)) \wedge (\neg Q \vee (P \wedge Q))$$

Use of absorbtion law and distributive law

$$P \wedge ((\neg Q \vee P) \wedge (\neg Q \vee Q))$$
$$P \wedge (\neg Q \vee P)$$

By absorbtion law

$$P$$ 

\noindent (c) $\neg (P \wedge \neg Q) \vee (\neg P \wedge Q)$

Use of De Morgan's law

$$(\neg P \vee Q) \vee (\neg P \wedge Q)$$

Use of distributive law

$$(\neg P \vee Q \vee \neg P) \wedge (\neg P \vee Q \vee Q)$$

Use of idempotent law

$$(\neg P \vee Q) \wedge (\neg P \vee Q)$$

$$\neg P \vee Q$$

\section*{Exercise 12}

Use the laws stated in the text to find simpler formulas equivalent to these formulas.

\noindent (a) $\neg (\neg P \vee Q) \vee (P \wedge \neg R)$

Use of De Morgan's law

$$(\neg \neg P \wedge \neg Q) \vee (P \wedge \neg R)$$

Use of double negation law

$$(P \wedge \neg Q) \vee (P \wedge \neg R)$$

Use of distributive law

$$P \wedge (\neg Q \vee \neg R)$$

\noindent (b) $\neg (\neg P \wedge Q) \vee (P \wedge \neg R)$

Use of De Morgan's law

$$(P \vee \neg Q) \vee (P \wedge \neg R)$$

Use of distributive law

$$(P \vee \neg Q \vee P) \wedge (P \vee \neg Q \vee \neg R)$$

Use of idempotent law

$$(P \vee \neg Q) \wedge (P \vee \neg Q \vee \neg R)$$

Use of absorbtion law

$$(P \vee \neg Q)$$

\noindent (c) $(P \wedge R) \vee (\neg R \wedge (P \vee Q))$

Use of distributive law

$$(P \wedge R) \vee (\neg R \wedge P) \vee (\neg R \wedge Q)$$

Use of distributive law again

$$(P \wedge (R \vee \neg R)) \vee (\neg R \wedge Q)$$

$$P \vee (\neg R \wedge Q)$$

\section*{Exercise 13}

Use the first De Morgan's law and the double negation law to derive the second De Morgan's law

Reminder that the first De Morgan's law is

$$\neg (P \wedge Q) = \neg P \vee \neg Q$$

and the double negation law is 

$$\neg \neg P = P$$

We want to prove that 

$$\neg (P \vee Q) = \neg P \wedge \neg Q$$

Beginning with the LHS and applying the double negation in reverse 

$$\neg (P \vee Q) = \neg (\neg \neg P \vee \neg \neg Q)$$

Use of De Morgan's first law

$$\neg(\neg(\neg P \wedge \neg Q))$$

$$\neg P \wedge \neg Q$$

\section*{Exercise 13}

% TODO

\section*{Exercise 14}

% TODO

\section*{Exercise 15}

% TODO

\section*{Exercise 16}

% TODO

\section*{Exercise 17}

% TODO

\section*{Exercise 18}

% TODO

\end{document}