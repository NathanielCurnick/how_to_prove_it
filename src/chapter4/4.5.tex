\documentclass[11pt]{article}
\usepackage{amssymb}
\usepackage{tipa}
\usepackage{amsmath}
\usepackage[scr]{rsfso}
\usepackage{graphicx}
\usepackage{float}

\newcommand{\then}{\rightarrow} 
\newcommand{\bicond}{\leftrightarrow}
\newcommand{\powerset}[1]{\mathscr{P}(#1)}
\newcommand{\family}[1]{\mathcal{#1}}
\newcommand{\dprime}{{\prime \prime}}

\title{\textbf{How to Prove It} \\ {\Large\itshape Daniel J. Velleman} \\ {\Large\itshape Chapter 4.5: Equivalence Relations}}

\author{\textbf{Nathaniel Curnick} \\ \textit{Textbook Solutions}}

\date{}

%----------------------------------------------------------------------------------------

\begin{document}

\maketitle

\section*{Exercise 1}

Find all partitions of the set $A = \{1,2,3\}$

$$\{\{1\}, \{2\}, \{3\}\}$$

$$\{\{1,2\}, \{3\}\}$$

$$\{\{1\}, \{2,3\}\}$$

$$\{\{1,3\}, \{2\}\}$$

$$\{\{1,2,3\}\}$$

\section*{Exercise 2}

Find all equivalence relations on the set $A = \{1,2,3\}$

They correspond to the partitions from Exercise 1

$$\{(1,1),(2,2),(3,3),(2,1),(2,2),(2,3),(3,1),(3,2)\}$$

$$\{(1,1),(1,2),(2,1),(2,2),(3,3)\}$$

$$\{(1,1),(1,3),(3,1),(3,3),(2,2)\}$$

$$\{(2,2),(2,3),(3,2),(3,3),(1,1)\}$$

$$\{(1,1),(2,2),(3,3)\}$$

\section*{Exercise 3}

Let $W =$ the set of all words in the English language. Which of the following 
relations on $W$ are equivalence relations? For those that are equivalence
relations, which are the equivalence classes?

\noindent (a) $R = \{(x,y) \in W \times W | \text{ the words } x \text{ and } y \text{ start with the same letter}\}$

Equivalence relation. There are 26 classes - one for each letter (this is basically the dictionary)

\noindent (b) $S = \{(x,y) \in W \times W | \text{the words} x \text{ and } y \text{ have at least one letter in commond}\}$

Not an equivalence relation as it is not transitive e.g. 
bar $\rightarrow$ car, car $\rightarrow$ cup, but NOT bar $\rightarrow$ cup.

\noindent (c) $T = \{(x,y) \in W \times W | \text{the words } x \text{ and } y \text{ have the same number of letters} \}$

Equivalence relation. Equivalence class is the number of letters in the word from 
1 to the largest word in English 

\section*{Exercise 4}

Which of the following relations on $\mathbb{R}$ are equivalence relations?
For those that are equivalence relations, what are the equivalence classes?

\noindent (a) $R = \{(x,y) \in \mathbb{R} \times \mathbb{R} | x - y \in \mathbb{N}\}$

Not an equivalence relation as it is not symmetric

\noindent (b) $S = \{(x,y) \in \mathbb{R} \times \mathbb{R} | x-y \in \mathbb{Q}\}$

Equivalence relation. For any $x \in \mathbb{R}$, $[x]_S = \{x + q | q \in \mathbb{Q} \}$

\noindent (c) $T = \{(x,y) \in \mathbb{R} \times \mathbb{R} | \exists n \in \mathbb{Z} (y = x \cdot 10^n)\}$

Equivalence relation. For any $x \in \mathbb{R}$, $[x]_T$ is the set of numbers 
which have the decimal point in a different place e.g.

$$[\pi]_T = \{3.141..., 31.41..., 314.1..., 0.3141...,\}$$

\section*{Exercise 5}

Let $L$ be the set of all nonvertical lines in the plane. Which of the following
relations on $L$ are equivalence relations? For those that are equivalence relations,
which are the equivalence classes?

\noindent (a) $R = \{(k,l) \in L \times L | \text{the lines } k \text{ and } l \text{ have the same slope}\}$

It is an equivalence relation. Equivalence classes are each unique gradient.

\noindent (b) $S = \{(k,l) \in L \times L | \text{the lines } k \text{ and } l \text{ are perpendicular}\}$

No an equivalence relation since it is not transitive

\noindent (c) $T = \{(k,l) \in L \times L | k \cap x = l \cap x \text{ and } k \cap y = l \cap y\}$,
where $x$ and $y$ are the x-axis and y-axis 

Equivalence relation.. The set of all lines (that are not horizontal) and pass 
through the origin is one class and each other line has its own class.

\section*{Exercise 6}

In the discussion of the same-birthday equivalence relation $B$ following 
Definition 4.5.3, we claimed that $P/B = \{P_d | d \in D\}$. 
Give a careful prood of this claim. You will find when you work out the 
proof there is an assumption you must make about people's birthdays 
(a very reasonable assumption) to make the proof work. What is this 
assumption?

Suppose $X \in P/B$. Then we can choose some $x \in P$ such that $X = [x]$.
Let $d$ be $x$'s birthday. Then for every $y \in P$, $y \in [x]$ iff $yBx$
iff $y \in P_D$, and therefore $X = [x] = P_D$. Since $X$ was arbitrary, this 
shows that $P/B \subseteq \{P_d | d \in D\}$

Now suppose $d \in D$. At this point we make the assumption that some person $x$ 
was born on day $d$. As before, it follows that $P_d = [x] \in P/B$. Since $d$
was arbitrary, we conclude that $\{P_d | d \in D\} \subseteq P/B$

\section*{Exercise 7}

Let $T$ be the set of all triangles, and let 
$S = \{(s,t) \in T \times T | \text{the triangles } s \text{ and } t \text{ are similar}\}$.
Verify that $S$ is an equivalence relation.

Reflexive since a triangle has the same angles as itself. Symmetric since if 
triangle $p$ has the same angles as triangle $q$, then $q$ has the same angles 
as $p$. Transitive since if $p$ has the same angles as $q$ and $q$ has the 
same angles as $r$ then $p$ has the same angles as $r$.

\section*{Exercise 8}

Complete the proof og Lemma 4.5.7

To show $R$ is symmetric, suppose $x \in A$. Since $\bigcup \family{F} = A$,
then $x \in \bigcup \family{F}$. Suppose $y \in A$ then $y \in \bigcup \family{F}$
for the same reason. Then we can choose some $X \in \family{F}$ such that 
$x \in X$ and $y \in X$. Thus, $(x,y) \in X \times X$ and $(y,x) \in X \times X$,
so $(x,y) \in \bigcup_{X \in \family{F}} (X \times X)$ and $(y,x) \in \bigcup_{X \in \family{F}} (X \times X)$
which is $R$.

To show $R$ is transitive suppose $x,y,z \in A$. As above, we now know that 
$(x,y), (y,z), (x,z) \in X \times X$ so 
$(x,y), (y,z), (x,z) \in \bigcup_{X \in \family{F}} (X \times X)$ which is $R$.

\section{Exercise 9}

Suppose $R$ and $S$ are equivalence relations on $A$ and $A/R = A/S$. Prove that 
$R=S$.

First, we need to prove that $\forall x \in A ([x]_R = [x]_S)$. Let $x \in A$ be
arbitrary. Then $[x]_R \in A/R = A/S$, so there is some $y \in A$ such that 
$[x]_R = [y]_S$. By part 1 of lemma 4.5.5, $x \in [x]_R$, so $x \in [y]_S$
and therefore by part 2 of lemma 4.5.5 $[x]_S = [y]_s = [x]_R$. Since $x$ was 
arbitrary, this proves that $\forall x \in A ([x]_R = [x]_S)$. To prove that 
$R=S$ let $(x,y) \in A \times A$. Then, $(x,y) \in R$ iff $x \in [y]_R$ iff 
$x \in [y]_S$ iff $(x,y) \in S$.

\section*{Exercise 10}

Suppose $R$ is an equivalence relation on A. Let $\family{F} = A/R$, and let 
$S$ be the equivalence relation determined by $\family{F}$. In other words,
$S = \bigcup_{X \in \family{F}} (X \times X)$. Prove that $S=R$.

Since $S$ is an equivalence relation determined by $\family{F}$, the proof 
of theorem 4.5.6 shows that $A/S = \family{F} = A/R$ and then Exercise 9 above 
starts from there.

\section*{Exercise 11}

Let $\equiv_m$ be the ``congruene modulo $m$'' relation defined in the text for 
a postivie integer $m$.

\noindent (a) Complete the proof of Theorem 4.5.10 by showing that $\equiv_m$ is 
reflexive and symmetric.

Let $x \in \mathbb{Z}$. Since $x-x = 0 \cdot m$, $x \equiv_m x$ so it is 
reflexive. Now suppose $x \equiv_m y$. Then $m | (x-y)$ so we choose some 
integer $k$ such that $x-y = km$. Therefore $y-x = (-k)m$, so $m|(y-x)$ and 
$y = mx$. This shows it is symmetric.

\noindent (b) Find all the equivalence classes for $\equiv_2$ and $\equiv_3$.
How many equivalence classes are there in each case? In general how many 
equivalence classes are there for $\equiv_m$.

There are two equivalence classes for $\equiv_2$. 
$[0]_{\equiv_2} = \{...,-4,-2,0,2,4,...\}$ and $[1]_{\equiv_2} = \{...,-3,-1,1,3,...\}$
(even and odd numbers).

There are three equivalence classes for $\equiv_3$. 
$[0]_{\equiv_3} = \{...,-6,-3,0,3,6,...\}$,
$[1]_{\equiv_3} = \{...,-5,-2,1,4,7,...\}$ and
$[2]_{\equiv_3} = \{...,-4,-1,2,5,8,...\}$

\section*{Exercise 12} Prove that for every integer $n$, 
either $n^2 \equiv 0 (\text{mod } 4)$ or $n^2 \equiv 1 (\text{mod } 4)$

All even integers are $2k$.  So, $4k^2 \equiv 0 (\text{mod } 4)$, which is 
$4 | 4k^2 - 0$ which is obviously true.

All off numbers are $2k+1$. So,

$$(2k+1)^2 \equiv 1 (\text{mod } 4)$$
$$4k^2 + 4k + 1 \equiv (\text{mod } 4)$$
$$4 | 4k^2 + 4k + 1 - 1$$
$$4 | 4k^2 + 4k$$

Which it obviously does.

\section*{Exercise 13}

Suppose $m$ is postivie integer. Prove that for all integers $a$, $a^\prime$,
$b$, and $b^\prime$ if $a^\prime \equiv a (\text{mod } m)$ and 
$b^\prime \equiv (\text{mod } m)$ then $a^\prime + b^\prime \equiv a + b (\text{mod } m)$
and $a^\prime b^\prime \equiv ab (\text{mod } m)$.

$m | a^\prime - a$ and $m | b^\prime - b$ therefore $m | a^\prime + b^\prime -a -b$
which is $a^\prime + b^\prime \equiv a + b (\text{mod } m)$.

We choose some integers $c$ and $d$ such that $a^\prime - a = cm$ and 
$b^\prime - b = dm$. Therefore

$$a^\prime b^\prime = (a + cm)(b + dm) - ab$$
$$a^\prime b^\prime = adm + bcm + cdm^2$$
$$a^\prime b^\prime = (ad + bc + cdm)m$$

So $m | (a^\prime b^\prime -ab)$ and therefore $a^\prime b^\prime \equiv ab (\text{mod } m)$.

\section*{Exercise 14}

Suppose that $R$ is an equivalence relation on $A$ and $B \subseteq A$. Let 
$S = R \cap (B \times B)$

\noindent (a) Prove that $S$ is an equivalence relation on $B$

Suppose $x \in B$. Therefore, $(x, x) \in B \times B$. Since $R$ is an equivalence
relation, then $(x,x) \in R$ due to its own reflexivity. Therefore,
$(x,x) \in R \cap (B \times B)$, so $S$ is reflexive.

Suppose $x \in B$ and $y \in B$ then since $B \subseteq A$, $x \in A$ and $y \in A$.
Since $R$ is an equivalence relation and is symmetric then $(x,y) \in R$ and 
$(y,x) \in R$. Equally $(x,y) \in B \times B$ and $(y,x) \in B \times B$, so 
$R \cap (B \times B)$ is symmetric, so $S$ is symmetric.

Suppose $x,y,z \in B$ so $x,y,z \in A$. Since $R$ is an equivalence relation 
on $A$ it is transitive, so $(x,y) \in R$, $(y,z) \in R$ and $(x,z) \in R$.
Equally $(x,y) \in B \times B$, $(y,z) \in B \times B$ and $(x,z) \in B \times B$,
so $R \cap (B \times B)$ is transitive so $S$ is transitive.

Since $R \cap B \times B$ is reflexive, symmetric and transitive it is an 
equivalence relation.

\noindent (b) Prove that for all $x \in B$, $[x]_S = [x]_R \cap B$.

Suppose $x \in B$, then for every $y$, 
$y \in [x]_S$ iff 
$(y,x) \in R \cap (B \times B)$ iff 
$(y,x) \in R \wedge y \in B$ iff 
$y \in [x]_R \wedge y \in B$ iff
$y \in [x]_R \cap B$.

\section*{Exercise 15}

Suppose $B \subseteq A$, and define a relation $R$ on $\powerset{A}$ as follows

$$R = \{(X,Y) \in \powerset{A} \times \powerset{A} | X \triangle Y \subseteq B\}$$

\noindent (a) Prove that $R$ is an equivalence relation on $\powerset{A}$

For every $X \in \powerset{A}$, $X \triangle X = \emptyset$ which is a subset 
of $B$, so $(X,X) \in \powerset{A} \times \powerset{A}$ so it is reflexive.

Suppose $X \in \powerset{A}$ and $Y \in \powerset{A}$. If 
$X \triangle Y \subseteq B$, then $Y \triangle X \subseteq B$. Therefore 
$(X, Y) \in \powerset{A} \times \powerset{A}$ and 
$(Y, X) \in \powerset{A} \times \powerset{A}$. Therefore, $R$ is symmetric.

Suppose $X, Y, Z \in \powerset{A}$. If $X \triangle Y \subset B$ and 
$Y \triangle Z \subseteq B$ then obviously $X \triangle Z \subseteq B$. 
Therefore $(X, Y) \in \powerset{A} \times \powerset{A}$,
$(Y, Z) \in \powerset{A} \times \powerset{A}$ and $(X, Z) \in \powerset{A} \times \powerset{A}$.
So $R$ is transitive.

Since $R$ is reflexive, symmetric, and transitive it is an equivalence relation.

\noindent (b) Prove that for every $X \in \powerset{A}$ there is exactly on 
$Y \in [X]_R$ such that $Y \cap B = \emptyset$.

Suppose $X \in \powerset{A}$. Let $Y = X \setminus B$. Then $Y \subseteq X$,
so $Y \setminus X = \emptyset$ and so 
$X \setminus Y = X \setminus (X \setminus B) = X \cap B$.
Therefore, 
$Y \triangle X = (Y \setminus X) \cup (X \setminus Y) = \emptyset \cup (X \cap B) = X \cap B \subseteq B$,
so $(Y, X)  \in R$ and $Y \in [X]_R$. Also $Y \cap B = (X \setminus B) \cap B = \emptyset$.
To show that $Y$ is unique, suppose that $Y^\prime \in [X]_R$ and 
$Y^\prime \cap B = \emptyset$. Then $(Y^\prime, X) \in R$, so since $(Y, X) \in R$,
by symmetry and transitivity of $R$, $(Y^\prime, Y) \in R$ so 
$Y^\prime \triangle Y \subseteq B$. Suppose $y \in Y$. If $y \notin Y^\prime$
then $y \in Y \setminus Y^\prime \triangle Y \subseteq B$. But this 
contradicts the fact that $Y$ and $B$ are disjoint. Therefore, $y \in Y^\prime$.
Since $y$ was arbitrary, this proves that $Y \subseteq Y^\prime$. A similar 
argument shows that $Y^\prime \subseteq Y$, so $Y^\prime = Y$.

\section*{Exercise 16} Suppose $\family{F}$ is a partition of $A$, $\family{G}$
is a partition of $B$ and $A$ and $B$ are disjoint. Prove that 
$\family{F} \cup \family{G}$ is a partition of $A \cup B$.

By exercise 16 of section 3.5, 
$\bigcup (\family{F} \cup \family{G}) = \bigcup \family{F} \cup \bigcup \family{G} = A \cup B$.
To see that $\family{F} \cup \family{G}$ is pairwise disjoint, suppose that 
$X \in \family{F} \cup \family{G}$, $Y \in \family{F} \cup \family{G}$ and 
$X \cap Y \neq \emptyset$. If $X \in \family{F}$ and $Y \in \family{G}$
then $X \subseteq A$ and $Y \subseteq B$ and since $A$ and $B$ are disjoint 
it follows that $X$ and $Y$ are disjoint, which is a contradiction. Thus, it 
can not be the case $X \in \family{F}$ and $Y \in \family{G}$ and a similar 
argument rules out the possibility that $Y \in \family{F}$ and $X \in \family{G}$.
Thus, $X$ and $Y$ are both either elements of $\family{F}$ or both elements 
of $\family{G}$. If they are both in $\family{F}$ then since $\family{F}$ is 
pairwise disjoint $X= Y$. A similar argument applies if they are both in 
$\family{G}$. Finally we have $\forall X \in \family{F} (X \neq \emptyset)$
and $\forall X \in \family{G} (X \neq \emptyset)$ and it follows by exercise 8
of 2.2 that $\forall X \in \family{F} \cup \family{G} (X \neq \emptyset)$

\section*{Exercise 17}

Suppose $R$ is an equivalence relation on $A$, $S$ is an equivalence relation 
on $B$ and $A$ and $B$ are disjoint. 

\noindent (a) Prove that $R \cup S$ is an equivalence relation on $A \cup B$.

Suppose $x \in A$ and $y \in B$. Since $R$ is an equivalence relation on $A$ then 
$(x,x) \in R$. similarly, $(y,y) \in S$. Obviously, $x \in A \cup B$ and $y \in A \cup B$.
Similarly $(x,x) \in R \cup S$ and $(y,y) \in R \cup S$. Therefore, $R \cup S$ 
is reflexive.

Suppose $x,y \in A$ and $x^\prime, y^\prime \in B$. Since $R$ is an equivalence
relation on $A$ then $(x,y) \in R$ and $(y,x) \in R$. Similarly, 
$(x^\prime, y^\prime) \in S$ and $(y^\prime, x^\prime) \in S$. Obviously
$x, y, x^\prime, y^\prime \in A \cup B$, but so is 
$(x,y),(y,x),(x^\prime,y^\prime),(y^\prime,x^\prime) \in R \cup S$ so it is 
symmetric.

Suppose $x,y,z \in A$ and $x^\prime, y^\prime, z^\prime \in B$. Since $R$ is 
an equivalence relation on $A$ then $(x,y), (y,z), (x,z) \in R$. Similarly,
$(x^\prime, y^\prime), (y^\prime, z^\prime), (x^\prime, z^\prime) \in S$.
So $(x,y), (y,z), (x,z), (x^\prime, y^\prime), (y^\prime, z^\prime), (x^\prime, z^\prime) \in R \cup S$
so it is transitive.

\noindent (b) Prove that for all $x \in A$, $[x]_{R \cup S} = [x]_R$ and for all 
$y \in B$, $[y]_{R \cup S} = [y]_S$

Suppose $x \in A$. $A$ and $B$ are pairwise disjoint so $x \notin B$. Now 
$[x]_{R \cup S} = \{y \in A \cup B | y R \cup S x\}$. But, $x$ is not relevant to 
$S$ since $S$ is the equivalence relation on $B$. So, 
$[x]_{R \cup S} = \{y \in A | yRx\} = [x]_R$. Given $y \in B$ is a similar argument 
that shows that $[y]_{R \cup S} = [y]_s$.

\noindent (c) Prove that $(A \cup B) / (R \cup S) = (A / R) \cup (B / S)$

($\rightarrow$) Suppose $X \in (A \cup B) / (R \cup S)$. Then for some 
$x \in A \cup B$, $X = [x]_{R \cup S}$. Since $x \in A \cup B$, either 
$x \in A$ or $x \in B$. By part (b) if $x \in A$ then $[x]_{R \cup S} = [x]_R$
and if $x \in B$ then $[x]_{R \cup S} = [x]_S$. Thus either $X = [x]_R \in A /R$
or $X = [x]_S \in B/S$ so $X \in (A/R) \cup (B/S)$.

($\leftarrow$) Now suppose $X \in (A/R) \cup (B/S)$. Either $X \in (A/R)$ or 
$X \in (B/S)$. If $X \in (A/R)$ for some $x \in A$, $X = [x]_R$ and by part (b)
it follows that $X = [x]_{R \cup S} \in (A \cup B) / (R \cup S)$. Similarly,
if $X \in B/S$ then for some $x \in B$, $X = [x]_{R \cup S} \in (A \cup B)/(R \cup S)$.
Thus, $(A \cup B)/(R \cup S) = (A/R) \cup (B/S)$.

\section*{Exercise 18}

Suppose $\family{F}$ and $\family{G}$ are partitions of a set $A$. We define a 
new family of sets $\family{F} \cdot \family{G}$ as follows

$$\family{F} \cdot \family{G} = \{Z \in \powerset{A} | Z \neq \emptyset 
\text{ and } \exists X \in \family{F} \exists Y \in \family{G} (Z = X \cap Y)\}$$

Since $\family{F} \cdot \family{G} \subseteq \powerset{A}$ then 
$\bigcup (\family{F} \cdot \family{G}) \subseteq A$. Suppose $x \in A$. Since 
$\family{F}$ and $\family{G}$ are partitions of $A$ there are sets $X \in \family{F}$
and $Y \in \family{G}$ such that $x \in X$ and $x \in Y$. Therefore 
$x \in X \cap Y \in \family{F} \cdot \family{G}$, so 
$x \in \bigcup (\family{F} \cdot \family{G})$. This proves that 
$\bigcup (\family{F} \cdot \family{G}) = A$.

Suppose $Z_1, Z_2 \in \family{F} \cdot \family{G}$ and $Z_1 \cap Z_2 \neq \emptyset$.
Since $Z_1 \in \family{F} \cdot \family{G}$ we can choose sets $X_1 \in \family{F}$
and $Y_1 \in \family{G}$ such that $Z_1 = X_1 \cap Y_1$. Similarly, since 
$Z_2 \in \family{F} \cdot \family{G}$ we can choose $X_2 \in \family{F}$
and $Y_2 \in \family{G}$ such that $Z_2 = X_2 \cap Y_2$. Since 
$Z_1 \cap Z_2 \neq \emptyset$ we can choose some $z \in Z_1 \cap Z_2$. Thus, 
$z \in Z_1 = X_1 \cap Y_1$ and $z \in Z_2 = X_2 \cap Y_2$, so 
$z \in X_1$, $z \in X_2$, $z \in Y_1$, $z \in Y_2$. Thus $X_1$ and $X_2$ are 
not disjoint, and neither is $Y_1$ and $Y_2$. Since $\family{G}$ is pairwise 
disjoint $Y_1 = Y_2$, and similarly $X_1 = X_2$. Therefore 
$Z_1 = X_1 \cap Y_1 = X_2 \cap Y_2 = Z_2$. This proves that 
$\family{F} \cdot \family{G}$ is pairwise disjoint.

Thus, $\family{F} \cdot \family{G}$ is a partition on $A$.

\section*{Exercise 19}

Let $\family{F} = \{\mathbb{R}^-, \mathbb{R}^+, \{0\}\}$ and 
$\family{G} = \{\mathbb{Z}, \mathbb{R} \setminus \mathbb{Z}\}$, and note that 
both $\family{F}$ and $\family{G}$ are partitions of $\mathbb{R}$. List the 
elements of $\family{F} \cdot \family{G}$.

$$\family{F} \cdot \family{G} = \{\mathbb{Z}^-, \mathbb{R}^- \setminus \mathbb{Z}^-,
\mathbb{Z}^+, \mathbb{R}^+ \setminus \mathbb{Z}^+, \{0\}\}$$

\section*{Exercise 20}

Suppose $R$ and $S$ are equivalence relations on a set $A$. Let $T = R \cap S$.

\noindent (a) Prove that $T$ is an equivalence relation on $A$.

Suppose $x \in A$. Since both $R$ and $S$ are equivalence relations on $A$ then
$(x,x) \in R$ and $(x,x) \in S$, so $(x,x) \in R \cap S$. So, $T = R \cap S$ is 
reflexive.

Suppose $x,y \in A$. Since $R$ and $S$ are both equivalence relations then 
$(x,y) \in R \cap S$ and $(y,x) \in R \cap S$, so $T$ is symmetric.

Suppose $x,y,z \in A$. Since $R$ is an equivalence relation
$(x,y) \in R$, $(y,z) \in R$, $(x,z) \in R$. The same is true for $S$ Thus
$(x,y), (y,z), (x,z) \in T$. Therefore $T$ is transitive. 

\noindent (b) Prove that for all $x \in A$, $[x]_T = [x]_R \cap [x]_S$.

Suppose $x \in A$, then for every $y$, $y \in [x]_T$ iff 
$(y,x) \in T$ iff
$(y,x) \in R \wedge (y,x) \in S$ iff
$y \in [x]_R \wedge y \in [x]_S$ iff 
$y \in [x]_R \cap [x]_S$ which shows that 
$[x]_T = [x]_R \cap [x]_S$.

\noindent (c) Prove that $A / T = (A/R) \cdot (A/S)$

($\leftarrow$) Suppose $X \in A/T$. Then for some $x \in A$, $X = [x]_T$. By 
part (b) $[x]_T = [x]_R \cap [x]_S$. Since $[x]_R \in A/R$ and $[x]_S \in A/S$,
$X \in (A/R) \cdot (A/S)$

($\rightarrow$) Suppose $X \in (A/R) \cdot (A/S)$. For some $y$ and $z$ in $A$,
$X = [y]_R \cap [z]_S$. Also, $X \neq \emptyset$ so we can choose some $x \in X$.
Therefore $x \in [y]_R$ and $x \in [z]_S$ and by part 2 of lemma 4.5.5 it follows 
that $[x]_R = [y]_R$ and $[x]_S = [z]_S$. 
Therefore, $X = [x]_R \cap [x]_S = [x]_T \in A/T$

\section*{Exercise 21}

Suppose $\family{F}$ is a partition of $A$ and $\family{G}$ is a partition of $B$.
We define a new family of sets $\family{F} \otimes \family{G}$ as follows 

$$\family{F} \otimes \family{G} = \{Z \in \powerset{A \times B} | 
\exists X \in \family{F} \exists Y \in \family{G} (Z = X \times Y)\}$$

Prove that $\family{F} \otimes \family{G}$ is a partition of $A \times B$

Since $\family{F}$ and $\family{G}$ are partitions, then neither contain 
$\emptyset$. Thus, for $X \in \family{F}$, $Y \in \family{G}$, $Z = X \times Y$,
$Z$ can never be $\emptyset$.

Since $\family{F} \otimes \family{G} \subseteq \powerset{A \times B}$
then $\bigcup (\family{F} \otimes \family{G}) \subseteq A \times B$.
Suppose $x \in A$ and $y \in B$. Since $\family{F}$ is a partition on $A$ then 
$X \in \family{F}$ where $x \in X$. Similarly $Y \in \family{G}$ where $y \in Y$.
Therefore, $(x,y) \in X \times Y \in \family{F} \otimes \family{G}$. So, 
$(x,y) \in \bigcup (\family{F} \otimes \family{G})$. This proves that 
$\bigcup (\family{F} \otimes \family{G}) = A \times B$.

Suppose $Z_1, Z_2 \in \family{F} \otimes \family{G}$ and 
$Z_1 \cap Z_2 \neq \emptyset$. Since $Z_1 \in \family{F} \otimes \family{G}$ 
we can choose some $X_1 \in \family{F}$ and $Y_1 \in \family{G}$ such that 
$Z_1 = X_1 \times Y_1$. Similarly we can choose $X_2 \in \family{F}$ and 
$Y_2 \in \family{G}$ such that $Z_2 = X_2 \times Y_2$. Since 
$Z_1 \cap Z_2 \neq \emptyset$ we can choose some $(a, b) \in A \times B$ such that 
$(a,b) \in Z_1 \cap Z_2$. Thus $(a,b) \in Z_1 = X_1 \times Y_1$ and 
$(a,b) \in Z_2 = X_2 \times Y_2$ so $a \in X_1$, $a \in X_2$, $b \in Y_1$ and 
$b \in Y_2$. Since $a \in X_1$ and $a \in X_2$, then $X_1$ and $X_2$ are not 
disjoint so since $\family{F}$ is pairwise disjoint then $X_1 = X_2$. Similarly,
$Y_1 = Y_2$. Therefore, $Z_1 = X_1 \times Y_1 = X_2 \times Y_2 = Z_2$ so 
$\family{F} \otimes \family{G}$ is pairwise disjoint

\section*{Exercise 22}

Let $\family{F} = \{\mathbb{R}^-, \mathbb{R}^+, \{0\}\}$, which is a partition
of $\mathbb{R}$. List the elements of $\family{F} \otimes \family{F}$ and 
describe them geometrically as subsets of the xy-plane

$$\family{F} \otimes \family{F} = \{
\mathbb{R}^+ \times \mathbb{R}^+,
\mathbb{R}^- \times \mathbb{R}^-,
\mathbb{R}^- \times \mathbb{R}^+,
\mathbb{R}^+ \times \mathbb{R}^-,
\mathbb{R}^+ \times \{0\},
\mathbb{R}^- \times \{0\},
\{0\} \times \mathbb{R}^+,
\{0\} \times \mathbb{R}^-,
\{(0,0)\}
\}$$

These are the four quadrants of the plane, the x and y axis, as the origin

\section*{Exercise 23}

Suppose $R$ is an equivalence relation on $A$ and $S$ is an equivalence
relation on $B$. Define a relation $T$ on $A \times B$ as follows 

$$T = \{((a,b), (a^\prime, b^\prime)) \in (A \times B) \times (A \times B) |
aRa^\prime \text{ and } bSb^\prime\}$$

\noindent (a) Prove that $T$ is an equivalence relation on $A \times B$

Suppose $(x,y) \in A \times B$. Since $R$ is reflexive on $A$ and $S$ is reflexive
on $B$ then $((x,y),(x,y)) \in (A \times B) \times (A \times B)$ so $T$ is 
reflexive

Suppose $((x,y), (x^\prime, y^\prime)) \in T$. Since $R$ and $S$ are symmetric,
then $((x^\prime, y^\prime), (x,y))$ is also in $T$, so $T$ is symmetric

Suppose $((x,y), (x^\prime, y^\prime)) \in T$ and 
$((x^\prime, y^\prime), (x^\dprime, y^\dprime)) \in T$. Since $R$ and $S$ are
both transitive, then $((x,y), (x^\dprime, y^\dprime)) \in T$. Therefore
$T$ is transitive.

\noindent (b) Prove that if $a \in A$ and $b \in B$ then $[(a,b)]_T = [a]_R \times [b]_S$

Suppose $a \in A$ and $b \in B$ then for all $(x,y) \in A \times B$,
$(x,y) \in [(a,b)]_T$ iff 
$((x,y), (a,b)) \in T$ iff 
$xRa \wedge ySb$ iff 
$x \in [a]_R \wedge y \in [b]_S$ iff 
$(x,y) \in [a]_R \times [b]_S$

\noindent (c) Prove that $(A \times B)/T = (A/R) \otimes (B/S)$

($\leftarrow$) Suppose $X \in (A \times B)/T$. 
Then, for some $(x,y) \in A \times B$, $X = [(x,y)]_T$. By part b,
$[(x,y)]_T = [x]_R \times [y]_S$. Since $[x]_R = A/R$ and $[y]_S = B/S$, 
then $X \in (A/R) \otimes (B/S)$

($\rightarrow$) Suppose $X \in (A/R) \otimes (B/S)$. Then we can choose some 
$X \in A/R$ and $Y \in B/S$ such that $Z = X \times Y$. Since $X \in A/R$
and $Y \in B/S$ we can choose $a \in A$ and $b \in B$ such that $X = [a]_R$ 
and $Y = [b]_S$. Therefore $Z = X \times Y = [a]_R \times [b]_S = [(a,b)]_T 
\in (A \times B) / T$. Thus, $(A \times B) / T = (A / R) \otimes (B/S)$.

\section*{Exercise 24}

Suppose $R$ and $S$ are relations on set $A$, and $S$ is an equivalence reltion.
We will say that $R$ is \textit{compatible} with $S$ if for all $x, y, x^\prime,
y^\prime$ in $A$ if $xSx^\prime$ and $ySy^\prime$ then $xRy$ iff $x^\prime R y^\prime$

\noindent (a) Prove that if $R$ is compatible with $S$, then there is a unique 
relation $T$ on $A/S$ such that for all $x$ and $y$ in $A$, $[x]_S T [y]_S$
iff $xRy$.

Suppose $R$ is compatible with $S$. Let 
$T = \{(X, Y) \in A/S \times A/S | \exists x \in X \exists y \in Y (xRy)\}$.
Let $x$ and $y$ be arbitrary elements of $A$. Suppose $[x]_S T [y]_S$. Then, by 
the definition of $T$ there are $x^\prime \in [x]_S$ and $y^\prime \in [y]_S$
such that $x^\prime R y^\prime$. Since $x^\prime \in [x]_S$ and 
$y^\prime \in [y]_S$, $x^\prime Sx$ and $y^\prime S y$ so since $R$ is 
compatible with $S$, $xRy$. Now suppose $xRy$. Since $x \in [x]_S$ and 
$y \in [y]_S$ it follows that $[x]_S T [y]_S$.

To see that $T$ is unique, suppose $T^\prime$ is another relation with 
the required properties. Let $X$ and $Y$ be arbitrary elements of $A/S$.
Then we choose some $x,y \in A$ such that $X = [x]_S$ and $Y = [y]_S$.
Therefore, $(X, Y) \in T^\prime$ iff $[x]_S T^\prime [y]_S$ iff $xRy$ iff 
$[x]_S T [y]_S$ iff $(X, Y) \in T$. So $T^\prime = T$.

\noindent (b) Suppose $T$ is a relation on $A/S$ and for all $x$ and $y$ in $A$,
$[x]_S T [y]_S$ iff $xRy$. Prove that $R$ is compatible with $S$.

Suppose $x,y,x^\prime,y^\prime \in A$, $xSx^\prime$ and $ySy^\prime$. Then 
$[x]_S = [x^\prime]_S$ and $[y]_S = [y^\prime]_S$ so $xRy$ iff $[x]_S T [y]_S$
iff $[x^\prime]_S T [y^\prime]_S$ iff $x^\prime R y^\prime$.

\section*{Exercise 25}

Suppose $R$ is a relation on $A$ and $R$ is reflexive and transitive. (Such a 
relation is called a \textit{preorder} on A). Let $S = R \cap R^{-1}$.

\noindent (a) Prove that $S$ is an equivalence relation on $A$.

$R$ is reflexive on $A$. Therefore, for $x \in A$, $(x,x) \in R$. Clearly, 
$(x,x) \in R^{-1}$ so $(x,x) \in R \cap R^{-1}$. Therefore, $S$ is reflexive.

Suppose $(x,y) \in S$. Then $(x,y) \in R$ and $(x,y) \in R^{-1}$. Equally,
$(y,x) \in R$ and $(y,x) \in R^{-1}$. So, $(y,x) \in S$ so $S$ is symmetric.

Suppose $(x,y) \in S$ and $(y,z) \in S$. Then $(x,y) \in R$ and $(y,z) \in R$,
so since $R$ is transitive then $(x,z) \in R$. Similarly, $(x,y) \in R^{-1}$
and $(y,z) \in R^{-1}$ and $R^{-1}$ is also transitive so $(x,z) \in R^{-1}$,
so $(x,z) \in S$. Therefore, $S$ is transitive

\noindent (b) Prove that there is a unique relation $T$ on $A/S$ such that for 
all $x$ and $y$ in $A$, $[x]_S T [y]_S$ iff $xRy$.

Let $x,y,x^\prime,y^\prime \in A$ and suppose $xSx^\prime$ and $ySy^\prime$.
Then $(x,x^\prime) \in S = R \cap R^{-1}$ and $(y,y^\prime) \in S = R \cap R^{-1}$
so $xRx^\prime$, $x^\prime Rx$, $yRy^\prime$ and $y^\prime Ry$. Now suppose 
$xRy$. By transitivity of $R$, since $x^\prime Rx$ and $xRy$, $x^\prime Ry$.
But since $yRy^\prime$ by using transitivity again we see $x^\prime R y^\prime$.
A similar argument shows that if $x^\prime R y^\prime$ then $xRy$. Therefore,
$R$ is compatible with $S$, so exercise 24 completes the proof.

\noindent (c) Prove that $T$ is a partial order on $A/S$, where $T$ is the relation 
from part (b)

Suppose $(X,Y) \in A/S$. Then, $((X, Y), (X,Y)) \in A/S \times A/S$,
so $((X, Y), (X,Y)) \in T$ so $T$ is reflexive.

Suppose $((X, Y), (X^\prime, Y^\prime)) \in T$. Suppose also 
$((X^\prime, Y^\prime), (X^\dprime, Y^\prime)) \in T$. Then 
$(X, Y), (X^\prime, Y^\prime), (X^\dprime, Y^\prime) \in A/S$, so 
$((X, Y), (X^\dprime, Y^\prime)) \in A/S \times A/S$, so 
$((X, Y), (X^\dprime, Y^\prime)) \in T$. So, $T$ is transitive.

Suppose $X, Y \in A/S$, $XTY$ and $YTX$. Choose $x,y \in A$ such that $X = [x]_S$
and $Y = [y]_S$. Then $[x]_S T [y]_S$ and $[y]_S T [x]_S$, so $xRy$ and $yRx$.
Therefore, $(x,y) \in R \cap R^{-1} = S$ so $[x]_S = [y]_S$, which means 
$X=Y$.

\section*{Exercise 26}

Let $I = \{1,2,...,100\}$, $A = \powerset{I}$ and 
$R = \{(X,Y) \in A \times A | Y \text{ has at least as many elements as } X\}$.

\noindent (a) Prove that $R$ is a preorder on $A$.

Suppose $X \in A$. $X$ has at least as many elements as $X$ so $(X,X) \in R$.
Therefore, $R$ is transitive.

Suppose $(X, Y) \in R$ and $(Y, Z) \in R$. Since this means $Z$ has as least as 
many elements as $X$ then $(X, Z) \in R$. Therefore, $R$ is transitive.

As a bonus, a preorder is not symmetric so to prove $R$ is not symmetric we suppose 
$(X, Y) \in R$. Then, $Y$ has at least as many elements as $X$, but $X$ does not 
have at least as many elements as $Y$, so $(Y, X) \notin R$.

\noindent (b) Let $S$ and $T$ be defined as in exercise 25. Describe the elements 
of $A/S$ and the partial order $T$. How many elements does $A/S$ have? Is $T$ a 
total order?

For all $X$ and $Y$ in $A$, $(X, Y) \in S$ iff $X$ and $Y$ have the same number
of elements. Let $J = \{0, 1,...,100\}$ and for each $j \in J$ let 
$P_j = \{X \in \powerset{I} | X \text{ has exactly } j \text{ elements}\}$. 
Then $A/S = \{P_0, P_1,...,P_100\} = \{P_j | j \in J\}$, so $A/S$ has 101 elements.
$T = \{(P_j, P_k) | j \in J, k \in J, j \leq k\}$. $T$ is therefore a total order.

\section*{Exercise 27}

Suppose $A$ is a set. If $\family{F}$ and $\family{G}$ are partitions of $A$,
then we'll say that $\family{F}$ \textit{refines} $\family{G}$ if 
$\forall X \in \family{F} \exists Y \in \family{G} (X \subseteq Y)$. Let $P$ be 
the set of all partitions of $A$ and let 
$R=\{(\family{F}, \family{G}) \in P \times P | \family{F} \textit{ refines } \family{G}\}$.

\noindent (a) Prove that $R$ is a partial order on $P$.

Suppose $\family{F} \in P$. Then for every $X \in \family{F}$, $X \subseteq X$,
so $\forall X \in \family{F} \exists Y \in \family{F} (X \subseteq Y)$ and therefore 
$(\family{F}, \family{F}) \in R$. Thus, $R$ is reflexive.

Suppose $(\family{F}, \family{G}) \in R$ and $(\family{G}, \family{H}) \in R$.
Then $\family{F}$ refines $\family{G}$ and $\family{G}$ refined $\family{H}$.
Let $X \in \family{F}$ be arbitrary. Since $\family{F}$ refines $\family{G}$
we can choose some $Y \in \family{G}$ such that $X \subseteq Y$. Since $\family{G}$
refines $\family{H}$ we can choose some $Z \in \family{H}$ such that $Y \subseteq Z$.
Since $X \subseteq Y$ and $Y \subseteq Z$ then $X \subseteq Z$. Therefore
$\forall X \in \family{F} \exists Z \in \family{H} (X \subseteq Z)$. So,
$\family{F}$ refines $\family{H}$ and therefore $(\family{F}, \family{H}) \in R$.
Thus, $R$ is transitive.

Suppose $(\family{F}, \family{G}) \in R$ and $(\family{G}, \family{F}) \in R$, 
then $\family{F}$ refines $\family{G}$ and $\family{G}$ refines $\family{F}$.
Let $X \in \family{F}$ be arbitrary. Then, since $\family{F}$ refines $\family{G}$,
there is some $Y \in \family{G}$ such that $X \subseteq Y$. Since $\family{G}$ 
refines $\family{F}$ there is some $X^\prime \in \family{F}$ such that 
$Y \subseteq x^\prime$. Since $X \subseteq Y$ and $Y \subseteq X^\prime$
then $X \subseteq X^\prime$. But then $X \cap X^\prime = X \neq \emptyset$, so 
since $\family{F}$ is pairwise disjoint $X^\prime = X$. Therefore $X \subseteq Y$
and $Y \subseteq X^\prime = X$, so $X = Y \subseteq \family{G}$. Since $X$ was 
an arbitrary element of $\family{F}$ this shows that $\family{F} \subseteq \family{G}$.
A similar argument shows that $\family{G} \subseteq \family{F}$, so 
$\family{F} = \family{G}$ and $R$ is antisymmetric.

\noindent (b) Suppose that $S$ and $T$ are equivalence relations on $A$. Let 
$\family{F} = A/S$ and $\family{G} = A/T$. Prove that $S \subseteq T$ iff 
$\family{F}$ refines $\family{G}$

($\rightarrow$) Suppose $S \subseteq T$. Suppose $X \in \family{F} = A/S$.
Then we can choose some $x \in A$ such that $X = [x]_S$. Let 
$Y = [x]_T \in A/T = \family{G}$. Let $a$ be an arbitrary element of $X$. Then 
$a \in [x]_S$, so $(a,x) \in S$. Since $S \subseteq T$, $(a,x) \in T$ so 
$a \in [x]_T = Y$. Since $a$ was arbitrary, then $X \subseteq Y$. Since $X$ was 
arbitrary, this shows that $\family{F}$ refines $\family{G}$.

($\leftarrow$) Suppose $\family{F}$ refines $\family{G}$. Suppose $(x,y) \in S$.
Let $Y = [y]_S \in A/S = \family{F}$. Since $\family{F}$ refines $\family{G}$,
there is some $Z \in \family{G} = A/T$ such that $Y \subseteq Z$. Since 
$Z \in A/T$ we can choose some $z \in A$ such that $Z = [z]_T$. By Lemma 
4.5.5, $y \in [y]_S = Y \subseteq Z = [z]_T$, so $[y]_T = [z]_T$. Since 
$(x,y) \in S$, $x \in [y]_S \subseteq [z]_T = [y]_T$, so $(x,y) \in T$.
Since $(x,y)$ was arbitrary then $S \subseteq T$.

\noindent (c) Suppose that $\family{F}$ and $\family{G}$ are partitions of $A$.
Prove that $\family{F} \cdot \family{G}$ is the greatest lower bound of the set 
$\{\family{F}, \family{G}\}$ in the partial order $R$.

Suppose $Z \in \family{F} \cdot \family{G}$. Then we can choose $X \in \family{F}$
and $Y \in \family{G}$ such that $Z = X \cap Y$. Therefore, $Z \subseteq X$ and
$Z \subseteq Y$. Since $Z$ was arbitrary, this shows that 
$\family{F} \cdot \family{G}$ refines both $\family{F}$ and $\family{G}$
so $\family{F} \cdot \family{G}$ is a lower bound for $\{\family{f}, \family{G}\}$.
Suppose now that $\family{H}$ is a lower bound. Then $\family{H}$ refines both 
$\family{F}$ and $\family{G}$. Suppose $Z \in \family{H}$. Since $\family{H}$
refines both $\family{F}$ and $\family{G}$, we can choose some $X \in \family{F}$
and $Y \in \family{G}$ such that $Z \subseteq X$ and $Z \subseteq Y$.
Therefore, $Z \subseteq X \cap Y$. Since $\family{H}$ is a partition and 
$Z \in \family{H}$ then $Z \neq \emptyset$, so $X \cap Y \neq \emptyset$
and so $X \cap Y \in \family{F} \cdot \family{G}$. Since $Z$ was arbitrary,
this shows that $\family{H}$ refines $\family{F} \cdot \family{G}$, so 
$(\family{H}, \family{F} \cdot \family{G}) \in R$. Since $\family{H}$ was arbitrary,
this shows that $\family{F} \cdot \family{G}$ is the greatest lower bound of 
$\{\family{F}, \family{G}\}$.

\end{document}
