\documentclass[11pt]{article}
\usepackage{amssymb}
\usepackage{tipa}
\usepackage{amsmath}
\usepackage[scr]{rsfso}
\usepackage{graphicx}
\usepackage{float}


\newcommand{\then}{\rightarrow} 
\newcommand{\bicond}{\leftrightarrow}
\newcommand{\powerset}[1]{\mathscr{P}(#1)}
\newcommand{\family}[1]{\mathcal{#1}}

\title{\textbf{How to Prove It} \\ {\Large\itshape Daniel J. Velleman} \\ {\Large\itshape Chapter 3.5: Proofs Involving Disjunctions}}

\author{\textbf{Nathaniel Curnick} \\ \textit{Textbook Solutions}}

\date{}

%----------------------------------------------------------------------------------------

\begin{document}

\maketitle

\section*{Exercise 1}

Find the domains and ranges of the following relations

\noindent (a) $\{(p, q) \in P \times P | \text{the person } p \text{ is a parent of the person } q\}$
where $P$ is the set of all living people 

Domain is $\{p \in P | p \text{ has a living child}\}$

Range is $\{p \in P | p \text{ has a living parent}\}$

\noindent (b) $\{(x, y) \in \mathbb{R}^2 | y > x^2\}$

Domain is $\mathbb{R}$

Range is $\mathbb{R}^2$

\section*{Exercise 2}

Find the domains and ranges of the following relations 

\noindent (a) $\{(p, q) \in P \times P | \text{the person } p \text{ is a brother of the person } q\}$
where $P$ is the set of all living people 

Domain is $\{p \in P | p \text{ has a brother}\}$

Range is $\{q \in P | q \text{ has a brother}\}$

\noindent (b) $\{(x, y) \in \mathbb{R}^2 | y^2 = 1 - 2/(x^2 + 1)\}$

Domain is $\mathbb{R}$

Range is $-1 < x < 1$

\section*{Exercise 3}

Let $L$ and $E$ be the relations defined in parts 4 and 5 of Example 4.2.2.
Describe the following relations

\noindent (a) $L^{-1} \circ L$

$\{(s, t) \in S \times S | \text{the students } s \text{ and } t \text{ that live in the same dorm}\}$

\noindent (b) $E \circ (L^{-1} \circ L)$

$\{(s,c) \in S \times C | \text{someone who lives with student } s \text{ is taking course } c\}$

\section*{Exercise 4}

Let $E$ and $T$ be the relations defined in parts 5 and 6 of Example 4.2.2.
Also, as in that example, let $C$ be the set of all courses at your school, and 
let $D = \{ \text{Monday, Tuesday, Wednesday, Thursday, Friday}\}$. Let 
$M = \{(c, d) \in C \times D | \text{the course } c \text{ meets on the day } d\}$.
Describe the following relations

\noindent (a) $M \circ E$ 

$\{(s,d) \in S \times D | \text{the student } s \text{ does a course that meets on day } d\}$

\noindent (b) $M \circ T^{-1}$

$\{(p,d) \in P \times D | \text{the professor } p \text{ teaches a course on day } d\}$

\section*{Exercise 5}

Suppose that $A = \{1,2,3\}, B = \{4,5,6\}, R = \{(1,4),(1,5),(2,5),(3,6)\}$ and 
$S = \{(4,5),(4,6),(5,4),(6,6)\}$. Note that $R$ is a relation from $A$ to $B$ 
and $S$ is a relation from $B$ to $B$. Find the following relations 

\noindent (a) $S \circ R$

$S \circ R = \{(5,4),(6,4),(4,5),(6,6)\}$

\noindent (b) $S \circ S^{-1}$

$S \circ S^{-1} = \{(5,5),(5,6),(6,5),(6,6),(4,4)\}$

\section*{Exercise 6}

Suppose that $A = \{1,2,3\}, B = \{4,5\}, C = \{6,7,8\}, R = \{(1,7), (3,6), (3,7)\}$
and $S = \{(4,7), (4,8), (5,6)\}$. Note that $R$ is a relation from $A$ to $C$ 
and $S$ is a relation from $B$ to $C$. Find the following relations 

\noindent (a) $S^{-1} \circ R$ 

$S^{-1} \circ R = \{(1,4),(3,5),(3,4)\}$

\noindent (b) $R^{-1} \circ S$

$R^{-1} \circ S = \{(4,1),(4,3),(5,3)\}$


\section*{Exercise 7}

\noindent (a) Prove part 3 of Theorem 4.2.5 by imitating proof of part 2 in the text

We need to prove that $Ran(R^{-1}) = Dom(R)$

Obviously both are subsets of $A$ and let $a$ be an arbitrary element of $A$

$$a \in Ran(R^{-1}) \text{ iff } \exists b \in B ((b, a) \in R^{-1})$$
$$a \in Ran(R^{-1}) \text{ iff } \exists b \in B ((a, b) \in R \text{ iff } a \in Dom(R))$$

\noindent (b) Give an alternative proof of part 3 of Theorem 4.2.5 by showing that 
it follows from parts 1 and 2 

$$Ran(R^{-1}) = Dom((R^{-1})^{-1})$$
$$Ran(R^{-1}) = Dom(R)$$

\noindent (c) Complete the proof of part 4 of Theorem 4.2.5 

We need to show that $(a,d) \in T \circ (S \circ R)$. Suppose then that 
$(a, d) \in (T \circ S) \circ R$. By definition we can choose some 
$(b, d) \in T \circ S$. Equally we can choose some $(a, b) \in R$. Since 
$(b, d) \in T \circ S$ we can again use the definition and choose some 
$c \in C$ so that $(b, c) \in S$ and $(c, d) \in T$. Since $(b, c) \in S$ and 
$(a,b) \in R$ we can conclude that $(a,c) \in S \circ R$. Again using the definition 
of composition we can conclude $(a, d) \in T \circ (S \circ R)$

\noindent (d) Prove part 5 of Theorem 4.2.5 

We need to prove $(S \circ R)^{-1} = R^{-1} \circ S^{-1}$. Note that 
$(S \circ R)^{-1}$ and $R^{-1} \circ S^{-1}$ are both relations from $C$ to $A$. 
Consider $(c, a) \in C \times A$

$$(c, a) \in (S \circ R)^{-1} \text{ iff } (a, c) \in S \circ R$$
$$(c, a) \in (S \circ R)^{-1} \text{ iff } \exists b \in B ((a, b) \in R \wedge (b, c) \in S)$$
$$(c, a) \in (S \circ R)^{-1} \text{ iff } \exists b \in B ((c,b) \in S^{-1} \wedge (b, a) \in R^{-1})$$
$$(c, a) \in (S \circ R)^{-1} \text{ iff } (c,a) \in R^{-1} \circ S^{-1}$$

\section*{Exercise 8}

Let $E = \{(p,q) \in P \times P | \text{the person } p \text{ is an enemy of the person } q\}$,
and $F = \{(p,q) \in P \times P | \text{the person } p \text{ is a friend of the person } q\}$,
where $P$ is the set of all people. What does the saying ``an enemy of one's enemy is my friend''
mean about the relations $E$ and $F$?

$$E \circ F \subseteq F$$

\section*{Exercise 9}

Suppose $R$ is a relation from $A$ to $B$ and $S$ is a relation from $B$ to $C$

\noindent (a) Prove that $Dom(S \circ R) \subseteq Dom(R)$

Suppose $a \in Dom(S \circ R)$. We can choose some $c \in C$ such that 
$(a,c) \in S \circ R$. This means we can choose some $b \in B$ such that 
$(a, b) \in R$ and $(b,c) \in S$. Since $(a,b) \in R$, $a \in Dom(R)$.

\noindent (b) Prove that if $Ran(R) \subseteq Dom(S)$ then $Dom(S \circ R) = Dom(R)$

Suppose $Ran(R) \subseteq Dom(S)$. In the last question, we proved that 
$Dom(S \circ R) \subseteq Dom(R)$, we just need to prove that $Dom(R) \subseteq Dom(S \circ R)$
(when $Ran(R) \subseteq Dom(S)$). Suppose $a \in Dom(R)$. Then we can choose some 
$b \in B$ such that $(a,b) \in R$. Since $(a,b) \in R$, $b \in Ran(R)$ and since 
$Ran(R) \subseteq Dom(S)$ it follows that $b \in Dom(S)$. Thus, we can choose 
some $c \in C$ such that $(b,c) \in S$. Since $(a, b) \in R$ and $(b, c) \in S$,
$(a, c) \in S \circ R$ and therefore $a \in Dom(S \circ R)$.

\section*{Exercise 10}

Suppose $R$ and $S$ are relations from $A$ to $B$. Must the following statements 
be true? Justify your answers with proofs or counterexamples.

\noindent (a) $R \subseteq Dom(R) \times Ran(R)$

This is true. Suppose $(a, b) \in R$ then $a \in Dom(R)$ and $b \in Ran(R)$ so 
$(a,b) \in Dom(R) \times Ran(R)$

\noindent (b) if $R \subseteq S$ then $R^{-1} \subseteq S^{-1}$

This is true. Suppose $R \subseteq S$ and suppose $(b,a) \in R^{-1}$. Then 
$(a,b) \in R$ so since $R \subseteq S$, $(a,b) \in S$. Therefore $(b,a) \in S^{-1}$

\noindent (c) $(R \cup S)^{-1} = R^{-1} \cup S^{-1}$

This is also true. Let $(b,a) \in B \times A$ then 

$$(b, a) \in (R \cup S)^{-1} \text{ iff } (a,b) \in R \cup S$$
$$(b, a) \in (R \cup S)^{-1} \text{ iff } (a,b) \in R \vee (a, b) \in S$$
$$(b, a) \in (R \cup S)^{-1} \text{ iff } (b,a) \in R^{-1} \vee (b,a) \in S^{-1}$$
$$(b, a) \in (R \cup S)^{-1} \text{ iff } (b,a) \in R^{-1} \cup S^{-1}$$

\section*{Exercise 11}

Suppose $R$ is a relation from $A$ to $B$ and $S$ is a relation from $B$ to $C$. 
Prove that $S \circ R = \emptyset$ iff $Ran(R)$ and $Dom(S)$ are disjoint 

Let's choose some $b_0 \in B$ which $b_0 \in Ran(R)$. We can also choose some 
$b_1 \in B$ such that $b_1 \in Dom(S)$. However, $Ran(R)$ and $Dom(S)$ are 
disjoint so $b_0 \notin Dom(S)$ and $b_1 \notin Ran(R)$. Since in $S \circ R$ 
the $Dom(S)$ is the $Ran(R)$, then $S \circ R = \emptyset$.

\section*{Exercise 12}

Suppose $R$ is a relation from $A$ to $B$ and $S$ and $T$ are relations from 
$B$ to $C$.

\noindent (a) Prove that $(S \circ R) \setminus (T \circ R) \subseteq (S \setminus T) \circ R$

Suppose $(a,c) \in (S \circ R) \setminus (T \circ R)$. Then $(a,c) \in S \circ R$ 
and $(a,c) \notin T \circ R$. Since $(a,c) \in S \circ R$ we can choose some 
$b \in B$ such that $(a,b) \in R$ and $(b,c) \in S$. If $(b,c) \in T$ then 
$(a, c) \in T \circ R$ which is a contradiction. Therefore, $(b,c) \notin T$ 
so $(b,c) \in S \setminus T$. Since $(a,b) \in R$ and $(b,c) \in S \setminus T$,
$(a,c) \in (S \setminus T) \circ R$

\noindent (b) What's wrong with the following proof that 
$(S \setminus T) \circ R \subseteq (S \circ R) \setminus (T \circ R)$?

\textit{Proof}. Suppose $(a,c) \in (S \setminus T) \circ R$. Then we can choose 
some $b \in B$ such that $(a,b) \in R$ and $(b,c) \in S \setminus T$, so 
$(b,c) \in S$ and $(b,c) \notin T$. Since $(a,b) \in R$ and $(b,c) \in S$,
$(a,c) \in S \circ R$. Similarly, since $(a,b) \in R$ and $(b,c) \notin T$,
$(a,c) \notin T \circ R$. Therefore $(a,c) \in (S \circ R) \setminus (T \circ R)$.
Since $(a, c)$ was arbitrary, this shows that 
$(s \setminus T) \circ R \subseteq (S \circ R) \setminus (T \circ R)$ 

The mistake is in the sentence ``since $(a,b) \in R$ and $(b,c) \notin T$
$(a,c) \notin T \circ R$''. However, we can't be sure of this since there could 
be some $b \in B$ such that $(a,b) \in R$ and $(b, c) \in T$.

\noindent (c) Must it be true that 
$(S \setminus T) \circ R \subseteq (S \circ R) \setminus (T \circ R)$?

No, consider the counterexample $A = \{1\}$, $B = \{2,3\}$, $C = \{4\}$,
$R=\{(1,2), (1,3)\}$, $S=\{(2,4)\}$, $T=\{(3,4)\}$

\section*{Exercise 13}

Suppose $R$ and $S$ are relations from $A$ to $B$ and $T$ is a relation from $B$
to $C$. Must the following statements be true? Justify your answers

\noindent (a) If $R$ and $S$ are disjoint, then so are $R^{-1}$ and $S^{-1}$

This is true. Suppose $R$ and $S$ are disjoint and $R^{-1}$ and $S^{-1}$ are not.
Then we can choose from $(b,a) \in B \times A$ such that $(b,a) \in R^{-1}$
and $(b,a) \in S^{-1}$. But then $(a,b) \in R$ and $(a,b) \in S$ which contradicts 
the fact that $R$ and $S$ are disjoint.

\noindent (b) If $R$ and $S$ are disjoint, then so are $T \circ R$ and $T \circ S$

False. Consider this counterexample. $A = \{1\}, B=\{2,3\}, C = \{4\}, 
R = \{(1,2)\}, S=\{(1,3)\}, T=\{(2,4),(3,4)\}$

\noindent (c) If $T \circ R$ and $T \circ S$ are disjoint then so are $R$ and $S$.

False. Consider the counterexample $A=\{1\}, B=\{2,3\}, C=\{4\}, R=S=\{(1,2)\},
T=\{(3,4)\}$.

\section*{Exercise 14}

Suppose $R$ is a relation from $A$ to $B$, and $S$ and $T$ are relations from 
$B$ to $C$. Must the following statements be true? Justify your answers

\noindent (a) If $S \subseteq T$ then $S \circ R \subseteq T \circ R$

True. Since $S \subseteq T$ then $(b,c) \in S$ and $(b,c) \in T$. Therefore 
$(a,c) \in S \circ R$ and $(a,c) \in T \circ R$ so $S \circ R \subseteq T \circ R$.

\noindent (b) $(S \cap T) \circ R \subseteq (S \circ R) \cap (T \circ R)$

True. Suppose $(a,c) \in (S \cap T) \circ R$. There must be some 
$(b,c) \in (S \cap T)$. $(b,c) \in S$ and $(b,c) \in T$ by definition. Now there 
is also some $(a,b) \in R$ by definition. Therefore $(a,c) \in (S \circ R)$ and 
$(a,c) \in T \circ R$ so $(a,c) \in (S \circ R) \cap (T \circ R)$.

\noindent (c) $(S \cap T) \circ R = (S \circ R) \cap (T \circ R)$

Not necessarily true. We know $(S \cap T) \circ R \subseteq (S \circ R) \cap (T \circ R)$.
But, consider $A=\{1\}, B=\{2,3\}, C=\{4\}, R=\{(1,2), (1,3)\}, S=\{(2,4)\},
T=\{(3,4)\}$

\noindent (d) $(S \cup T) \circ R = (S \circ R) \cup (T \circ R)$

True

Suppose $(a,c) \in (S \cup T) \circ R$. Then we can choose some $b \in B$ such 
that $(a,b) \in R$ and $(b,c) \in S \cup T$ which means either $(b,c) \in S$
or $(b,c) \in T$.

Case 1: $(b,c) \in S$. Since $(a,b) \in R$ and $(b,c) \in S$ then 
$(a,c) \in S \circ R$. So $(a,c) \in (S \circ R) \cup (T \circ R)$

Case 2: $(b,c) \in T$. Since $(a,b) \in R$ and $(b,c) \in T$ then $(a,c) \in T \circ R$
so $(a,c) \in (S \circ R) \cup (T \circ R)$

Suppose now $(a,c) \in (S \circ R) \cup (T \circ R)$ then either 
$(a,c) \in S \circ R$ or $(a,c) \in T \circ R$. 

Case 1: $(a,c) \in S \circ R$. Then we can choose some $b \in B$ such that 
$(a,b) \in R$ and $(b,c) \in S$. Since $(b,c) \in S$ then $(b,c) \in S \cup T$. 
Since $(a,b) \in R$ and $(b,c) \in S \cup T$ then $(a,c) \in (S \cup T) \circ R$

Case 2: $(a,c) \in T \circ R$. Then we can choose some $b \in B$ such that 
$(a,b) \in R$ and $(b,c) \in T$. Since $(b,c) \in T$ then 
$(b,c) \in S \cup T$.

Since $(a,b) \in R$ and $(b,c) \in S \cup T$ then $(a,c) \in (S \cup T) \circ R$.
Therefore, $(a,c) \in (S \cup T) \circ R$.

\section*{Exercise 15}

Suppose $R$ is a relation from $A$ to $B$ and $S$ is a relation from $C$ to $D$.
Show that there is a set $E$ such that $R$ is a relation from $A$ to $E$ and $S$ 
is a relation from $E$ to $D$, and therefore the definition of $S \circ R$ in 
Definition 4.2.3 can be applied. Furthermore, the definition gives the same 
result no matter which such set $E$ is used.

Let $E = B \cup C$. Then $B \subseteq E$ and $C \subseteq E$, so 
$R \subseteq A \times B \subseteq A \times E$ and 
$S \subseteq C \times D \subseteq E \times D$. Now suppose $E_1$ and $E_2$ are 
two sets such that $R \subseteq A \times E_1$, $S \subseteq E_1 \times D$,
$R \subseteq A \times E_2$ and $S \subseteq E_2 \times D$. Let 

$$T_1 = \{(a,d) \in A \times D | \exists e \in E_1 ((a,e) \in R \wedge (e,d) \in S)\}$$
$$T_2 = \{(a,d) \in A \times D | \exists e \in E_2 ((a,e) \in R \wedge (e,d) \in S)\}$$

We need to prove that $T_1 = T_2$. Suppose $(a,d) \in T_1$. Then we can choose 
some $e \in E_1$ such that $(a,e) \in R$ and $(e,d) \in S$. Since3 $(a,e) \in R$
and $R \in A \times E_2$ then $(a,e) \in A \times E_2$ and therefore $e \in E_2$.
Since $e \in E_2$ then $(a,e) \in R$ and $(e,d) \in S$ then $(a,d) \in T_2$.
A similar argument shows that if $(a,d) \in T_2$ then $(a,d) \in T_1$ so $T_1 = T_2$.


\end{document}
