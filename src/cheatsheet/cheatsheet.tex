\documentclass[11pt]{article}
\usepackage{amssymb}
\usepackage{tipa}
\usepackage{amsmath}
\usepackage[scr]{rsfso}

\newcommand{\then}{\rightarrow} 
\newcommand{\bicond}{\leftrightarrow}
\newcommand{\powerset}[1]{\mathscr{P}(#1)}

\title{\textbf{How to Prove It} \\ {\Large\itshape Daniel J. Velleman} \\ {\Large\itshape Cheatsheet of Formulas}}

\author{\textbf{Nathaniel Curnick}}

\date{}

%----------------------------------------------------------------------------------------

\begin{document}

\maketitle

\section{Logical Formulas}

\subsection{De Morgan's Laws}

$$\neg (P \wedge Q) \equiv \neg P \vee \neg Q$$
$$\neg (P \vee Q) \equiv \neg P \wedge \neg Q$$

\subsection{Commutative Laws}

$$P \wedge Q \equiv Q \wedge P$$
$$P \vee Q \equiv Q \vee P$$

\subsection{Associative Laws}

$$P \wedge (Q \wedge R) \equiv  (P \wedge Q) \wedge P$$
$$P \vee (Q \vee R) \equiv  (P \vee Q) \vee P$$

\subsection{Idempotent Laws}

$$P \wedge P \equiv P$$
$$P \vee P \equiv P$$

\subsection{Distributive Laws}

$$P \wedge (Q \vee R) \equiv (P \wedge Q) \vee (P \wedge R)$$
$$P \vee (Q \wedge R) \equiv (P \vee Q) \wedge (P \vee R)$$

\subsection{Absorbtion Laws}

$$P \vee (P \wedge Q) \equiv P$$
$$P \wedge (P \vee Q) \equiv P$$

\subsection{Double Negation}

$$\neg \neg P \equiv P$$

\section{Conditional and Biconditional Laws}

$$P \then Q \equiv \neg P \vee Q$$
$$P \then Q \equiv \neg (P \wedge \neg Q)$$
$$P \bicond Q \equiv (P \then Q) \wedge (Q \then P)$$
$$P \bicond Q \equiv (P \then Q) \wedge (\neg P \then \neg Q)$$
$$P \bicond Q \equiv (\neg P \vee Q) \wedge (\neg Q \vee P)$$

\section{Set Laws}
$$A \subseteq B \equiv \forall x (x \in A \then x \in B)$$

\section{Quantifier Negation Laws}

$$\neg \exists P(x) \equiv \forall x \neg P(x)$$
$$\neg \forall x P(x) \equiv \exists x \neg P(x)$$
$$\neg \forall x (x \in A \then P(x)) \equiv \exists x \in A \neg P(x)$$
$$\forall x \in A P(x) \equiv \neg \exists x \in A \neg P(x)$$

\section{Other Quantifier Laws}

$$\exists! x P(x) \equiv \exists x (P(x) \wedge \neg \exists y (P(y) \wedge y \neq x))$$

\section{More Set Operations}

$$\powerset{A}$$
\end{document}