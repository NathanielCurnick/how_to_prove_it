\documentclass[11pt]{article}
\usepackage{amssymb}
\usepackage{tipa}
\usepackage{amsmath}
\usepackage[scr]{rsfso}


\newcommand{\then}{\rightarrow} 
\newcommand{\bicond}{\leftrightarrow}
\newcommand{\powerset}[1]{\mathscr{P}(#1)}
\newcommand{\family}{\mathcal{F}}

\title{\textbf{How to Prove It} \\ {\Large\itshape Daniel J. Velleman} \\ {\Large\itshape Chapter 1: Operations on Sets}}

\author{\textbf{Nathaniel Curnick} \\ \textit{Textbook Solutions}}

\date{}

%----------------------------------------------------------------------------------------

\begin{document}

\maketitle

\section*{Exercise 1}

Consider the following theorem. (This theorem was proven in the introduction)

\textbf{Theorem}. Suppose $n$ is an integer larger than 1 and $n$ is not prime.
Then $2^n - 1$ is not prime 

\noindent (a) Identify the hypotheses and conclusion of the theorem. Are the 
hypotheses true when n = 6? What does the theorem tell you in this instance? 
Is it right?

Hypotheses are:

\begin{itemize}
    \item Universe of discourse is $\mathbb{N}$;
    \item $n > 1$;
    \item $n$ is not prime.
\end{itemize}

Conclusion is $2^n - 1$ is not prime.

When $n = 6$ then $2^6 - 1 = 63$ which is not prime. This is consistent with the 
theorem, but does not prove it.

\noindent (b) What can you conclude from the theorem in the case $n = 15$? 
Check directly that this conclusion is correct

$2^15 - 1 = 32767$, and 32767 is not prime, so the theorem is again consistent

\noindent (c) What can you conclude from the theorem in the case $n=11$?

Nothing, since 11 is prime, which contradicts the hypotheses

\section*{Exercise 2}

Consider the following theorem. (The theorem is correct, but we will not ask 
you to prove it here)

\textbf{Theorem}. Suppose that $b^2 > 4ac$. Then the quadratic equation 
$ax^2 + bx + c = 0$ has exactly two real solutions.

\noindent (a) Identify the hypotheses and conclusions of the theorem

Hypotheses are $b^2 > 4ac$.

Conslusions are $x$ has two real solutions in the equation $ax^2 + bx + c = 0$

\noindent (b) To give an instance of the theorem, you must specify values for 
$a$, $b$, and $c$, but not $x$. Why?

$a$, $b$, and $c$ are free values in the theorem.

\noindent (c) What can you conclude from the theorem in the case $a=2, b=-5, 
c=3$? Check directly that the conclusion is correct

$-5^2 = 25$ and $4 \times 2 \times 3 = 24$, thus our hypotheses are satisfied.

The solutions to $2x^2 -5x + 3 = 0$ are 1.5 and 1, which are both real numbers.
This is consistent with the theorem.

\noindent (d) What can you conclude from the theorem in the case $a=2, b=4, c=3$?

$4^2 = 16$ and $4 \times 2 \times 3 = 24$, so $b^2 < 4ac$. This contradicts the 
hypotheses, so we can say nothing about the theorem in this case.

\section*{Exercise 3}

Consider the following incorrect theorem

\textbf{Incorrect Theorem}. Suppose $n$ is a natural number larger than 2, and 
$n$ is not a prime number. Then $2n + 13$ is not a prime number.

What are the hypotheses and conclusion of this theorem? Show that the theorem is 
incorrect by finding a counterexample

Hypotheses are 

\begin{itemize}
    \item Universe of discourse is $\mathbb{N}$;
    \item $n > 2$,
    \item $n$ is not prime.
\end{itemize}

Conclusion is $2n + 13$ is not prime.

If $n=8$, then $2 \times 8 + 13 = 29$, which is prime. Thus, the theorem is 
disproven.

\section*{Exercise 4}

Complete the following alternative proof of the theorem in Example 3.1.2

\textit{Proof}. Suppose $0 < a < b$. Then $b - a > 0$

[Fill in proof of $b^2 - a^2 > 0$ here]

Since $b^2 - a^2 > 0$, it follows that $a^2 < b^2$. Therefore, if $0 < a < b$
then $a^2 < b^2$.

\vspace{5pt}
\hrule
\vspace{5pt}

Suppose $0 < a < b$, then $b - a > 0$. Multiply both sides by $(b + a)$

$$(b + a)(b - a) > 0$$

$$b^2 - a^2 > 0$$

\section*{Exercise 5}

Suppose $a$ and $b$ are real numbers. Prove that if $a < b < 0$ then $a^2 > b^2$

Since $a < b$, if we multiply by $a$ then $a^2 > ab$. Equally, $ab > b^2$. 
So, $a^2 > ab > b^2$, so $a^2 > b^2$ as required

\section*{Exercise 6}

Suppose $a$ and $b$ are real numbers. Prove that if $0 < a < b$ then $1/b < 1/a$

$$0 < a < b$$

$$0 < \frac{a}{ab} < \frac{b}{ab}$$

$$0 < \frac{1}{b} < \frac{1}{a}$$

\section*{Exercise 7}

Suppose that $a$ is a real number. Prove that if $a^3 > a$ then $a^5 > a$.

So, $a^3 - a > 0$. Multiply by $a^2 + 1$ (which is a positive number)

$$(a^3 - a)(a^2 + 1) > 0$$

$$a^5 - a > 0$$

$$a^5 > a$$

\section*{Exericse 8}

Suppose $A \setminus B \subseteq C \cap D$ and $x \in A$. Prove that if 
$x \notin D$ then $x \in B$.

We will prove the contrapositive by assuming that $x \notin B$ and proving 
$x \in D$.

Since $x \in A$ and $x \notin B$ then $x \in A \setminus B$. Since 
$A \setminus B \subseteq C \cap D$ then $x \in C \cap D$. Thus $x \in D$.

Therefore, if $x \notin D$ then $x \in B$.


\section*{Exercise 9}

Suppose $A \cap B \subseteq C \setminus D$. Prove that if $x \in A$, then if 
$x \in D$, then $x \notin B$

We will assume that $x \in B$, and prove that $x \notin D$.

Since $x \in A$ and $x \in B$, then $x \in A \cap B$. Since 
$A \cap B \subseteq C \setminus D$, then $x \in C \setminus D$. 
However, if $x \in D$, then $x \notin C \setminus D$. 
Therefore $x \notin D$.

Prooven by contrapositive, so if $x \in D$ then $x \notin B$.

\section*{Exercise 10}

Suppose $a$ and $b$ are real numbers. Prove that if $a < b$ then $(a + b)/2<b$

$$\frac{a + b}{2} < b$$

$$a + b < 2b$$

$$a < b$$

\section*{Exercise 11}

Suppose $x$ is a real number and $x \neq 0$. Prove that if 
$(\sqrt[3]{x} + 5)/(x^2 + 6) = 1/x$ then $x \neq 8$.

Let's assume $x = 8$

$$\frac{(\sqrt[3]{x} + 5)}{(x^2 + 6)} = \frac{1}{10}$$

This is clearly not $1/8$.

\section*{Exercise 12}

Suppose $a, b, c$ and $d$ are real numbers, $0 < a < b$, and $d > 0$. Prove 
that if $ac \geq bd$ then $c > d$.

We will assume $c < d$ and prove $ac < bd$. Since $a < b$ and $c < d$, then 
$ac < db$. We have proven the contrapositive, so if $ac \geq bd$ then $c > d$.

\section*{Exercise 13}

Suppose $x$ and $y$ are real numbers. Prove that if $x^2 + y = -3$ and 
$2x - y = 2$ then $x = -1$.

We will prove the contrapositive by assuming that $y > 1$ and proving that 
$x < 1$. 

$$3x + 2y \leq 5$$

$$3x \leq 5 - 2y$$

Since $y > 1$ then the largest the RHS can be is 

$$3x < 3$$

Thus,

$$x < 1$$

\section*{Exercise 14}

Suppose that $x$ and $y$ are real numbers. Prove that if $x^2 + y = -3$
and $2x - y = 2$ then $x = -1$.

Solve the simultaneous equations by adding theorem

$$x^2 + 2x + 1 = 0$$

Factorise to 

$$(x + 1)(x + 1) = 0$$

Thereofre, $x = -1$

\section*{Exercise 15}

Prove the first theorem in Example 3.1.1

Suppose $x > 3$ and $y < 2$. Then $x^2 - 2y > 5$.

This can become 

$$x^2 > 5 + 2y$$

$y < 2$ so the largest the RHS can be is $x^2 > 9$, so $x > 3$.

\section*{Exercise 16}

Consider the following theorem

\textbf{Theorem}. Suppose $x$ is a real number and $x \neq 4$. If 
$(2x - 5)/(x - 4) = 3$ then $x = 7$.

\noindent (a) What's wrong with the following proof of the theorem?

\textit{Proof}. Suppose $x=7$. Then $(2x - 5)/(x-4) = (2(7) - 5)/(7-4)=9/3=3$.
Therefore if $(2x-5)/(x-4)=3$ then $x = 7$.

The reasoning is backwards. We want to prove $(2x - 5)/(x - 4) = 3 \then x = 7$.
But this proves that if $x = 7 \then (2x - 5)/(x - 4) = 3$.

\noindent (b) Give a correct proof of the theorem 

Suppose $(2x - 5).(x - 4) = 3$

$$2x - 5 = 3x - 12$$
$$3x - 2x = -5 + 12$$
$$x = 7$$

\section*{Exercise 17}

Consider the following incorrect theorem:

\textbf{Incorrect Theorem}. Suppose that $x$ and $y$ are real numbers and
$x \neq 3$. If $x^2 y = 9y$ then $y = 0$

\noindent (a) What's wrong with the following proof of the theorem?

\textit{Proof}. Suppose that $x^2 y = 9y$. Then $(x^2 - 9)y = 0$. Since 
$x \neq 3, x^2 \neq 9$ so $x^2 - 9 = 0$. Therefore we can divide both sides of 
the equation by $(x^2 - 9)y = 0$ by $x^2 - 9$, which leads to the conclusion 
that $y = 0$. Thus, if $x^2 y = 9y$ then $y = 0$.

The mistake is in assuming $x^2 \neq 9$, since $-3^2 = 9$.

\noindent (b) Show that the theorem is incorrect by finding a counterexample

$x = -3, y = 1$ is a suitable counterexample

\end{document}