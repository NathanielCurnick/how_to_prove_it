\documentclass[11pt]{article}
\usepackage{amssymb}
\usepackage{tipa}
\usepackage{amsmath}
\usepackage[scr]{rsfso}


\newcommand{\then}{\rightarrow} 
\newcommand{\bicond}{\leftrightarrow}
\newcommand{\powerset}[1]{\mathscr{P}(#1)}
\newcommand{\family}[1]{\mathcal{#1}}

\title{\textbf{How to Prove It} \\ {\Large\itshape Daniel J. Velleman} \\ {\Large\itshape Chapter 3.5: Proofs Involving Disjunctions}}

\author{\textbf{Nathaniel Curnick} \\ \textit{Textbook Solutions}}

\date{}

%----------------------------------------------------------------------------------------

\begin{document}

\maketitle

\section*{Exercise 1}

Prove that for every real number $x$ there is a unique real number $y$ such that 
$x^y = x - y$

Let $x$ be an arbitrary real number. Let $y = x / (x^2 + 1)$ then 

$$x - y =
x - \frac{x}{x^2 + 1} =
\frac{x^3 + x}{x^2 + 1} - \frac{x}{x^2 + 1} =
\frac{x^3}{x^2 + 1} =
x^2 \frac{x}{x^2 + 1} =
x^2 y$$

To see that $y$ is unique, suppose that $x^2 z = x - z$. Then $z(x^2 + 1) = x$,
and since $x^2 + 1 \neq 0$, we can divide both sides by $x^2 + 1$ to show that 
$z = x / (x^2 + 1) = y$

\section*{Exercise 2}

Prove that there is a unique real number $x$ such that for every real number 
$y$, $xy + x - 4 = 4y$

Let $x = 4$. Let $y$ be an arbitrary real number. Then 
$xy + x - 4 = 4y + 4 - 4 = 4y$

To prove that $x$ is unique, suppose $z$ is a real number such that for every 
real number $y$, $zy + z - 4 = 4y$. So setting $y = 0$ we get $z - 4 = 0$,
so $z = 4 = x$.

\section*{Exercise 3}

Prove that for every real number $x$, if $x \neq 0$ and $x \neq 1$ then there is 
a unique real number $y$ such that $y/x = y - x$ 

Let $x$ be an arbitrary real number, and assume that $x \neq 0$ and $x \neq 1$.
Let $y = x^2 / (x-1)$ (which we can since we assumed $x \neq 1$).

$$y - x = 
\frac{x^2}{x - 1} - x =
\frac{x^2}{x-1} - \frac{x^2 - x}{x - 1} =
\frac{x}{x - 1} =
\frac{\frac{x^2}{x-1}}{x} =
\frac{y}{x}$$

To show that $y$ is unique, suppose that $z$ is a real number such that 
$z/x = z - x$. Multiplying both sides by $x$, we conclude that 
$z = zx - x^2$, so $x^2 = z(x-1)$ and therefore $z = x^2/(x-1) = y$.

\section*{Exercise 4}

Prove that for every real number $x$, if $x \neq 0$ then there is a unique real 
number $y$ such that for every real number $z$, $zy = z/x$.

let $y = \frac{1}{x}$ (which is defined since we suppose $x \neq 0$).
Then $zy = z(1/x) = z/x$.

We can see $y$ is unique by supposing $w$ which has the property
$\forall z \in \mathbb{R} (zy = z/x)$, then taking $z = 1$ we have $w = 1/x$
so $w = y$.

\section*{Exercise 5}

Recall that if $\family{F}$ is a family of sets, then 
$\bigcup \family{F} = \{ x | \exists A (A \in \family{F} \wedge x \in A) \}$.
Suppose we define a new set $\bigcup! \family{F}$ by the formula 
$\bigcup! \family{F} = \{ x | \exists! A (A \in \family{F} \wedge x \in A) \}$

\noindent (a) Prove that for any family of sets, $\family{F}$, 
$\bigcup! \family{F} \subseteq \bigcup \family{F}$.

Suppose $A \in \family{F}$ and suppose $x \in A$. Then $x \in \bigcup! \family{F}$.
However, all $x \in \bigcup! \family{F}$ came from an $A$ which is also in 
$\family{F}$. In other words, all $x \in \bigcup! \family{F}$ is also 
$x \in \bigcup \family{F}$. 
Thus, $\bigcup! \family{F} \subseteq \bigcup \family{F}$.

\noindent (b) A family of sets $\family{F}$ is said to be 
\textit{pairwise disjoint} if every pair of distinct elements of $\family{F}$ are 
disjoint; that is, 
$\forall A \in \family{F} \forall B \in \family{F} (A \neq B \then A \cap B = \emptyset)$.
Prove that for any family of sets $\family{F}$, 
$\bigcup! \family{F} = \bigcup \family{F}$ iff $\family{F}$ is pairwise disjoint.

In other words, we need to prove 
$\bigcup! \family{F} = \bigcup \family{F} \bicond 
\forall A \in \family{F} \forall B \in \family{F} (A \neq B \then A \cap B = \emptyset)$

($\rightarrow$) Suppose $\bigcup! \family{F} = \bigcup \family{F}$. Suppose 
$A \cap B \neq \emptyset$. Then $A \in \family{F}, x \in A, B \in \family{F},
x \in B$. Since $\bigcup! \family{F} = \bigcup \family{F}$, then there is only 
one set in $\family{F}$ which contains $x$, but we have just demonstrated that 
$A$ and $B$ contains $x$. Thus, $A \cap B = \emptyset$, by contradiction.

($\leftarrow$) Suppose $A \in \family{F}$ and $B \in \family{F}$. Suppose 
$A \cap B = \emptyset$. Thus, we choose some $x \in A$ but $x \notin B$. 
Since the only element of $\family{F}$ that contains $x$ is $A$, then 
$\bigcup \family{F} = \bigcup! \family{F}$.

\section*{Exercise 6}

Let $U$ be any set

\noindent (a) Prove that there is a unique $A \in \powerset{U}$ such that for 
every $B \in \powerset{U}, A \cup B = B$

Suppose $A = \emptyset, A \in \powerset{U}$. Then, for every $B \in \powerset{U},
B \cup \emptyset = B$. To show that $A$ is unique suppose 
$A^\prime \in \powerset{U}$ and for all $B \in \powerset{U}, A^\prime \cup B = B$.
Then, taking $B = \emptyset$ we can conclude that 
$A^\prime \cup \emptyset = \emptyset$. But, clearly, 
$A^\prime \cup \emptyset = A^\prime$, so $A^\prime = \emptyset = A$.

\noindent (b) Prove that there is a unique $A \in \powerset{U}$ such that for 
every $B \in \powerset{U}, A \cup B = A$

Let $A = U$. Let $B \in \powerset{U}$ be arbitrary. Then $B \subseteq U = A$ so 
$A \cap B = B$. To see that $A$ is unique, suppose $A^\prime \in \powerset{U}$
and $\forall B \in \powerset{U} (A^\prime \cup B = B)$. In particular,
since $U \in \powerset{U}, A^\prime \cup U = U$. But $A^\prime \in \powerset{U}$
so $A^\prime \subseteq U$, and $A^\prime \cup U = U$. So $A^\prime = U = A$.

\section*{Exercise 7}

Let $U$ be any set 

\noindent (a) Prove that there is a unique $A \in \powerset{U}$ such that for 
every $B \in \powerset{U}, A \cap B = B$

Let $A = U$, also let $B \in \powerset{U}$ be arbitrary. Then $B \subseteq U = A$,
so $A \cap B = B$. To show that $A$ is unique suppose $A^\prime \in \powerset{U}$
and for all $B \in \powerset{U}$, $A \cap B = B$. Taking $B = U$ we can conclude 
that $A^\prime \cap B = U$, but this is $A$ so $A^\prime = A$.

\noindent (b) Prove that there is a unique $A \in \powerset{U}$ such that for 
every $B \in \powerset{U}, A \cap B = A$

Let $A = \emptyset$. Let $B \in \powerset{U}$ be arbitrary then 
$\emptyset \cap B = \emptyset$. To prove $A$ is unique suppose 
$A^\prime \in \powerset{U}$ and 
$\forall B \in \powerset{U} (A^\prime \cap B = A^\prime)$. Since 
$\emptyset \in \powerset{U}, A^\prime \cap \emptyset = A^\prime$. But 
$A^\prime \cap \emptyset = \emptyset$, so $A^\prime = \emptyset = A$.

\section*{Exercise 8}

Let $U$ be any set 

\noindent (a) Prove that for every $A \in \powerset{U}$ such that for every 
$C \in \powerset{U}, C \setminus A = C \cap B$

Suppose $A \in \powerset{U}$, let $B = U \setminus A$. Let $C \in \powerset{U}$
be arbitrary. Then $C \subseteq U$. Now to prove $C \setminus A = C \cap B$
suppose $x \in C \setminus A$. Then $x \in C$ and $x \notin A$. Since 
$x \in C$, $C \subseteq U$ then $x \in U$. Since $x \in U$ and $x \notin A$ 
then $x \in U \setminus A = B$. Therefore $x \in C \cap B$. 

Suppose now $x \in C \cap B$. Then $x \in C$ and $x \in B = U \setminus A$ so 
$x \notin A$. Therefore $x \in C \setminus A$, so $C \setminus A = C \cap B$

Now to prove $B$ is unique suppose $B^\prime \in \powerset{U}$ and 
$\forall C \in \powerset{U} (C \setminus A = C \cap B^\prime)$, 
$U \setminus A = U \cap B$. But $B^\prime \subseteq U$ so 
$U \cap B^\prime = B^\prime$. Therefore $B^\prime = U \setminus A = B$.

\noindent (b) Prove that for every $A \in \powerset{U}$ such that for every 
$C \in \powerset{U}, C \cap A = C \setminus B$

Suppose $A \in \powerset{U}$. Let $B = U \setminus A$. Let $C \in \powerset{U}$
be arbitrary. Then $C \subseteq U$. We claim now that $C \cap A = C \setminus B$.
To prove this, suppose $x \in C \cap A$. Then, $x \in C$ and $x \in A$. Since 
$x \in A$, $x \notin U \setminus A = B$, so $x \in C \setminus B$. Now suppose 
$x \in C \setminus B$. Then $x \in C$ and $x \notin B = U \setminus A$, so 
either $x \notin U$ or $x \in A$. But, since $x \in C$ and $C \subseteq U$,
so $x \in U$. Therefore, $x \in A$ so $x \in C \cap A$.

To prove that $B$ is unique suppose $B^\prime \in \powerset{U}$ and 
$\forall C \in \powerset{U} (C \cap A = C \setminus B^\prime)$. Then 
$U \cap A = U \setminus B^\prime$. Since $A \subseteq U$, $U \cap A = A$ so 
$U \setminus B^\prime = A$. We claim now $B^\prime = U \setminus A$. To prove 
this suppose $x \in B^\prime$. Then $x \notin U \setminus B^\prime = A$. Also,
since $x \in B^\prime$ and $B^\prime \subseteq U$ then $x \in U$. Thus, 
$x \in U \setminus A$. Suppose $x \in U \setminus A$ then 
$x \in U$ and $x \notin A = U \setminus B^\prime$, so either $x \notin U$ or 
$x \in B^\prime$. Since $x \in U$, $x \in B^\prime$. Thus, 
$B^\prime = U \setminus A = B$.

\section*{Exercise 9}

Recall that you showed in exercise 14 of section 1.4 that symmetric difference is 
associative; in other words, for all sets $A$, $B$ and $C$ 
$A \triangle (B \triangle C) = (A \triangle B) \triangle C$. You mat also find it 
useful in this problem to note that symmetric difference is clearly 
commutative; in other words, for all sets $A$ and $B$, 
$A \triangle B = B \triangle A$.

\noindent (a) Prove that there is a unique identity element for symmetric 
difference. In other words, there is a unique set $X$ such that for every 
set $A$, $A \triangle X = A$.

$x = \emptyset$, then 
$A \triangle X = (A \setminus \emptyset) \cup (\emptyset \setminus A) = A$.

To see that this is unique suppose $X_1$ and $X_2$ such that 
$\forall A (A \triangle X_1 = A)$ and $\forall A (A \triangle X_2 = A)$, but 
applying these statements to each other yields 
$X_2 \triangle X_1 = X_2$ and $X_1 \triangle X_2 = X_1$. This is obvious
nonsense unless $X_1 = X_2$.

\noindent (b) Prove that every set has a unique inverse for the operation of 
symmetric difference. In other fors, for every set $A$ there is a unique 
set $B$ such that $A \triangle B = X$, where $X$ is the identity element from 
part (a).

Suppose $A = B$ then 
$A \triangle B = (A \setminus B) \cup (B \setminus A) = \emptyset = X$. Now 
suppose $B_1$ and $B_2$ fulfil this condition 

$$B_1 = 
B_1 \triangle \emptyset = 
B_1 \triangle (A \triangle B_2) =
(B_1 \triangle A) \triangle B_2 =
\emptyset \triangle B_2 =
B_2$$

\noindent (c) Prove that for any sets $A$ and $B$ there is a unique set $C$ 
such that $A \triangle C = B$

Suppose $C = A \triangle B$ then 

$$A \triangle C =
A \triangle (A \triangle B) = 
(A \triangle A) \triangle B =
\emptyset \triangle B =
B$$

Now suppose $C_1$ and $C_2$ are such that $A \triangle C_1 = A \triangle C_2 = B$ then 

$$C_1 =
\emptyset \triangle C_1 =
(A \triangle A) \triangle C_1 = 
A \triangle (A \triangle C_1) =
A \triangle (A \triangle C_2) = 
(A \triangle A) \triangle C_2 =
\emptyset \triangle C_2 =
C_2$$

\noindent (d) Prove that for every set $A$ there is a unique set $B \subseteq A$
such that for every set $C \subseteq A, B \triangle C = A \setminus C$.

Let $B = A$, suppose $C \subseteq A$ then

$$B \triangle C = 
(A \setminus C) \cup (C \setminus A) =
(A \setminus C) \cup \emptyset = 
A \setminus C$$

\section*{Exercise 10}

Suppose $A$ is a set, and for every family of sets $\family{F}$, if 
$\bigcup \family{F} = A$ then $A \in \family{F}$. Prove that $A$ has exactly one 
element.

I think there is an error in the book, and that this is not actually true.
Instead, I present a counterexample showing that it is a false claim. 
Suppose $A = \{1,2\}$, which obviously has more than one element. Suppose 
$\family{F} = \{ \{1,2\}, \{2\} \}$, and obviously $A \in \family{F}$.
$\bigcup \family{F} = \{1,2\} = A$, despite $A$ having more than one element.

\section*{Exercise 11}

Suppose $\family{F}$ is a family of sets that has the property that for every 
$\family{G} \subseteq \family{F}, \bigcup \family{G} \in \family{F}$. Prove that 
there is a unique set $A$ such that $A \in \family{F}$ and 
$\forall B \in \family{F} (B \subseteq A)$.

Since $\family{F} \subseteq \family{F}, \bigcup \family{F} \in \family{F}$. 
Let $A = \bigcup \family{F}$ and suppose $B \in \family{F}$ thus 
$B \subseteq \bigcup \family{F} = A$. Suppose now that 
$A_1 \in \family{F}, A_2 \in \family{F}, \forall B \in \family{F} (B \subseteq A_1)$
and $\forall B \in \family{F} (B \subseteq A_2)$. Applying this last fact with 
$B = A_1$ we can conclude that $A_1 \subseteq A_2$ but by symmetry $A_2 \subseteq A_1$
or in other words $A_1 = A_2$.

\section*{Exercise 12}

\noindent (a) Suppose $P(x)$ is a statement with a free variable $x$. Find a formula using 
the logical symbols we have studied that means ``there are exactly two values 
of $x$ for which $P(x)$ is true.''

$$\exists x \exists y (P(x) \wedge P(y) \wedge x \neq y \wedge \forall z (P(x) \then (z = x \vee z = y)))$$

\noindent (b) Based on your answer to part (a), design a proof strategy for proving
a statement for the form ``there are exactly two values 
of $x$ for which $P(x)$ is true.''

Find two objects $a$ and $b$ such that $P(a)$ and $P(b)$ are both true,
$a \neq b$ and 
$\forall z (P(z) \then (z = a \vee z = b))$

\noindent (c) Prove that there are exactly two solutions to the equation 
$x^3 = x^2$

Obviously $x = 1$ and $x = 0$ are both solutions. Suppose we have some other 
solution $z$. Then we could divide both sides by $z^2$, so $z^3 = z^2$, then 
$z = 1$ but this is the same as one of our original solutions.

\section*{Exercise 13}

\noindent (a) Prove that there is a unique real number $c$ such that there is a 
unique real number $x$ such that $x^2 + 3x + c = 0$

Let $c = 9/4$ so $x^2 + 3x + 9/4 = 0$. Consider the quadratic formula 

$$\frac{-3 \pm \sqrt{9-9}}{2} = \frac{-3}{2}$$

Consider if $c > 9/4$ (which I call $d$) then 

$$\frac{-3 \pm \sqrt{9-d}}{2}$$

which clearly produces additional imaginary solutions.

Consider now $c < 9/4$ (which I call $e$) then 

$$\frac{-3 \pm \sqrt{9-e}}{2}$$

which produces additional real solutions.

\noindent (b)

Show that it is \textit{not} the case that there is a unique real number $x$ 
such that there is a unique real number $c$ such that $x^2 + 3x + c = 0$.

For every $x$ there is always $x = x^2 -3x$ which is a unique solution.


\end{document}
