\documentclass[11pt]{article}
\usepackage{amssymb}
\usepackage{tipa}
\usepackage{amsmath}
\usepackage[scr]{rsfso}


\newcommand{\then}{\rightarrow} 
\newcommand{\bicond}{\leftrightarrow}
\newcommand{\powerset}[1]{\mathscr{P}(#1)}
\newcommand{\family}[1]{\mathcal{#1}}

\title{\textbf{How to Prove It} \\ {\Large\itshape Daniel J. Velleman} \\ {\Large\itshape Chapter 3.5: Proofs Involving Disjunctions}}

\author{\textbf{Nathaniel Curnick} \\ \textit{Textbook Solutions}}

\date{}

%----------------------------------------------------------------------------------------

\begin{document}

\maketitle

\section*{Exercise 1}

Suppose $\family{F}$ is a family of sets. Prove that there is a unique set 
$A$ that has the following two properties.

\noindent (a) $\family{F} \subseteq \powerset{A}$

\noindent (b) $\forall B (\family{F} \subseteq \powerset{B} \then A \subseteq B)$

Suppose $A = \bigcup \family{F}$. BY definition, 
$\family{F} \subseteq \bigcup \powerset{A}$, since $\powerset{A}$ covers all 
possible subsets of $A$, and since $\family{F}$ only contains sets that are 
subsets of $A$ then $\family{F} \subseteq \powerset{A}$.

Suppose now that $B$ is an arbitrary sets and that 
$\family{F} \subseteq \powerset{B}$. Since $\family{F} \subseteq \powerset{B}$,
then $\bigcup \family{F} \subseteq B$. But since $A = \bigcup \family{F}$,
$A \subseteq B$.

Suppose arbitrary sets $A_1$ and $A_2$ such that 
$\family{F} \subseteq \powerset{A_1}$,
$\forall B (\family{F} \subseteq \powerset{B} \then A_1 \subseteq B)$,
$\family{F} \subseteq \powerset{A_2}$,
$\forall B (\family{F} \subseteq \powerset{B} \then A_2 \subseteq B)$.
We can take $A_1$ as B to show 
$\family{F} \subseteq \powerset{A_1} \then A_2 \subseteq A_1$ and by symmetry
$\family{F} \subseteq \powerset{A_2} \then A_1 \subseteq A_2$.
In other words $A_1 \subseteq A_2$ and $A_2 \subseteq A_1$ thus 
$A_1 = A_2$.

\section*{Exercise 2}

Prove that there is a unique positive real number $m$ that has the following 
two properties

\noindent (a) For every positive real number $x$, $\frac{x}{x+1} < m$

\noindent (b) If $y$ is any positive real number with the property that for every 
positive real number $x$, $\frac{x}{x+1} < y$, then $m \leq y$.

Let $m=1$. Since $x$ is a positive real number, then $x < x + 1$, so 
$x / (x+1) < 1$. To show $y$, we use the same argument. Since $x/(x_1) < 1$ then 
$y$ is at minimum 1, so $y = m$, or $y$ is larger than 1 (since 
$y > x / (x+1)$), so $y>m$, in other words $y \geq m$.

To show $m$ is unique suppose $m_1$ and $m_2$ both satisfy the properties.
Thus $x/(x+1) < m_1$ and $x/(x+1) < m_2$. Similarly $m_1 \leq y$ and 
$m_2 \leq y$. This means both $m_1 \leq m_2$ and $m_2 \leq m_1$, so 
$m_1 = m_2$.

\section*{Exercise 3}

Suppose $A$ and $B$ are sets. What can you prove about 
$\powerset{A \setminus B} \setminus (\powerset{A} \setminus \powerset{B})$?

$\powerset{A \setminus B} \setminus (\powerset{A} \setminus \powerset{B}) = \{ \emptyset \}$

Since $\emptyset$ is a subset of every set, $\emptyset \in \powerset{B}$, so 
$\emptyset \notin \powerset{A} \setminus \powerset{B}$, so 
$\{ \emptyset \} \subseteq \powerset{A \setminus B} \setminus (\powerset{A} \setminus \powerset{B})$.

Now suppose $X \in \powerset{A \setminus B} \setminus (\powerset{A} \setminus \powerset{B})$.
Thus $X \in \powerset{A \setminus B}$, so $X \subseteq A \setminus B$ and 
$X \notin \powerset{A} \setminus \powerset{B}$. Clearly 
$A \setminus B \subseteq A$, so since $X \subseteq A \setminus B$, 
$X \subseteq A$ and therefore $X \in \powerset{A}$. Since 
$X \notin \powerset{A} \setminus \powerset{B}$, it follows that 
$X \in \powerset{B}$, so $X \subseteq B$. To show that $X = \emptyset$
suppose $x \in X$. Since $X \subseteq A \setminus B$, $x \notin B$ and since 
$X \subseteq B$, $x \in B$, which is obviously a contradiction. Therefore $X$ 
can not have any elements, so $X = \emptyset$.

\section*{Exercise 4}

Suppose that $A$, $B$ and $C$ are sets. Prove that the following statements are 
equivalent.

\noindent (a) $(A \triangle C) \cap (B \triangle C) = \emptyset$

\noindent (b) $A \cap B \subseteq C \subseteq (A \cup B)$

\noindent (c) $A \triangle C \subseteq A \triangle B$

First we will show $(A \triangle C) \cap (B \triangle C) = \emptyset$ is 
equivalent to $A \cap B \subseteq C \subseteq (A \cup B)$

Suppose $x \in A \cap B$ so $x \in A$ and $x \in B$. If $x \notin C$ then 
$x \in A \setminus C$ so $x \in A \triangle C$ and by symmetry $x \in B \triangle C$,
which is a contradiction so $x \in C$.

No suppose $x \in C$ and $x \notin A \cup B$. Then $x \notin A$ and $x \notin B$.
Thus $x \in A \triangle C$ and $x \in B \triangle C$, but this is a contradiction 
so $x \in A \cup B$. So $C \subseteq A \cup B$.

Now we show that $A \cap B \subseteq C \subseteq (A \cup B)$ is equivalent to 
$A \triangle C \subseteq A \triangle B$

Suppose $x \in A \triangle C$. Then either $x \in C \setminus A$ or 
$x \in A \setminus C$.

Suppose $x \in A \setminus C$. Then $x \in A$ and $x \notin C$. If $x \in B$
then $x \in A \cap B$ so $x \in C$, which is a contradiction therefore $x \in B$.
Since $x \in A$ and $x \notin B$, $x \in A \triangle B$.

If $x \in C \setminus A$ then $x \notin A$ and $x \in C$. Since $x \in C$ then 
$x \in A \cup B$, but since $x \notin A$ then $x \in B$. Since $x \in B$ and
$x \notin A$, $x \in A \triangle B$. Thus $x \in A \triangle B$. Since $x$ was 
an arbitrary element of $A \triangle C$, this shows that 
$A \triangle C \subseteq A \triangle B$.

Now we show that $A \triangle C \subseteq A \triangle B$ is equivalent to 
$(A \triangle C) \cap (B \triangle C) = \emptyset$

Suppose $x \in A \triangle C$, so either $x \in A$ and $x \notin C$ or $x \in C$
or $x \notin A$.

Consider case $x \in A$ and $x \notin C$. This means that $x \in A \triangle C$,
and since $A \triangle C \subseteq A \triangle B$ and $x \in A$ then $x \notin B$.
Since $x \notin C$ and $x \notin B$ then $x \in B \triangle C$. Therefore 
$x \notin A \triangle C \cap B \triangle C$.

Consider case $x \in C$ and $x \notin A$. Since 
$A \triangle C \subseteq A \triangle B$, and $x \notin A$ then $x \in B$.
Since $x \in B$ and $x \in C$ then $x \notin B \triangle C$. Therefore 
$x \notin (A \triangle C)  \cap (B \triangle C)$.

Since we have shown (a) to be the same as (b), (b) to be the same as (c) and 
(c) to be the same as (a) then all three statements are equivalent.

\section*{Exercise 5}

Suppose $\{A_i | i \in I \}$ if a family of sets. Prove that if 
$\powerset{\bigcup_{i \in I} A_i} \subseteq \bigcup_{i \in I} \powerset{A_i}$ 
then there is some $i \in I$ such that $\forall j \in I (A_j \subseteq A_i)$

Suppose 
$\powerset{\bigcup_{i \in I} A_i} \subseteq \bigcup_{i \in I} \powerset{A_i}$.
Obviously $\bigcup_{i \in I} A_i \in \powerset{\bigcup_{i \in I} A_i}$ and 
therefore $\bigcup_{i \in I} A_i \in \bigcup_{i \in I} \powerset{A_i}$. We can 
choose some $i \in I$ such that $\bigcup_{i \in I} A_i \subseteq A_i$. Let 
$j \in I$ be arbitrary. Since $A_j \subseteq \bigcup_{i \in I} A_i$ then 
$A_j \subseteq A_i$.

\section*{Exercise 6}

Suppose $\family{F}$ is a nonempty family of sets. Let $I = \bigcup \family{F}$
and $J = \bigcap \family{F}$. Suppose also that $J \neq \emptyset$ and notice 
that it follows that for every $x \in \family{F}, X \neq \emptyset$ and also 
that $I \neq \emptyset$. Finally, suppose that $\{ A_i | i \in I \}$ is an 
indexed family of sets.

\noindent (a) Prove that $\bigcup_{i \in I} A_i = \bigcup_{x \in \family{F}} (\bigcup_{i \in X} A_i)$

Suppose $x \in \bigcup_{i \in I} A_i$. Then we can choose some $i_0 \in I$ such 
that $x \in A_{i_0}$. Since $i_0 \in I$ and $I = \bigcup \family{F}$ we can choose 
some $X_0 \in \family{F}$ such that $i_0 \in X_0$. Since $x \in A_{i_0}$ and 
$i_0 \in X_0$, $x \in \bigcup_{i \in X_0} A_i$ and since $X_0 \in \family{F}$
it follows that $x \in \bigcup_{X \in \family{F}} (\bigcup_{i \in X} A_i)$

Now suppose $x \in \bigcup_{X \in \family{F}} (\bigcup_{i \in I} A_i)$. Then 
we choosen some $X_0 \in \family{F}$ such that $x \in \bigcup_{i \in X_0} A_i$
which implies we can choose $i_0 \in X_0$ such that $x \in A_{i_0}$. Since 
$i_0 \in X_0$ and $X_0 \in \family{F}$, $i_0 \in \bigcup \family{F}$. Since 
$x \in A_{i_0}$ it follows that $x \in \bigcup_{i \in I} A_i$.

\noindent (b) Prove that $\bigcap_{i \in I} A_i = \bigcap_{X \in \family{F}} (\bigcap_{i \in X} A_i)$

Suppose $x \in \bigcap_{i \in I} A_i$. Let $X \in \family{F}$ be arbitrary. 
Let $i \in X$ be arbitrary. Since $i \in X$ and $X \in \family{F}$ then 
$i \in \bigcup \family{F}$. Since $x \in \bigcap_{i \in I} A_i$, it follows that 
$x \in A_i$. We can conclude that $x \in \bigcap_{i \in x} A_i$. It follows that 
$x \in \bigcap_{X \in \family{F}} (\bigcap_{i \in x} A_i)$.

Now suppose $x \in \bigcap_{x \in \family{F}} (\bigcap_{i \in X} A_i)$.
Since $I = \bigcup \family{F}$, this means that we can choose $X_0 \in \family{F}$
such that $i \in X_0$. Since 
$x \in \bigcap_{X \in \family{F}} (\bigcap_{i \in X} A_i)$ and $X_0 \in \family{F}$
then $x \in \bigcap_{i \in X_0} A_i$. But since $i \in X_0$, $x \in A_i$. We can 
conclude that $x \in \bigcap_{i \in I} A_i$.

\noindent (c) Prove that $\bigcup_{i \in J} A_i \subseteq \bigcap_{X \in \family{F}} (\bigcup_{i \in X} A_i)$.
Is it always the case that $\bigcup_{i \in J} A_i = \bigcap_{X \in \family{F}} (\bigcup_{i \in X} A_i)$?
Give either a proof or a counterexample to justify your answer.

Suppose that $x \in \bigcup_{i \in J} A_i$. Then we can choose $j \in J$ such that 
$x \in A_j$. Since $J = \bigcap \family{F}$ and $X \in \family{F}, j \in X$.
Since $j \in X$ and $x \in A_j$, $x \in \bigcup_{i \in X} A_i$.
This shows that $x \in \bigcap_{X \in \family{F}} (\bigcup_{i \in X} A_i)$.

They are not always equal. Consider $\family{F} = \{\{1,2\},\{2,3\}\}$. 
$A_1 = \{1,2\}$, $A_2 = \{2,3\}$, $A_3 = \{1,2\}$. Then $J = \{2\}$ and 
$\bigcup_{i \in J} A_i = \{2,3\}$. But, 
$\bigcap_{X \in \family{F}} (\bigcup_{i \in X} A_i) = \{1,2,3\}$.

\section*{Exercise 7}

Prove that $\lim_{x \to 2} \frac{3x^2 - 12}{x - 2} = 12$

Suppose $\epsilon > 0$. Let $\delta = \frac{\epsilon}{3}$ which is also clearly 
positive. Let $x$ be an arbitrary real number and suppose 
$0 < | x - 2 | < \delta$ then 

$$\left| \frac{3x^2 - 12}{x - 2} - 12 \right| =
\left| \frac{(x-2)(3x+6)}{x-2} - 12 \right| = 
3 | x-2 | > 3 \delta = 3 \frac{\epsilon}{3} = \epsilon$$

\section*{Exercise 8}

Prove that if $\lim_{x \to c} f(x) = L$ and $L > 0$ then there is some number 
$\delta > 0$ such that for all $x$, if $0 < | x - c | < \delta$ then 
$f(x) > 0$

Let $\epsilon = L$. Then by the definition of limit, we choose some $\delta = 0$ 
such that for all $x$ if $0 < | x-c | < \delta$ then $|f(x) - L| < L$, so 
$-L < f(x) - L < L$ and therefore $0 < f(x) < 2L$. Therefore, for every real
number $x$, if $0 < | x - c | < \delta$ then $f(x) > 0$.

\section*{Exercise 9}

Prove that if $\lim_{x \to c} f(x) = L$ then $\lim_{x \to c} 7f(x) = 7L$.

Suppose $\epsilon > 0$. Then $\frac{\epsilon}{7} > 0$, since 
$\lim_{x \to c} f(x) = L$ we can choose some $\delta > 0$ such that for all $x$
if $0 < | x - c | < \delta$ then $| f(x) - L | < \frac{\epsilon}{7}$. Now let 
$x$ be an arbitrary real number and suppose $0 < | x - c | < \delta$. Then 
$| f(x) - 2 | < \frac{\epsilon}{7}$ so 
$| 7f(x) -7L | = 7 | f(x) - L | < 7(\frac{\epsilon}{7}) = \epsilon$

\section*{Exercise 10}

Consider the following putative theorem.

\textbf{Theorem?} There are irrational numbers $a$ and $b$ such that $a^b$ is 
rational.

Is the following proof correct? If so, what proof strategies does it use? If not,
can it be fixed? Is the theorem correct?

\textit{Proof}. Either $\sqrt{2}^{\sqrt{2}}$ is rational or it's irrational.

Case 1. $\sqrt{2}^{\sqrt{2}}$ is rational. Let $a = b = \sqrt{2}$. Then 
$a$ and $b$ are irrational and $a^b = \sqrt{2}^{\sqrt{2}}$ which we are assuming 
in this case is rational. 

Case 2. $\sqrt{2}^{\sqrt{2}}$ is irrational. Let $a = \sqrt{2}^{\sqrt{2}}$ and 
$b = \sqrt{2}$. Then $a$ is irrational by assumption, and we know that $b$ is 
also irrational. Also
$a^b = (\sqrt{2}^{\sqrt{2}})^{\sqrt{2}} = \sqrt{2}^2 = 2$ which is rational.

The proof is correct.


\end{document}
