\documentclass[11pt]{article}
\usepackage{amssymb}
\usepackage{tipa}
\usepackage{amsmath}
\usepackage[scr]{rsfso}


\newcommand{\then}{\rightarrow} 
\newcommand{\bicond}{\leftrightarrow}
\newcommand{\powerset}[1]{\mathscr{P}(#1)}
\newcommand{\family}{\mathcal{F}}

\title{\textbf{How to Prove It} \\ {\Large\itshape Daniel J. Velleman} \\ {\Large\itshape Chapter 1: Operations on Sets}}

\author{\textbf{Nathaniel Curnick} \\ \textit{Textbook Solutions}}

\date{}

%----------------------------------------------------------------------------------------

\begin{document}

\maketitle

\section*{Exercise 1}

\noindent (a) Suppose $P \then Q$ and $Q \then R$ are both true. Prove that 
$P \then R$ is true.

Suppose $P$. If $P$, then $Q$. Since $Q$, then $R$. Therefore, if $P$, then $R$.

\noindent (b) Suppose $\neg R \then (P \then \neg Q)$ is true. Prove that 
$P \then (Q \then R)$ is true.

Suppose $P$. To prove $Q \then R$ we will prove the contrapositive. So, suppose 
$\neg R$. Since $\neg R$, then $P \then \neg Q$. Since $P$ then $\neg Q$. Thus,
if $Q \then R$. Thus $P \then (Q \then R)$.

\section*{Exercise 2}

\noindent (a) Suppose $P \then Q$ and $R \then \neg Q$ are both true. Prove that 
$P \then \neg R$ is true.

Suppose $P$. Since $P$ then $Q$. Since $R \then \neg Q$, by 
\textit{modus tollens} R. Thus $P \then \neg R$.

\noindent (b) Suppose that $P$ is true. Prove that 
$Q \then \neg (Q \then \neg P)$ is true.

Suppose $P$. Suppose also $Q$. Assume also $Q \then \neg P$. Since $Q$ therefore 
$\neg P$. But this contradicts $P$. Therefore $Q \then \neg (Q \then \neg P)$

\section*{Exercise 3}

Suppose $A \subseteq C$ and $B$ and $C$ ar disjoint. Prove that if $x \in A$
then $x \notin B$.

Suppose $A \subseteq C$. Also suppose $B \cap C = \emptyset$. This also 
necessarily means $x \in C \then x \notin B$. Suppose $x \in A$. Since 
$A \subseteq C$, then $x \in C$. But if $x \in C$, therefore $x \notin B$.

\section*{Exercise 4}

Suppose $A \setminus B$ is disjoint from $C$ and $x \in A$. Prove that if 
$x \in C$ then $x \in B$.

Suppose $A \setminus B \cap C = \emptyset$. This means 
$x \in C \then x \notin A \setminus B$. Also suppose $x \in A$ and $x \in C$.
Since $x \in C$, then $x \notin A \setminus B$. Since $x \in A$, in order for 
$x \notin A \setminus B$, then $x \in B$.

\section*{Exercise 5}

Prove that is can not be the case that $x \in A \ setminus B$ and 
$x \in B \setminus C$.

If $x \in B \setminus C$ then $x \in B$. If $x \in A \setminus B$ then 
$x \notin B$. This is a contradiction, so both can not be true at once.

\section*{Exercise 6}

Use the method of proof by contradiction to prove the theorem in Example 3.2.1.

Suppose $A \cap B \setminus B$ and $a \in C$. Prove that 
$a \notin A \setminus B$.

Suppose $a \in A \setminus B$, therefore $a \in A$ and $a \notin B$. Since 
$a \in C$ then $a \in A \cap C$. Since $A \cap C \subseteq B$ then $a \in B$.
However, this is a contradiction so $a \notin A \setminus B$.

\section*{Exercise 7}

Use the method of proof by contradiction to prove the theorem in Example 3.2.5

Suppose $A \subseteq B$, $a \in A$ and $a \notin B \setminus C$. Prove $a \in C$

Suppose $a \notin C$. Since $a \notin B \setminus C$, then $a \notin B$. 
$a \in A$, since $A \subseteq B$, then $a \in B$. This is a contradiction so 
$a \in C$.

\section*{Exercise 8}

Suppose that $y + x = 2y - x$ and $x$ and $y$ are not both zero. Prove that 
$y \neq 0$.

Rearrange to $2x = y$. Suppose $y = 0$, then $x = 0$. But since $x$ and $y$ 
are both not 0, then this is a contradiction. Therefore $y \neq 0$.

\section*{Exercise 9}

Suppose that $a$ and $b$ are nonzero real numbers. Prove that if 
$a < 1/a < b < 1/b$ then $a < -1$

Suppose $a \geq 1$. Thus $1 \geq 1/a$. Thus $a \geq 1/a$, which contradicts that 
$a < 1/a$. Thus $a < 1$. 

Suppose $a > 0$, then $1 < 1/a$, and since $1/a < b$ it follows that $b > 1$.
Thus $1 > 1/b$ and with $b > 1$ this implies that $b > 1/b$, but that 
contradicts that $b < 1/b$. Also, $a \neq 0$, so $a < 0$.

Suppose $a \geq -1$. We can conclude that $-1 \geq 1/a$. Combining $a \geq -1$ 
and $-1 \geq 1/a$ we get $a \geq 1/a$ contradicting that $a < 1/a$. Therefore 
$a < -1$

\section*{Exercise 10}

Suppose that $x$ and $y$ are real numbers. Prove that if $x^2 y = 2x + y$,
then if $y \neq 0$ then $x \neq 0$.

Assume that $x = 0$. Therefore, $0 \times y = 0 + y$, or in other words, 
$y = 0$. This proves the contrapositive, so when $y \neq 0$ then $x \neq 0$.


\section*{Exercise 11}

Suppose that $x$ and $y$ are real numbers. Prove that if $x \neq 0$, then 
if $y = (3x^2 + 2y)/(x^2 + 2)$ then $y = 3$.

Assume $x \neq 0$. Then, rearrange into $x^2 y = 3x^2$ and finally to $y = 3$. 

\section*{Exercise 12}

Consider the following incorrect theorem:

\textbf{Incorrect Theorem}. Suppose $x$ and $y$ are real numbers and
$x + y = 10$. Then $x \neq 3$ and $y \neq 8$.

\noindent (a) What's wrong with the following proof of the
theorem?

\textit{Proof}. Suppose the conclusion of the theorem is false. Then $x = 3$ 
and $y = 8$. But then $x + y = 11$, which contradicts the given information 
$x + y = 10$. Therefore, the conclusion must be true.

The issue is that $x + y = 10$ does not contradict $x + y = 11$.

\noindent (b) Show that the theorem is incorrect by finding a counterexample

$x = 3, y = 7$ is a suitable counterexample

\section*{Exercise 13}

Consider the following incorrect theorem:

\textbf{Incorrect Theorem}. Suppose that $A \subseteq B, B \subseteq C$ and
$x \in A$. Then $x \in B$.

\noindent (a) What's wrong with the following proof of the theorem?

\textit{Proof}. Suppose that $x \notin B$. Since $x \in A$ and $A \subseteq C$,
$x \in C$. Since $x \notin B$ and $B \subseteq C$, $x \notin C$. But now we have 
proven both $x \in C$ and $x \notin C$, so we have reached a contradiction.
Therefore $x \in B$.

The issue with this proof is in the reasoning 
$x \notin B \wedge B \subseteq C \then x \notin C$. $B$ is a subset of $C$, so 
while all the elements of $B$ are in $C$, not all of the elements of $C$ are 
necessarily in $B$.

\noindent (b) Show that the theorem is incorrect by finding a counterexample

$A = \{1,2,3,4\}, C=\{1,2,3,4,5\}, B=\{2,3,4,5\}$

\section*{Exercise 14}

Use truth tables to show that modus tollens is a valid rule of inference 

\begin{center}
    \begin{tabular}{ c c c  }
     $P$ & $Q$ & $\neg P \vee Q$\\ 
    T & T & T\\  
    T & F & F\\
    F & T & T\\  
    F & F & T
    \end{tabular}
\end{center}

\section*{Exercise 15}

Use truth tables to check the correctness of the theorem in Example 3.2.4

Suppose $P \then (Q \then R)$ prove $\neg R \then (P \then \neg Q)$

\begin{center}
    \begin{tabular}{ c c c c c }
     $P$ & $Q$ & $R$ & $(P \vee Q)$ & $(\neg P \vee R)$\\ 
    T & T & T & T & T \\  
    T & T & F & F & / \\
    T & F & T & T & T \\  
    T & F & F & T & T \\
    F & T & T & T & T \\  
    F & T & F & T & T \\
    F & F & T & T & T \\  
    F & F & F & T & T 
    \end{tabular}
\end{center}

\section*{Exercise 16}

Use truth tables to check the correctness of the statements in exercise 1

\noindent (a) If $P \then Q$ and $Q \then R$, prove $P \then R$.

\begin{center}
    \begin{tabular}{ c c c c c c }
     $P$ & $Q$ & $R$ & $\neg P \vee Q$ & $\neg Q \vee R$ & $\neg P \vee R$\\ 
    T & T & T & T & T & T\\  
    T & T & F & T & F & /\\
    T & F & T & F & T & /\\  
    T & F & F & F & T & /\\
    F & T & T & T & T & T\\  
    F & T & F & T & F & /\\
    F & F & T & F & T & T\\  
    F & F & F & F & T & T
    \end{tabular}
\end{center}

\noindent (b) If $\neg R \then (P \then Q)$ prove $P \then (Q \then R)$

\begin{center}
    \begin{tabular}{ c c c c c }
     $P$ & $Q$ & $R$ & $R \vee (\neg P \vee \neg Q)$ & $\neg P \vee (\neg Q \vee R)$\\ 
    T & T & T & T & T \\  
    T & T & F & F & / \\
    T & F & T & T & T \\  
    T & F & F & T & T \\
    F & T & T & T & T \\  
    F & T & F & T & T \\
    F & F & T & T & T \\  
    F & F & F & T & T 
    \end{tabular}
\end{center}

\section*{Exercise 17}

Use truth tables to check the correctness of the statements in exercise 2

\noindent (a) If $P \then Q$ and $R \then \neg Q$ prove $P \then \neg R$

\begin{center}
    \begin{tabular}{ c c c c c c }
     $P$ & $Q$ & $R$ & $\neg P \vee Q$ & $\neg R \vee \neg Q$ & $\neg P \vee \neg R$\\ 
    T & T & T & T & F & /\\  
    T & T & F & T & T & T\\
    T & F & T & F & T & /\\  
    T & F & F & F & T & /\\
    F & T & T & T & F & /\\  
    F & T & F & T & T & T\\
    F & F & T & T & T & T\\  
    F & F & F & T & T & T
    \end{tabular}
\end{center}

\noindent (b) Suppose $P$, then prove $Q \then \neg (Q \then \neg P)$

\begin{center}
    \begin{tabular}{ c c c  }
     $P$ & $Q$ & $\neg Q \vee \neg (\neg Q \vee \neg P) $\\ 
    T & T & T\\  
    T & F & T
    \end{tabular}
\end{center}

\section*{Exercise 18} 

Can the proof in Example 3.2.2 be modified to prove that if $x^2 + y = 13$ and 
$x \neq 3$ then $y \neq 4$? Explain

No, if we try and prove it that way around there is the counterexample of 
$x = -3$ and $y = 4$

\end{document}