\documentclass[11pt]{article}
\usepackage{amssymb}
\usepackage{tipa}
\usepackage{amsmath}
\usepackage[scr]{rsfso}


\newcommand{\then}{\rightarrow} 
\newcommand{\bicond}{\leftrightarrow}
\newcommand{\powerset}[1]{\mathscr{P}(#1)}
\newcommand{\family}[1]{\mathcal{#1}}

\title{\textbf{How to Prove It} \\ {\Large\itshape Daniel J. Velleman} \\ {\Large\itshape Chapter 3.5: Proofs Involving Disjunctions}}

\author{\textbf{Nathaniel Curnick} \\ \textit{Textbook Solutions}}

\date{}

%----------------------------------------------------------------------------------------

\begin{document}

\maketitle

\section*{Exercise 1}

Suppose $A$, $B$ and $C$ are sets. Prove that $A \cap (B \cup C) \subseteq (A \cap B)$

Consider $x \in A \cap B$, then $x \in A$ and $x \in B$. By definition, 
$x \in (A \cap B) \cup C$. Since $x \in B$, then $x \in B \cup C$. And since 
$x \in A$ also, then $x \in A \cap (B \cup C)$. Subce $x$ was arbitrary, we can 
conclude that $A \cap (B \cup C) \subseteq (A \cap B) \cup C$

\section*{Exercise 2}

Suppose $A$, $B$, and $C$ are sets. Prove that 
$(A \cup B) \setminus C \subseteq A \cup (B \setminus C)$.

Suppose $x \in B \setminus C$, then $x \in B$ and $x \notin C$. Therefore, 
$x \in A \cup (B \setminus C)$. Since $x \in B$, then $x \in A \cup B$ and 
since $x \notin C$, $x \in (A \cup B) \setminus C$. Since $x$ is arbitrary, we
can conclude that $(A \cup B) \setminus A \cup (B \setminus C)$

\section*{Exercise 3}

Suppose $A$ and $B$ are sets. Prove that $A \setminus (A \setminus B) = A \cap B$.

First prove $A \setminus (A \setminus B) \subseteq A \cap B$

Suppose $x \in A \setminus (A \setminus B)$ then $x \in A$ and 
$x \notin A \setminus B$. This means $x \in B$. Since $x \in A$ and $x \in B$, 
then $x \in A \cap B$. In other words $A \setminus (A \setminus B) \subseteq A \cap B$.

Now prove $A \cap B \subseteq A \setminus (A \setminus B)$

Suppose $x \in A \cap B$, then $x \in A$ and $x \in B$. Therefore, 
$x \notin A \setminus B$, so $x \in A \setminus (A \setminus B)$. Thus 
$A \cap B \subseteq A \setminus (A \setminus B)$.

Since we have proven both sets to be subsets of the other, then we have proven 
them to be equal.

\section*{Exercise 4}

Suppose $A$, $B$, and $C$ are sets. Prove that 
$A \setminus (B \setminus C) = (A \setminus B) \cup (A \cap C)$

First prove $A \setminus (B \setminus C) = (A \setminus B) \cup (A \cap C)$

Suppose $x \in A \setminus (B \setminus C)$. Thus 
$x \in A$ and $x \notin B \setminus C$. Suppoe $x \in B$, then $x \in C$. In this 
instance, $x \notin A \setminus B$, but $x \in A \cap C$, so 
$x \in (A \setminus B) \cup (A \cap C)$. Suppose now $x \notin B$, then 
$x \in A \setminus B$, so $x \in (A \setminus B) \cup (A \cap C)$. Thus 
$A \setminus (B \setminus C) \subseteq (A \setminus B) \cup (A \cap C)$.

Now prove $(A \setminus B) \cup (A \cap C) \subseteq A \setminus (B \setminus C)$

Suppose $x \in (A \setminus B) \cup (A \cap C)$. Suppose $x \in A$ and 
$x \notin B$, such that $x \in A \setminus B$. Since $x \in A$ and $x \notin B$,
then $x \in A \setminus (B \setminus C)$. Suppose instead that $x \in (A \cap C)$,
then $x \in A$ and $x \in C$. This means that $x \in A \setminus (B \setminus C)$,
therefore $(A \setminus B) \cup (A \cap C) \subseteq A \setminus (B \setminus C)$

Since we have proven both sets to be subsets of the other, then we have proven 
them to be equal.

\section*{Exercise 5}

Suppose $A \cap C \subseteq B \cap C$, and $A \cup C \subseteq B \cup C$. Prove 
that $A \subseteq B$.

Suppose $x \in A$. This gives two cases, either $x \in C$ or $x \notin C$.

For $x \in C$, then $x \in A \cap C$, and since $A \cap C \subseteq B \cap C$,
$x \in B \cap C$. Therefore, $x \in B$. For $x \notin B$, then $x \in A \cup C$,
and since $A \cup C \subseteq B \cup C$ and $x \notin C$ then $x \in B$. 

Either way, $x \in B$, so $A \subseteq B$.

\section*{Exercise 6}

Recall from Section 1.4 that the symmetric difference of two sets $A$ and $B$ is 
the set $A \triangle B = (A \setminus B) \cup (B \setminus A) = 
(A \cup B) \setminus (A \cap B)$. Prove that if $A \triangle B \subseteq A$ then 
$B \subseteq A$.

Suppose $B \nsubseteq A$. Let's choose some $x \in B$, $x \notin A$. This would 
mean $x \in (B \setminus A)$, and so $x \in (A \setminus B) \cup (B \setminus A)$.
However, $(A \setminus B) \cup (B \setminus A) \subseteq A$, so this would imply 
that $x \in A$, but this is a contradiction with our assumptions, so 
$B \subseteq A$.

\section*{Exercise 7}

Suppose $A$, $B$ and $C$ are sets. Prove that $A \cup C \subseteq B \cup C$ iff 
$A \subseteq C \subseteq B \setminus C$.

First we will prove $A \cup C \subseteq B \cup C \then A \subseteq C \subseteq B \setminus C$

Suppose $x \in A \setminus C$ which means $x \in A$ and $x \notin C$. $x \in A \cup C$,
and since $A \cup C \subseteq B \cup C$, then $x \in B \cup C$. Since $x \notin C$, then 
$x \in B$. Therefore $x \in B \setminus C$. Since $x \in A \setminus C$, and 
$x \in B \setminus C$, therefore $A \setminus C \subseteq B \setminus C$.

Now we will prove $A \subseteq C \subseteq B \setminus C \then A \cup C \subseteq B \cup C$

Since $A \setminus C \subseteq B \setminus C$, we know that $x \in A \setminus C$.
Thus $x \in A$ and $x \notin C$. Since $A \setminus C \subseteq B \setminus C$, then 
$x \in B \setminus C$. Since $x \notin C$, then $x \in B$. Since $x \in B$, then 
$x \in B \cup C$, and since $x \in A$, $x \in A \cup C$. 
Therefore, $A \cup C \subseteq B \cup C$

\section*{Exercise 8}

Prove that for any sets $A$ and $B$, $\powerset{A} \cup \powerset{B} \subseteq \powerset{A \cup B}$

Suppose $x \in \powerset{A \cup B}$. This means that $x \in A \cup B$. If 
$x \in A$ then $x \in \powerset{A}$, thus $x \in \powerset{A} \cup \powerset{B}$.
If on the other hand, $x \in B$, then $x \in \powerset{B}$, so 
$x \in \powerset{A} \cup \powerset{B}$.

Thus, $\powerset{A} \cup \powerset{B} \subseteq \powerset{A \cup B}$.

\section*{Exercise 9}

Prove that for any sets $A$ and $B$, if 
$\powerset{A} \cup \powerset{B} = \powerset{A \cup B}$
then either $A \subseteq B$ or $B \subseteq A$.

Consider $A$ and $B$ not being subsets of each other in any way. So, we have 
$x \in A$ which $x \notin B$ and $y \in B$ which $y \notin A$. This would mean
that at least one $X \subseteq \powerset{A \cup B}$ contains both $x$ and $y$.
However, $\powerset{A}$ contains no sets which have $y$ as an element, and 
conversely $\powerset{B}$ contains no sets which have $x$ as an element. 
Therefore, in order for $\powerset{A} \cup \powerset{B} = \powerset{A \cup B}$
then either $A \subseteq B$ or $B \subseteq A$.

\section*{Exercise 10}

Suppose $x$ and $y$ are real numbers and $x \neq 0$. 
Prove that $y + 1/x = 1 + y/x$ iff either $x = 1$ or $y = 1$.

First prove $y + 1/x = 1 + y/x \then x=1 \vee y=1$

$y - 1 = y / x - 1/x = (y - 1)/x$. If $y \neq 1$ then $y-1 \neq 0$ so we can 
divide both sides of the equation by $y-1$ and conclude that $1 = 1/x$ or $x=1$.
Thus either $x = 1$ or $y = 1$.

Now prove $x = 1 \vee y = 1 \then y + 1/x = 1 + y/x$

If $x = 1$ then $y + 1/1 = 1 + y/1 \equiv y + 1 = y + 1$. If $y = 1$ then 
$1 + 1/x = 1 + 1/x$

\section*{Exercise 11}

Prove that for every real number $x$, if $| x - 3 | > 3$ then $x^2 > 6x$

If $x - 3 \geq 0$ then $x - 3 > 3$ then $x > 6$. Thus $x^2 > 6x$.

If $x - 3 < 0$ then $3 - x > 3$, so $-x > 0$ or $x < 0$. Therefore 
$x^2 > 0 > 6x$

\section*{Exercise 12}

Prove that for every real number $x$, $| 2x - 6 | > x$ iff $| x - 4 | > 2$.

First prove $| 2x - 6 | > x \then | x - 4 | > 2$

If $2x - 6 \geq 0$ then $2x - 6 > x$ so $x > 6$. Thus $| x - 4 | > 2$. If 
$2x - 6 < 0$ then $6 - 2x > x$, so $x < 2$. Thus, $| x - 4 | > 2$.

Now prove $| x - 4 | > 2 \then | 2x - 6 | > x$

If $x - 4 \geq 0$ then $x - 4 > 2$, so $x > 6$, and so $| 2x - 6 > x$.
If $x - 4 < 0$ then $4 - x > 2$, so $x < -6$ and so $| 2x - 6 | > x$.

\section*{Exercise 13}

TODO

\section*{Exercise 14}

Prove that for every integer $x$, $x^2 + x$ is even.

If $x$ is even, $x^2$ is even, to which we add an even number, so $x^2+x$ is even

If $x$ is odd, then $x^2$ is odd, to which we add an odd number, so $x^2+x$ is odd

\section*{Exercise 15}

Prove that for every integer $x$, the remainder when $x^4$ is divided by 8 is 
either 0 or 1.

If $x$ is even then $x = 2m$. $(2m)^4 = 2^4 m^4$. $8 = 2^3$. So 
$\frac{2^4 m^4}{2^3} = 2m^4$ i.e. no remainder 

If $x$ is odd, then $x = 2m + 1$. 
$(2m + 1)^4 = 16m^4 + 32m^3 + 24m^2 + 8m + 1$. 

$$\frac{16m^4 + 32m^3 + 24m^2 + 8m}{8} = 8m^4 + 4m^3 + 3m^2 + m$$

with remainder 1

\section*{Exercise 16}

Suppose $\family{F}$ and $\family{G}$ are nonempty families of sets

\noindent (a) Prove that $\bigcup (\family{F} \cup \family{G}) = \bigcup \family{F} \cup \bigcup \family{G}$

First prove $\bigcup (\family{F} \cup \family{G}) \subseteq \bigcup \family{F} \cup \bigcup \family{G}$

Suppose $x \in \bigcup (\family{F} \cup \family{G})$. This means $x$ is in some 
$A$, which is in either $\family{F}$ or $\family{G}$. Whichever $A$ is an element of 
then $x \in \bigcup \family{F} \cup \bigcup \family{G}$, so 
$\bigcup (\family{F} \cup \family{G}) = \bigcup \family{F} \cup \bigcup \family{G}$.

Now prove $\bigcup \family{F} \cup \bigcup \family{G} \subseteq \bigcup (\family{F} \cup \family{G})$.

Suppose $x \in \bigcup \family{F}$, then $x$ is in some $A$ in $\family{F}$. 
Thus $A \in \family{F} \cup \family{G}$ and so $x \in \bigcup (\family{F} \cup \family{G})$.
Suppose now $x \in \bigcup \family{G}$. Then there is some $A$ in $\family{G}$.
Thus $A \in \family{F} \cup \family{G}$ and so $x \in \bigcup (\family{F} \cup \family{G})$.
Therefore $\bigcup \family{F} \cup \bigcup \family{G} \subseteq \bigcup (\family{F} \cup \family{G})$.

Since we have proven both sets to be subsets of the other, then we have proven 
them to be equal.

\noindent (b) Prove that $B \cup \bigcup \family{F} = \bigcup_{A \in \family{F}} (B \cup A)$.

First prove $B \cup \bigcup \family{F} \subseteq \bigcup_{A \in \family{F}} (B \cup A)$

If $x \in B$ then $x \in B \cup \bigcup \family{F}$.
Also, $x \in B \cup A$ where $A \in \family{F}$. 
If $x \in \bigcup \family{F}$, then $x \in A$, where $A \in \family{F}$. Since 
$x \in A$ then $x \in B \cup A$.Therefore 
$B \cup \bigcup \family{F} \subseteq \bigcup_{A \in \family{F}} (B \cup A)$

No prove $\bigcup_{A \in \family{F}} (B \cup A) \subseteq B \cup \bigcup \family{F}$

Suppose $x \in B$ then $x \in B \cup A$, and thus $x \in B \cup \bigcup \family{F}$.
If $x \in A$, then $x \in B \cup A$. Since $A \in \family{F}$, then 
$x \in \bigcup \family{F}$. Therefore $x \in B \cup \bigcup \family{F}$. 
So $\bigcup_{A \in \family{F}} (B \cup A) \subseteq B \cup \bigcup \family{F}$.

Since we have proven both sets to be subsets of the other, then we have proven 
them to be equal.

\noindent (c) Can you discover and prove a similar theorem about 
$\bigcap (\family{F} \cup \family{G})?$

$$\bigcap (\family{F} \cup \family{G}) = \bigcap \family{F} \cap \bigcap \family{G}$$

First prove $\bigcap (\family{F} \cup \family{G}) \subseteq \bigcap \family{F} \cap \bigcap \family{G}$

Suppose $x \in \bigcap (\family{F} \cup \family{G})$. This means there was some 
set $A \in \family{F}$ and some set $B \in \family{G}$ such that $x \in A$ and 
$x \in B$. Thus, $x \in \bigcap \family{F}$ and $x \in \bigcap \family{G}$.
Thus, $x \in \bigcap \family{F} \cap \bigcap \family{G}$. 
Therefore, $\bigcap (\family{F} \cup \family{G}) \subseteq \bigcap \family{F} \cap \bigcap \family{G}$.

Now prove $\bigcap \family{F} \cap \bigcap \family{G} \subseteq \bigcap (\family{F} \cup \family{G})$

Suppose $x \in \bigcap \family{F} \cap \bigcap \family{G}$. Then, 
$x \in \bigcap \family{F}$, and $x \in \bigcap \family{G}$. Suppose $x \in A$ 
where $A \in \family{F}$ and $x \in B$ where $B \in \family{G}$. Thus 
$A \in \family{F} \cup \family{G}$ (and $B \in \family{F} \cup \family{G}$), so 
$x \in \bigcap (\family{F} \cup \family{G})$. Thus 
$\bigcap \family{F} \cap \bigcap \family{G} \subseteq \bigcap (\family{F} \cup \family{G})$.

Since we have proven both sets to be subsets of the other, then we have proven 
them to be equal.

\section*{Exercise 17}

Suppose $\family{F}$ is a nonempty family of sets and $B$ is a set

\noindent (a) Prove that $B \cup \bigcup \family{F} = \bigcup (\family{F} \cup \{B\})$

First prove $B \cup \bigcup \family{F} \subseteq \bigcup (\family{F} \cup \{B\})$

Suppose $x \in B$, then $x \in B \cup \bigcup \family{F}$. Then $B$ obviously 
is in $\family{F} \{B\}$, and so $x \in \bigcup (\family{F} \cup \{B\})$, 
thus $B \cup \bigcup \family{F} \subseteq \bigcup (\family{F} \cup \{B\})$

Now prove $\bigcup (\family{F} \cup \{B\}) \subseteq B \cup \bigcup \family{F}$

Suppose $x \in \bigcup (\family{F} \cup \{B\})$. Suppose $x \in A$ where $A \in \family{F}$.
Thus $x \in \bigcup \family{F}$, and so $x \in B \cup \bigcup \family{F}$. Suppose 
now $x \in B$, thus $x \in B \cup \bigcup \family{F}$, so 
$\bigcup (\family{F} \cup \{B\}) \subseteq B \cup \bigcup \family{F}$.

\noindent (b) Prove that $B \cup \bigcap \family{F} = \bigcap_{A \in \family{F}} (B \cup A)$

First prove $B \cup \bigcap \family{F} \subseteq \bigcap_{A \in \family{F}} (B \cup A)$

Suppose $x \in B$, then $x \in B \cup A$, so $x \in \bigcap_{A \in \family{F}} (B \cup A)$.
Suppose $x \in \bigcap \family{F}$. Then $x \in A$, where $A \in \family{F}$.
Since $x \in A$ then $x \in B \cup A$, so $x \in \bigcap_{A \in \family{F}} (B \cup A)$.

Now prove $\bigcap_{A \in \family{F}} (B \cup A) \subseteq B \cup \bigcap \family{F}$.

Suppose $x \in \bigcap_{A \in \family{F}} (B \cup A)$. Suppose that $x \in B$,
then $x \in B \cup \bigcap \family{F}$. Suppose instead that $x \in A$. Since 
$A \in \family{F}$, then $x \in \bigcap \family{F}$. Thus $x \in B \cup \bigcap \family{F}$
Therefore $\bigcap_{A \in \family{F}} (B \cup A) \subseteq B \cup \bigcap \family{F}$.

Since we have proven both sets to be subsets of the other, then we have proven 
them to be equal.

\noindent (c) Can you discover and prove similar theorems about 
$B \cap \bigcup \family{F}$ and $B \cap \bigcap \family{F}$?

$$B \cap \bigcup \family{F} = \bigcup_{A \in \family{F}} (B \cap A)$$

First prove $B \cap \bigcup \family{F} \subseteq \bigcup_{A \in \family{F}} (B \cap A)$

Suppose $x \in B \cap \cup \family{F}$. Thus $x \in B$ and $x \in \bigcup \family{F}$.
Since $x \in \bigcup \family{F}$, $x \in A$ since $A \in \family{F}$. Since 
$x \in B$ and $x \in A$ then $x \in \bigcup_{A \in \family{F}} (B \cap A)$. So 
$B \cap \bigcup \family{F} \subseteq \bigcup_{A \in \family{F}} (B \cap A)$.

Now prove $\bigcup_{A \in \family{F}} (B \cap A) \subseteq B \cap \bigcup \family{F}$

Suppose $x \in \bigcup_{A \in \family{F}} (B \cap A)$. Since $A \in \family{F}$,
so $x \in \bigcup \family{F}$. Since $x \in B$ and $x \in \bigcup \family{F}$,
then $x \in B \cap \bigcup \family{F}$. So 
$\bigcup_{A \in \family{F}} (B \cap A) \subseteq B \cap \bigcup \family{F}$

Since we have proven both sets to be subsets of the other, then we have proven 
them to be equal.

\section*{Exercise 18}

Suppose $\family{F}$, $\family{G}$ and $\family{H}$ are nonempty families of 
sets and for every $A \in \family{F}$, and every $B \in \family{G}$, $A \cup B \in \family{H}$.
Prove that $\bigcap \family{H} \subseteq \bigcap \family{F} \cup \bigcap \family{G}$.

Suppose $x \in A$ and $y \in B$. $A \cup B \in \family{H}$. Suppose $A \in \family{H}$,
then $x \in \bigcap \family{H}$. Since $A \in \family{F}$, $x \in \bigcap \family{F}$
and thus $x \in \bigcap \family{F} \cup \bigcap \family{G}$. 
Suppose $B \in \family{H}$, then $y \in \bigcap \family{H}$. Since 
$B \in \family{G}$, then $y \in \bigcap \family{G}$, and thus 
$y \in \bigcap \family{F} \bigcup \bigcap \family{G}$. So 
$\bigcap \family{H} \subseteq \bigcap \family{F} \cup \bigcap \family{G}$.

\section*{Exercise 19}

Suppose $A$ and $B$ are sets. Prove that 
$\forall x ((x \in A \triangle B) \bicond (x \in A \bicond x \notin B))$

First prove $x \in A \triangle B \then x \in A \bicond x \notin B$

Suppose $x \in A \setminus B \cup B \setminus A$ then $x \in A \setminus B$ or 
$x \in B \setminus A$. In order for $x \in A \setminus B$, then $x \in A$ but 
iff $x \notin B$ i.e. $x \in A \bicond x \notin B$.

Now prove $x \in A \bicond x \notin B \then x \in A \triangle B$

If $x \in A$ and $x \notin B$, then $x \in A \setminus B$ and so 
$x \in A \setminus B \cup B \setminus A$, so $x \in A \triangle B$. However,
if $x \in A$ and $x \in B$ then this no longer holds (or indeed if $x \notin A$).

\section*{Exercise 20}

Suppose $A$, $B$ and $C$ are sets. Prove that $A \triangle B$ and $C$ are 
disjoint iff $A \cap C = B \cap C$

In other words we need to prove 
$A \triangle B \cap C = \emptyset \bicond A \cap C = B \cap C$

First prove $A \triangle B \cap C = \emptyset \then A \cap C = B \cap C $

If $x \in A \setminus B$ then $x \in A$ and $x \notin B$. In order to be 
disjoint, then $x \notin C$. So $x \notin A \cap C$ and $x \notin B \cap C$.
If $x \in B \setminus A$ then $x \in A$ and $x \notin B$. In order to be disjoint,
then $x \notin C$. So $x \notin A \cap C$ and $x \notin B \cap C$.
Suppose instead $x \in C$, then $x \notin A \setminus B \cup B \setminus A$.
Thus $x \in A \cap B$ or $x \notin A$ and $x \notin B$. Thus, either 
$x \in A \cap C$ and $x \in B \cap C$ or $x \notin A \cap C$ and $x \notin B \cap C$.

In all four cases, $A \cap C = B \cap C$.

Now prove $A \cap C = B \cap C \then A \triangle B \cap C = \emptyset$

If $x \in A \cap C$ then $x \in B \cap C$ so $x \in A \cap B \cap C$. This means 
$x \notin A \triangle B$ but $x \in C$ so $A \triangle B \cap C = \emptyset$.
If instead $A \notin A \cap C$ then $x \notin B \cap C$. If $x \in A \cap B$ but 
$x \notin C$ then $x \notin A \triangle B$ and $A \triangle B \cap C = \emptyset$.
If $x \in C$ then $x \notin A$ and $x \notin B$. So $x \notin A \triangle B$, 
so $A \triangle B \cap C = \emptyset$.

\section*{Exercise 21}

Suppose $A$, $B$, $C$ are sets. Prove that $A \triangle B \subseteq C$ iff 
$A \cup C = B \cup C$.

Prove $A \triangle B \subseteq C \then A \cup C = B \cup C$.

For $A \triangle B \subseteq C$, then $x \in A \triangle B$ and $x \in C$. There 
are two cases for $x \in A \triangle B$ to be considered. 
Consider $x \in A \setminus B$ where $x \in A$ and $x \notin B$. Thus 
$x \in A \cup C$ and $x \in B \cup C$. Consider not $x \in B \setminus A$, thus 
$x \in B$ and $x \notin A$, so still $x \in A \cup C$ and $x \in B \cup C$.
Thus, $A \cup C = B \cup C$.

Now prove $A \cup C = B \cup C \then A \triangle B \subseteq C$

Suppose $A \cup C = B \cup C$. Suppose $x \in A \triangle B$. Then, 
$x \in A \setminus B \cup B \setminus A$. Then either $x \in A \setminus B$
or $x \in B \setminus A$. Consider $x \in A \setminus B$ - this means 
$x \in A$ and $x \notin B$. Since $A \cup C = B \cup C$ then $x \in C$. 
Thus $A \triangle B \subseteq C$. Consider now $x \in B \setminus A$ - this means 
$x \in B$ and $x \notin A$. Since $A \cup C = B \cup C$ then $x \in C$. 
Thus $A \triangle B \subseteq C$.

\section*{Exercise 22}

Suppose $A$, $B$ and $C$ are sets. Prove that $C \subseteq A \triangle B$ iff 
$C \subseteq A \cup B$ and $A \cap B \cap C = \emptyset$.

In other words, prove $C \subseteq A \triangle B \bicond 
C \subseteq A \cup B \cap (A \cap B \cap C = \emptyset)$

First prove $C \subseteq A \triangle B \then 
C \subseteq A \cup B \cap (A \cap B \cap C = \emptyset)$

Suppose $x \in C$. Since $C \subseteq A \triangle B$, then $x \in A \triangle B$.
There are two cases. First consider $A \in A \setminus B$. This means $x in A$ and
$x \notin B$. So, $x \in A \cup B$, and since $x \in C$, then $C \subseteq A \cup B$.
But recall that $x \notin B$ so $A \cap B \cap C = emptyset$. Consider now 
$x \in B \setminus A$, which follows a symmetric argument where $x \in B$ and 
$x \notin A$ to reach the same result.

Now prove $C \subseteq A \cup B \cap (A \cap B \cap C = \emptyset) \then 
C \subseteq A \triangle B$

Suppose $x \in C$, then $x \in A \cup B$ too. Since $A \cap B \cap C = \emptyset$, 
and $x \in C$, then $x \in A$ or $x \in B$ but not both. Therefore we can consider 
two cases. First consdier $x \in A$ and $x \notin B$, so $x \in A \setminus B$,
so $x \in A \triangle B$, thus $C \subset A \triangle B$. Now consider 
$x \in B$ and $x \notin A$, and by symmetric argument $x \in B \setminus A$, 
so $x \in A \triangle B$ thus $C \subseteq A \triangle B$.

\section*{Exercise 23}

Suppose $A$, $B$ and $C$ are sets.

\noindent (a) Prove that $A \setminus B \subseteq (A \setminus B) \cup (B \setminus C)$

Suppose $x \in A \setminus C$. This means $x \in A$ and $x \notin C$. $B$ has two
cases. Consider first $x \in B$. This means $x \notin A \setminus B$ but 
$x \in B \setminus C$, so $x \in A \setminus B \cup B \setminus C$, so 
$A \setminus C \subseteq A \setminus B \cup B \setminus C$. Consider now $x \notin B$,
so $x \in A \setminus B$ but $x \notin B \setminus C$. Still, 
$x \in A \setminus B \cup B \setminus C$, so 
$A \setminus C \subseteq A \setminus B \cup B \setminus C$

\noindent (b) Prove that $A \triangle C \subseteq (A \triangle B) \cup (B \triangle C)$

$x \in A \triangle C$. This means either $x \in A \setminus C$ or $x \in c \setminus A$.
Consider $x \in A \setminus C$, so $x \in A$ and $x \notin C$. This has two cases to consider.
First, $x \in B$, thus $x \in B \setminus C$ so $x \in B \triangle C$. Thus 
$x \in A \triangle B \cup B \triangle C$. Second, $x \notin B$, in this case 
$x \in A \setminus B$, so $x \in A \triangle B$, thus $x \in A \triangle B \cup B \triangle C$.
Consider now $x \in C \setminus A$, so $x \in C$ and $x \notin A$. This has 
two cases to consider. First, $x \in B$, so $x \in A \triangle B$, so 
$x \in A \triangle B \cup B \triangle C$. 
Thus $A \triangle C \subseteq A \triangle B \cup B \triangle C$. Secondly, 
$x \notin B$, so $x \in B \triangle C$, so 
$x \in A \triangle B \cup B \triangle C$, thus 
$A \triangle C \subseteq A \triangle B \cup B \triangle C$

In all cases $A \triangle C \subseteq A \triangle B \cup B \triangle C$.

\section*{Exercise 24}

Suppose $A$, $B$ and $C$ are sets.

\noindent (a) Prove that $(A \cup B) \triangle C \subseteq (A \triangle C) \cup (B \triangle C)$

$x \in (A \cup B) \triangle C$ so 
$x \in (A \cup B) \setminus C \cup C \setminus (A \cup B)$.

There are three truth cases to consider. First consider $x \in A, x \notin C$.
Thus $x \in A \triangle C$ so $x \in A \triangle C \cup B \triangle C$ so 
$(A \cup B) \triangle C \subseteq (A \triangle C) \cup (B \triangle C)$. Second,
consider $x \in B$ and $x \notin C$, we use a symmetric argument as the first case
to arrive at the same conclusion.
Third, consider $x \in C$, $x \notin A$ and $x \notin B$. 
Then $x \in A \triangle C$ and $x \in B \triangle C$, so 
$x \in A \triangle C \cup B \triangle C$, so 
$(A \cup B) \triangle C \subseteq A \triangle C \cup B \triangle C$

\noindent (b) Find an example of sets $A$, $B$ and $C$ such that 
$(A \cup B) \triangle C \neq (A \triangle C) \cup (B \triangle C)$.

One example is $A = \{1\}, B = \{2\}, C = \{3\}$.

\section*{Exercise 25} 

Suppose $A$, $B$ and $C$ are sets 

\noindent (a) Prove that 
$(A \triangle C) \cap (B \triangle C) \subseteq (A \cap B) \triangle C$

Suppose $x \in (A \triangle C \cap B \triangle C)$, so $x$ needs to be in both 
$A \triangle C$ and $B \triangle C$. So, either $x \in C, x \notin A, x \notin B$
or $x \in A, x \in B, x \notin C$. 

Consider first $x \in C, x \notin A, x \notin B$. Thus, 
$x \in C \setminus (A \cap B)$, so $x \in (A \cap B) \triangle C$.

Consider now $x \in A, x \in B, x \notin C$. $x \in (A \cap B) \setminus C$ 
so $x \in (A \cap B) \triangle C$. 

Both cases have been covered and we can see that 
$(A \triangle C) \cap (B \triangle C) \subseteq (A \cap B) \triangle C$

\noindent (b) Is it always true that 
$(A \cap B) \triangle C \subseteq (A \triangle C) \cap (B \triangle C)$? Give 
either a proof or a counterexample.

No, a counterexample is $A = \{1\}, B = \{2\}, C = \{1,2\}$

\section*{Exercise 26}

Suppose $A$, $B$ and $C$ are sets. Consider the sets $(A \setminus B) \triangle C$
and $(A \triangle C) \setminus (B \triangle C)$. Can you prove either is a subset 
of the other? Justify your conclusions with either proofs or counterexamples.

Let's prove $(A \triangle C) \setminus (B \triangle C)$

Suppose $x \in (A \triangle C) \setminus (B \triangle C)$, so $x \in A \triangle C$
and $x \notin B \triangle C$. In order for this to be true, there are two cases 
$x \in A, x \notin B, x \notin C$ or $x \in C, x \in B, x \notin A$.

Consider $x \in A, x \notin B, x \notin C$. $x \in A \setminus B$, so 
$x \in (A \setminus B) \setminus C$, thus $x \in (A \setminus B) \triangle C$.

Consider $x \in C, x \in B, x \notin A$. $x \in C \setminus (A \setminus B)$,
so $x \in (A \setminus B) \triangle C$.

In all possible cases we see $(A \triangle C) \setminus (B \triangle C)$

\section*{Exercise 27}

Conside the follow putative theorem.

\textbf{Theorem?} For every real number $x$, if $| x - 3 | < 3$ then $0 < x < 6$

Is the following proof correct? If so, what proof strategies does it use? If not, 
can it be fixed? Is the theorem correct?

\textit{Proof.} Let $x$ be an arbitrary real number, and cuppose $| x - 3 | < 3$.
We consider two cases

Case 1. $x - 3 \geq 0$. Then $| x - 3 | = x - 3$. Plugging this into the 
assumption that $| x - 3 | < 3$, we get $x - 3 < 3$, so clearly $x < 6$

case 2. $x - 3 < 0$. Then $| x - 3 | = 3 - x$, so the assumption $| x - 3 | < 3$
means that $3 - x < 3$. Therefore $3 < 3 + x$, so $0 < x$.

Since we have proven both $0 < x$ and $x < 6$, we can conclude that $0 < x < 6$

\par\noindent\rule{\textwidth}{0.4pt}

The proof is incorrect, since it only establishes $0 < x$ or $x < 6$, but we 
need to prove $0 < 6$ and $x < 6$. It can be fixed though. In case 1 we 
establish that $x < 6$, but also since $x - 3 > 0$, then $x > 0$. Case 2
already shows that $x > 0$ but also $3 - x < 3$, so $x < 6$. Thus the proof is 
fixed.

\section*{Exercise 28}

Consider the following putative theorem.

\textbf{Theorem?} For any sets $A$, $B$ and $C$ if $A \setminus B \subseteq C$
and $A \nsubseteq C$ then $A \cap B \neq \emptyset$.

Is the following proof correct? If so, what proof strategies does it use? If not, 
can it be fixed? Is the theorem correct?

\textit{Proof.} Supppose $A \setminus B \subseteq C$ and $A \nsubseteq C$. 
Since $A \nsubseteq C$, we can choose some $x$ such that $x \in A$ and $x \notin C$.
Since $x \notin C$ and $A \setminus B \subseteq C$, $x \notin A \setminus B$.
Therefore either $x \notin A$ or $x \in B$. But we already know that $x \in A$,
so it follows that $x \in B$. Since $x \in A$ and $x \in B$, $x \in A \cap B$.
Therefore $A \cap B \neq \emptyset$.

\par\noindent\rule{\textwidth}{0.4pt}

The theory is correct % TODO: Elaborate 

\section*{Exercise 29}

Consider the following putative theorem 

\textbf{Theorem?} $\forall x \in \mathbb{R} \exists y \in \mathbb{R} ( x y^2 \neq y - x)$

Is the following proof correct? If so, what proof strategies does it use? If not, 
can it be fixed? Is the theorem correct?

\textit{Proof.} Let $x$ be an arbitrary real number.

Case 1. $x = 0$. Let $y = 1$. Then $xy^2 = 0$ and $y - x = 1 - 0 = 1$ 
so $xy^2 \neq y - x$

Case 2. $x = 0$. Let $y = 0$. Then $xy^2 = 0$ and $y - x \neq -x = 0$, 
so $xy^2 \neq y - x$

Since these cases are exhaustive, we have shown that 
$\exists y \in \mathbb{R} (xy^2 \neq y - x)$. Since $x$ was arbitrary, this shows 
that $\forall x \in \mathbb{R} \exists y \in \mathbb{R} (x y^2 \neq y - x)$

\par\noindent\rule{\textwidth}{0.4pt}

The proof is correct % TODO: Elaborate 

\section*{Exercise 30}

Prove that if $\forall x P(x) \then \exists x Q(x)$ then 
$\exists x (P(x) \then Q(x))$

Demonstrate that LHS is equivalent to RHS

$$\neg \forall x P(x) \vee \exists x Q(x)$$
$$\exists x \neg P(x) \vee \exists x Q(x)$$
$$\exists x (\neg P(x) \vee Q(x))$$
$$\exists x (P(x) \then Q(x))$$

\section*{Exercise 31}

Consider the following putative theorem.

\textbf{Theorem?} Suppose $A$, $B$, and $C$ are sets and $A \subseteq B \cup C$.
Then either $A \subseteq B$ or $A \subseteq C$.

Is the following proof correct? If so, what proof strategies does it use? If not, 
can it be fixed? Is the theorem correct?

\textit{Proof.} Let $x$ be an arbitrary element of $A$. 
Since $A \subseteq B \cup C$ it follows that $x \in B$ or $x \in C$.

Case 1. $x \in B$. Since $x$ was an arbitrary element of $A$, it follows that 
$\forall x \in A(x \in B)$, which means that $A \subseteq B$.

Case 2. $x \in C$. Similarly, since $x$ was an arbitrary element of $A$, we can 
conclude that $A \subseteq C$.

Thus either $A \subseteq B$ or $A \subseteq C$.

\par\noindent\rule{\textwidth}{0.4pt}

The proof is incorrect. A counterexample is $A = \{1,2\}, B = \{1\}, C = \{2\}$.

\section*{Exercise 32}

Suppose $A$, $B$ and $C$ are sets $A \subseteq B \cup C$. Prove that either 
$A \subseteq B$, or $A \cap B \neq \emptyset$.

Suppose $x \in A$, so $x \in B \cup C$. Consider the case $x \in B, x \notin C$.
Since $x \in A$ and $x \in B$ then $A \subseteq B \cup C$. Consider the case 
$x \notin B, x \in C$. Since $x \in A$ and $x \in C$, then $x \in A \cap C$, 
so $A \cap C \neq \emptyset$

\section*{Exercise 33}

Prove $\exists x (P(x) \then \forall y P(y))$

Case 1. $\forall y P(y)$ is true. This means that there was some 
$P(a) \then \forall y P(y)$. Thus $\exists x (P(x) \then \forall y P(y))$.

Case 2. $\neg \forall y P(y)$ then $\exists x \neg P(x)$, so we can choose 
some object $a$ such that $P(a)$ is false. Therefore $P(a) \then \forall y P(y)$
is true, so $\exists x (P(x) \then \forall y P(y))$.


\end{document}
