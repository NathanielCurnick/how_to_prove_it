\documentclass[11pt]{article}
\usepackage{amssymb}
\usepackage{tipa}
\usepackage{amsmath}
\usepackage[scr]{rsfso}


\newcommand{\then}{\rightarrow} 
\newcommand{\bicond}{\leftrightarrow}
\newcommand{\powerset}[1]{\mathscr{P}(#1)}
\newcommand{\family}[1]{\mathcal{#1}}

\title{\textbf{How to Prove It} \\ {\Large\itshape Daniel J. Velleman} \\ {\Large\itshape Chapter 3.5: Proofs Involving Disjunctions}}

\author{\textbf{Nathaniel Curnick} \\ \textit{Textbook Solutions}}

\date{}

%----------------------------------------------------------------------------------------

\begin{document}

\maketitle

\section*{Exercise 1}

Suppose $A$, $B$ and $C$ are sets. Prove that $A \cap (B \cup C) \subseteq (A \cap B)$

Consider $x \in A \cap B$, then $x \in A$ and $x \in B$. By definition, 
$x \in (A \cap B) \cup C$. Since $x \in B$, then $x \in B \cup C$. And since 
$x \in A$ also, then $x \in A \cap (B \cup C)$. Subce $x$ was arbitrary, we can 
conclude that $A \cap (B \cup C) \subseteq (A \cap B) \cup C$

\section*{Exercise 2}

Suppose $A$, $B$, and $C$ are sets. Prove that 
$(A \cup B) \setminus C \subseteq A \cup (B \setminus C)$.

Suppose $x \in B \setminus C$, then $x \in B$ and $x \notin C$. Therefore, 
$x \in A \cup (B \setminus C)$. Since $x \in B$, then $x \in A \cup B$ and 
since $x \notin C$, $x \in (A \cup B) \setminus C$. Since $x$ is arbitrary, we
can conclude that $(A \cup B) \setminus A \cup (B \setminus C)$

\section*{Exercise 3}

Suppose $A$ and $B$ are sets. Prove that $A \setminus (A \setminus B) = A \cap B$.

First prove $A \setminus (A \setminus B) \subseteq A \cap B$

Suppose $x \in A \setminus (A \setminus B)$ then $x \in A$ and 
$x \notin A \setminus B$. This means $x \in B$. Since $x \in A$ and $x \in B$, 
then $x \in A \cap B$. In other words $A \setminus (A \setminus B) \subseteq A \cap B$.

Now prove $A \cap B \subseteq A \setminus (A \setminus B)$

Suppose $x \in A \cap B$, then $x \in A$ and $x \in B$. Therefore, 
$x \notin A \setminus B$, so $x \in A \setminus (A \setminus B)$. Thus 
$A \cap B \subseteq A \setminus (A \setminus B)$.

Since we have proven both sets to be subsets of the other, then we have proven 
them to be equal.

\section*{Exercise 4}

Suppose $A$, $B$, and $C$ are sets. Prove that 
$A \setminus (B \setminus C) = (A \setminus B) \cup (A \cap C)$

First prove $A \setminus (B \setminus C) = (A \setminus B) \cup (A \cap C)$

Suppose $x \in A \setminus (B \setminus C)$. Thus 
$x \in A$ and $x \notin B \setminus C$. Suppoe $x \in B$, then $x \in C$. In this 
instance, $x \notin A \setminus B$, but $x \in A \cap C$, so 
$x \in (A \setminus B) \cup (A \cap C)$. Suppose now $x \notin B$, then 
$x \in A \setminus B$, so $x \in (A \setminus B) \cup (A \cap C)$. Thus 
$A \setminus (B \setminus C) \subseteq (A \setminus B) \cup (A \cap C)$.

Now prove $(A \setminus B) \cup (A \cap C) \subseteq A \setminus (B \setminus C)$

Suppose $x \in (A \setminus B) \cup (A \cap C)$. Suppose $x \in A$ and 
$x \notin B$, such that $x \in A \setminus B$. Since $x \in A$ and $x \notin B$,
then $x \in A \setminus (B \setminus C)$. Suppose instead that $x \in (A \cap C)$,
then $x \in A$ and $x \in C$. This means that $x \in A \setminus (B \setminus C)$,
therefore $(A \setminus B) \cup (A \cap C) \subseteq A \setminus (B \setminus C)$

Since we have proven both sets to be subsets of the other, then we have proven 
them to be equal.

\section*{Exercise 5}

Suppose $A \cap C \subseteq B \cap C$, and $A \cup C \subseteq B \cup C$. Prove 
that $A \subseteq B$.

Suppose $x \in A$. This gives two cases, either $x \in C$ or $x \notin C$.

For $x \in C$, then $x \in A \cap C$, and since $A \cap C \subseteq B \cap C$,
$x \in B \cap C$. Therefore, $x \in B$. For $x \notin B$, then $x \in A \cup C$,
and since $A \cup C \subseteq B \cup C$ and $x \notin C$ then $x \in B$. 

Either way, $x \in B$, so $A \subseteq B$.

\section*{Exercise 6}

Recall from Section 1.4 that the symmetric difference of two sets $A$ and $B$ is 
the set $A \triangle B = (A \setminus B) \cup (B \setminus A) = 
(A \cup B) \setminus (A \cap B)$. Prove that if $A \triangle B \subseteq A$ then 
$B \subseteq A$.

Suppose $B \nsubseteq A$. Let's choose some $x \in B$, $x \notin A$. This would 
mean $x \in (B \setminus A)$, and so $x \in (A \setminus B) \cup (B \setminus A)$.
However, $(A \setminus B) \cup (B \setminus A) \subseteq A$, so this would imply 
that $x \in A$, but this is a contradiction with our assumptions, so 
$B \subseteq A$.

\section*{Exercise 7}

Suppose $A$, $B$ and $C$ are sets. Prove that $A \cup C \subseteq B \cup C$ iff 
$A \subseteq C \subseteq B \setminus C$.

First we will prove $A \cup C \subseteq B \cup C \then A \subseteq C \subseteq B \setminus C$

Suppose $x \in A \setminus C$ which means $x \in A$ and $x \notin C$. $x \in A \cup C$,
and since $A \cup C \subseteq B \cup C$, then $x \in B \cup C$. Since $x \notin C$, then 
$x \in B$. Therefore $x \in B \setminus C$. Since $x \in A \setminus C$, and 
$x \in B \setminus C$, therefore $A \setminus C \subseteq B \setminus C$.

Now we will prove $A \subseteq C \subseteq B \setminus C \then A \cup C \subseteq B \cup C$

Since $A \setminus C \subseteq B \setminus C$, we know that $x \in A \setminus C$.
Thus $x \in A$ and $x \notin C$. Since $A \setminus C \subseteq B \setminus C$, then 
$x \in B \setminus C$. Since $x \notin C$, then $x \in B$. Since $x \in B$, then 
$x \in B \cup C$, and since $x \in A$, $x \in A \cup C$. 
Therefore, $A \cup C \subseteq B \cup C$


\end{document}
